\begin{exerciseS}[Linee di corrente, traiettorie e linee di fumo]
 Sia dato il campo di moto
\begin{equation}
 \bm{u}(x,y) = 2Ax \bm{\hat{x}} - 2Ay \bm{\hat{y}} 
\end{equation}
Calcolare l'equazione delle linee di corrente, delle traiettorie e delle linee di fumo (curve di emissione) e disegnarle.
\end{exerciseS}


\sol

\partone Definizione di linee di corrente, traiettorie, linee di fumo, tracce. Soluzione di sistemi di equazioni differenziali ordinarie (problemi di Cauchy, ai valori iniziali).

\parttwo
Partendo dalle definizioni, si ricavano le equazioni delle curve caratteristiche. 
\begin{itemize}
\item \textbf{Linee di corrente.} Dalla scrittura in componenti della definizione di linee di corrente si ottiene il sistema 
\begin{equation}
 \begin{cases}
  \dfrac{dX}{dp} = \lambda(p) 2 A X \\
  \dfrac{dY}{dp} = - \lambda(p) 2 A Y \ , \\
 \end{cases}
%  \quad\Rightarrow\quad
%  \frac{dX}{dY} = -\frac{X}{Y}
%  \quad\Rightarrow\quad
%  \ln X = -\ln Y + c 
%  \quad\Rightarrow\quad
%  X Y = c
\end{equation}
risolvibile ad esempio ricavando $\lambda(p) = \frac{X'(p)}{2 A X(p)}$ dalla prima equazione e inserendolo nella seconda. Integrando tra $p_0$ e $p$, dopo aver semplificato i fattori $2 A$, si ottiene (derivare per credere)
\begin{equation}
 0 = \int_{p_0}^{p} \left( \dfrac{X'(p')}{X(p')} + \dfrac{Y'(p')}{Y(p')} \right) dp' =
 \ln{\dfrac{X(p)}{X(p_0)}} + \ln{\dfrac{Y(p)}{Y(p_0)}}
\end{equation} \vspace{-0.5cm}
\begin{equation}
 \quad \rightarrow \quad
 X(p)Y(p) = X(p_0)Y(p_0)
\end{equation}
Le linee di corrente appena ricavate sono delle iperboli equilatere con gli asintoti coincidenti con gli assi. Nel procedimento svolto, per poter dividere per $X(p)$ e $Y(p)$ dobbiamo imporre la condizione che $X(p)$, $Y(p)$ siano diversi da zero. Nella ricerca degli equilibri del sistema, si nota che
\begin{itemize}
 \item il punto $(x,y) = (0,0)$ è l'unico punto di equilibrio del sistema, punto di ristagno del campo di velocità;
 \item gli assi coordinati coincidono con linee di corrente: la derivata $dX/dp$ è nulla quando $X=0$ (se la parametrizzazione della curva è regolare, cioè $\lambda(p) \ne 0$); la derivata $dY/dp$ è nulla quando $Y=0$ (se la parametrizzazione della curva è regolare, cioè $\lambda(p) \ne 0$). Nel primo caso, la linea di corrente coincide con l'asse $y$, avendo coordinata $X=0$ costante e coordinata $Y(p)$ descritta dalla seconda equazione; nel secondo caso, la linea di corrente coincide con l'asse $x$, avendo coordinata $Y=0$ costante e coordinata $X(p)$ descritta dalla prima equazione.
\end{itemize}

\item \textbf{Traiettorie.}
\begin{equation}
 \begin{cases}
  \dfrac{dx}{dt} = 2 A x(t) \\
  \dfrac{dy}{dt} = -  2 A y(t) \\
  x(t_0) = x_0 , \quad y(t_0) = y_0
 \end{cases}
 \quad\rightarrow\quad
 \begin{cases}
  x(t;\bm{r_0},t_0) = x_0 e^{2A(t-t_0)} \\
  y(t;\bm{r_0},t_0) = y_0 e^{-2A(t-t_0)} \\
 \end{cases}
\end{equation}
\paragraph{Osservazione.} Per ricavare la forma cartesiana dell'equazione delle traiettorie bisogna esplicitare il parametro $t$ in funzione di una delle due coordinate e inserire la formula ottenuta nell'equazione delle altre componenti. In questo caso è possibile eliminare la dipendenza da $t$, moltiplicando tra di loro le componenti delle traiettorie e ottenendo $x y = x_0 y_0$: si osserva l'equazione delle traiettorie coincide con l'equazione delle linee di corrente per il campo di velocità considerato.
Le linee di corrente coincidono con le linee di corrente e le linee di fumo nel caso in cui il \textbf{campo di veloictà} è \textbf{stazionario}: in questo caso, il sistema differenziale con il quale si ricavano linee di corrente e linee di fumo è \textbf{autonomo}, cioè il termine forzante non dipende esplicitamente dal tempo. La soluzione di un problema differenziale di un sistema autonomo non dipende dal tempo $t$ in sè, ma dalla differenza tra il tempo $t$ e il tempo al quale viene imposta la condizione iniziale $t_0$: nella formula parametrica delle traiettorie, $t$ e $t_0$ compaiono sempre come differenza $t-t_0$ e mai ``in altre forme'', come ad esempio nell'esercizio precedente, nel quale il campo di moto non è stazionario. Per questo motivo si arriva alla stessa equazione in forma cartesiana per le traiettorie e le linee di fumo, dopo aver esplicitato rispettivamente $t$ e $t_0$ in funzione di una coordinata e aver inserito questa espressione nelle formule delle altre componenti.

\item \textbf{Linee di fumo.} Da quanto riportato nel punto e nell'osservazione precedenti, è immediato ricavare sia la forma parametrica delle linee di fumo,
\begin{equation}
 \begin{cases}
  x(t_0;t,\bm{r_0}) = x_0 e^{2A(t-t_0)} \\
  y(t_0;t,\bm{r_0}) = y_0 e^{-2A(t-t_0)} \\
 \end{cases}
\end{equation}
sia la forma cartesiana, $ x y = x_0 y_0$.



\end{itemize}





