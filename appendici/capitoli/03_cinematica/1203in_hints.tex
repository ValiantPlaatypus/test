\begin{exerciseS}[Linee di corrente, traiettorie e linee di fumo: 3D]
 Sia dato il campo di moto
\begin{equation}
 \bm{u}(x,y,z) = \frac{x}{x^2 + y^2 + z^2} \bm{\hat{x}} +
                 \frac{y}{x^2 + y^2 + z^2} \bm{\hat{y}} +
                 \frac{z}{x^2 + y^2 + z^2} \bm{\hat{z}}
\end{equation}
Calcolare l'equazione delle linee di corrente, delle traiettorie e delle linee di fumo (curve di emissione) e disegnarle.
\end{exerciseS}


% \sol
% 
% \partone Definizione di linee di corrente, traiettorie, linee di fumo, tracce. Soluzione di sistemi di equazioni differenziali.

%Le linee di corrente sono curve $\bm{S}$ tangenti al campo vettoriale $\bm{u}(\bm{r},t)$ in ogni punto dello spazio $\bm{r}$ e per ogni istante temporale $t$. Essendo curve (1 dimensione), possono essere espresse in forma parametrica, come funzioni di un parametro scalare $p$.
%\begin{equation}
% \frac{d\bm{S}(p)}{dp} = \bm{u}(\bm{S},t)
%\end{equation}

%Le traiettorie descrivono il moto della singola particella fluida e sono descritte dall'equazione
%\begin{equation}
%\begin{cases}
% \frac{d\bm{R}(t)}{dt} = \bm{u}(\bm{R},t) \\
% \bm{R}(0) = \bm{R}_0
%\end{cases}
%\end{equation}

%Le linee di fumo sono un modo per tracciare tutte le particelle di fluido passate per un determinato punto nello spazio a diversi istanti temporali. La loro equazione è:
%\begin{equation}
%\begin{cases}
% \frac{d\bm{R}(t)}{dt} = \bm{u}(\bm{R},t) \\
% \bm{R}(\tau) = \bm{\bar{R}}
%\end{cases}
%\end{equation}

% \parttwo
\textbf{Suggerimento.}
Per risolvere l'esercizio in maniera semplice, si osservi che il campo di moto è stazionario e ha simmetria sferica: è quindi conveniente usare un sistema di coordiante sferiche.

% Il campo di moto scritto in coordinate sferiche $(R,\theta,\phi)$ è
% \begin{equation}
%   \bm{u}(R,\theta,\phi) = \dfrac{\bm{\hat{R}}}{R}
% \end{equation}

% Le linee di corrente, le traiettorie e le linee di fumo sono delle rette
% uscenti dall'origine.

% \begin{itemize}
% \item Linee di corrente.
%  \dots
% \item Traiettorie.
%  \dots
% \item Linee di fumo.
%  \dots
% 
% \end{itemize}
