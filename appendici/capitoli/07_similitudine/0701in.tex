\noindent
\begin{exerciseS}[Similitudine ad alta velocità: missile]
Un missile vola alla quota di $7000\  m$, dove la densit\`{a} 
dell'aria \`{e} $\rho = 0.59\  kg/m^3$ e la sua temperatura 
\`e $T=-30.45^\circ   C$, alla velocit\`{a} costante $V_v=505\  km/h$.

Determinare:
\begin{itemize}
  \item il fattore di scala geometrico $\lambda=L_m/L_v$,
  \item la velocit\`{a} dell'aria $V_m$,
\end{itemize}
necessari per riprodurre correttamente i coefficienti aerodinamici 
del missile in una galleria del vento che operi a condizioni 
atmosferiche standard ($\rho=1.225\ kg/m^3$, $p=101325\ Pa$,
$T=15^\circ  C$).
    
($V_m =152.8\  m/s$, $\lambda=0.507$)
\end{exerciseS}

\sol
%
\partone
Similitudine fluidodinamica: numeri di Reynolds e di Mach.
\begin{equation}
 Re = \frac{\rho U L}{\mu} \quad , \quad M = \frac{U}{c} \ .
\end{equation}
Formula di Sutherland per la viscosità dinamica \textbf{dei gas},
\begin{equation}
 \mu(T) = \mu_0 \displaystyle\left(\frac{T}{T_0}\right)^{1.5}
 \frac{C+T_0}{C+T} \ .
\end{equation}
%
\parttwo
Assumendo che l'aria si comporti come gas ideale, per il quale vale l'equazione di stato $p = \rho R T$, la velocità del suono vale $c = \sqrt{\gamma R T}$, dove $\gamma = c_p / c_v$ è il rapporto dei calori specifici a pressione e volume costante, che vale $\gamma = 1.4$ per un gas biatomico. La costante del gas $R$ è definita come il rapporto tra la costante universale dei gas $\mathscr{R}$ e la massa molare $M_m$, $R = \mathscr{R}/M_m$.
La massa molare dell'aria secca vale $M_m = 28.96 \ kg / kmol$ e la sua costante $R$ vale
\begin{equation}
 R = \dfrac{\mathscr{R}}{M_m} = \dfrac{8314.4 \ J / (kmol \ K)}{28.97 \ kg/kmol} = 287.0 \dfrac{J }{kg \ K} \ .
\end{equation}
La velocità del suono nell'aria alle condizioni termodinamiche del problema vale $c = 312.3 \ m/s$. Il numero di Mach caratteristico della corrente è quindi $M=0.45$ e gli effetti di comprimibilità non possono essere trascurati, poichè il numero di Mach è maggiore della valore convenzionale $0.3$ che identifica il limite della validità dell'approssimazione di fluido incomprimibile.
%
Per ottenere la similitudine tra problema reale e quello modellato (di dimensioni ridotte) è necessaria la similitudine geometrica e l'uguaglianza dei numeri adimensionali che caratterizzano il problema, il numero di Reynolds $Re$ e il numero di Mach $M$.
%Se si fa l'ipotesi di gas ideale perfetto, la velocità del suono può essere scritta in funzione della sola variabile termodinamica temperatura $c = \sqrt{\gamma R T}$, con $\gamma$ rapporto dei calori specifici a pressione e volume costanti, $R$ costante universale del gas ($R = \mathscr{R}/M_m$).
\begin{equation}
\begin{cases}
 M_1 = M_2 \\
 Re_1 = Re_2  
\end{cases} \ .
\end{equation}
Utilizzando l'equazione di stato dei gas perfetti,
\begin{equation}
\begin{cases}
 \dfrac{V_v}{\sqrt{\gamma R T_v}} = \dfrac{V_m}{\sqrt{\gamma R T_m}} \\
 \dfrac{\rho_v V_v L_v}{\mu(T_v)} = \dfrac{\rho_m V_m L_m}{\mu(T_m)}
\end{cases}
\end{equation}
%
Risolvendo il sistema, si ottiene l'espressione delle incognite
\begin{equation}
% \Rightarrow
   \begin{cases}
     V_m = V_v \sqrt{\dfrac{T_m}{T_v}} \\
     \lambda = \dfrac{L_m}{L_v} = \dfrac{\rho_v}{\rho_m} 
     \sqrt{\dfrac{T_v}{T_m}} \dfrac{\mu(T_m)}{\mu(T_v)}
   \end{cases}
\end{equation}
%
Per trovare i valori ancora incogniti della viscosità dinamica si usa la formula di Sutherland: per l'aria i coefficienti sono $T_0 = 288 K$, $C = 110.4 K$.
Si ottengono i valori numerici $V_m = 152.8 m/s$, $\lambda = 0.507$.

\vspace{0.5cm}
\textbf{Osservazioni.} Non è sempre possibile imporre l'uguaglianza di $Re$ e $M$. Si pensi ad esempio a un'applicazione in aria in condizioni standard e prove sul modello in galleria ad aria in condizioni standard.
Per ottenere l'uguaglianza dei numeri di Mach, bisogna avere la stessa velocità caratteristica (poichè la celerità del suono è la stessa tra condizione reale e modello). Avendo uguagliato le velocità caratteristiche ed essendo uguali le variabili termodinamiche $\rho$ e $\mu$, si ottiene l'uguaglianza della dimensione caratteristica del modello. Questo significa che sarebbe necessario avere un modello in scala 1:1 per soddisfare la similitudine utilizzando nella prova sperimentale lo stesso fluido nelle stesse condizioni termodinamiche delle condizioni ``al vero''.
Per limiti tecnologici e di costi, dovuti alle dimensioni degli apparati sperimentali, spesso è necessario utilizzare un modello in scala dell'originale. Esistono gallerie controllate in pressione per variare lo stato termodinamico dell'aria di prova e gallerie che utilizzando acqua come fluido di prova: entrambe queste scelte comportano complicazioni nel progetto e nell'utilizzo dell'impianto, traducibile spesso in costi elevati.
%
\newline
Allora per quale numero adimensionale o secondo quale combinazione dei numeri adimensionali conviene ottenere la similitudine? ``Arte'', esperienza e alcuni ``espedienti'' sperimentali, che non sono oggetto di questo corso hanno lo scopo di ottenere risultati rappresentativi del problema al vero, anche se la perfetta similitudine non è soddisfatta.
