\noindent
\begin{exerciseS}[Similitudine ad alta velocità: velivolo]
Un aeromobile vola nell'alta atmosfera a velocit\`{a} costante $V_v=252\  m/s$,
in condizioni di densit\`{a} $\rho_v$ e temperatura $T_v$ assegnate: 
$\rho_v = 0.424\ kg/m^3$, $T_v = -50.3^\circ  C$.
%
\newline
Determinare la velocit\`{a}, la densit\`{a} e la pressione dell'aria da utilizzarsi 
in una galleria del vento pressurizzata che operi alla temperatura di $15^\circ {\rm C}$
per ottenere la similitudine dinamica corretta con un modello in scala ridotta $\lambda = 0.2$.

($V_m = 286.6\ m/s$, $\rho_m = 2.292\ kg/m^3$, $p_m= 189560\ Pa$)
\end{exerciseS}

\sol
%
\partone
Similitudine fluidodinamica: numeri di Reynolds e di Mach.
\begin{equation}
 Re = \frac{\rho U L}{\mu} \quad , \quad M = \frac{U}{c} \ .
\end{equation}
Formula di Sutherland per la viscosità dinamica \textbf{dei gas},
\begin{equation}
 \mu(T) = \mu_0 \displaystyle\left(\frac{T}{T_0}\right)^{1.5}
 \frac{C+T_0}{C+T} \ .
\end{equation}
%
\parttwo
Assumendo che l'aria si comporti come gas ideale, per il quale vale l'equazione di stato $p = \rho R T$, la velocità del suono vale $c = \sqrt{\gamma R T}$, dove $\gamma = c_p / c_v$ è il rapporto dei calori specifici a pressione e volume costante, che vale $\gamma = 1.4$ per un gas biatomico. La costante del gas $R$ è definita come il rapporto tra la costante universale dei gas $\mathscr{R}$ e la massa molare $M_m$, $R = \mathscr{R}/M_m$.
La massa molare dell'aria secca vale $M_m = 28.96 \ kg / kmol$ e la sua costante $R$ vale
\begin{equation}
 R = \dfrac{\mathscr{R}}{M_m} = \dfrac{8314.4 \ J / (kmol \ K)}{28.97 \ kg/kmol} = 287.0 \dfrac{J }{kg \ K} \ .
\end{equation}
La velocità del suono nell'aria alle condizioni termodinamiche del problema vale $c = 299.2 \ m/s$. Il numero di Mach caratteristico della corrente è quindi $M=0.84$ e gli effetti di comprimibilità non possono essere trascurati, poichè il numero di Mach è maggiore della valore convenzionale $0.3$ che identifica il limite della validità dell'approssimazione di fluido incomprimibile.
%
Per ottenere la similitudine tra problema reale e quello modellato (di dimensioni ridotte) è necessaria la similitudine geometrica e l'uguaglianza dei numeri adimensionali che caratterizzano il problema, il numero di Reynolds $Re$ e il numero di Mach $M$.
%Se si fa l'ipotesi di gas ideale perfetto, la velocità del suono può essere scritta in funzione della sola variabile termodinamica temperatura $c = \sqrt{\gamma R T}$, con $\gamma$ rapporto dei calori specifici a pressione e volume costanti, $R$ costante universale del gas ($R = \mathscr{R}/M_m$).
\begin{equation}
\begin{cases}
 M_1 = M_2 \\
 Re_1 = Re_2  
\end{cases} \ .
\end{equation}
Utilizzando l'equazione di stato dei gas perfetti,
\begin{equation}
\begin{cases}
 \dfrac{V_v}{\sqrt{\gamma R T_v}} = \dfrac{V_m}{\sqrt{\gamma R T_m}} \\
 \dfrac{\rho_v V_v L_v}{\mu(T_v)} = \dfrac{\rho_m V_m L_m}{\mu(T_m)}
\end{cases}
\end{equation}
Risolvendo il sistema, si ottiene l'espressione delle incognite:
\begin{equation}
 \Rightarrow
   \begin{cases}
     V_m = V_v \sqrt{\frac{T_m}{T_v}} \\
     \rho_m = \frac{1}{\lambda} \rho_v \sqrt{\frac{T_v}{T_m}}
     \frac{\mu(T_m)}{\mu(T_v)} \\
     P_m = \rho_m R T_m = \frac{1}{\lambda} \frac{\mu(T_m)}{\mu(T_v)} \rho_v R \sqrt{T_v T_m}
   \end{cases}
\end{equation}
Per trovare i valori ancora incogniti della viscosità dinamica si usa la formula di Sutherland: per l'aria i coefficienti sono $T_0 = 288 K$, $C = 110.4 K$.
Si ottengono i valori numerici $V_m = 286.6 \ m/s$, $\rho_m = 2.292 \ kg/m^3$, $p_m = 189560 \ Pa$.


%Il problema dell'esercizio è caratterizzato da due numeri adimensionali, il numero di Reynolds $Re$ e il numero di Mach $M$.
%Per ottenere la similitudine tra problema reale e quello modellato (di dimensioni ridotte) è necessaria la similitudine geometrica e l'uguaglianza dei due numeri adimensionali.
%Se si fa l'ipotesi di gas ideale perfetto, la velocità del suono può essere scritta in funzione della sola variabile termodinamica temperatura $c = \sqrt{\gamma R T}$, con $\gamma$ rapporto dei calori specifici a pressione e volume costanti, $R$ costante universale del gas ($R = \mathscr{R}/M_m$).

%\begin{equation}
%\begin{cases}
% M_1 = M_2 \\
% Re_1 = Re_2  
%\end{cases}
%\end{equation}

%cioè

%\begin{equation}
%\begin{cases}
% \frac{V_v}{\sqrt{\gamma R T_v}} = \frac{V_m}{\sqrt{\gamma R T_m}} \\
% \frac{\rho_v V_v L_v}{\mu(T_v)} = \frac{\rho_m V_m L_m}{\mu(T_m)}
%\end{cases}
%\end{equation}

%Risolvendo il sistema per le incognite:
%\begin{equation}
% \Rightarrow
%   \begin{cases}
%     V_m = V_v \sqrt{\frac{T_m}{T_v}} \\
%     \rho_m = \frac{1}{\lambda} \rho_v \sqrt{\frac{T_v}{T_m}}
%     \frac{\mu(T_m)}{\mu(T_v)} \\
%     P_m = \rho_m R T_m = \frac{1}{\lambda} \frac{\mu(T_m)}{\mu(T_v)} \rho_v R \sqrt{T_v T_m}
%   \end{cases}
%\end{equation}

%Per trovare i valori di viscosità dinamica si usa la formula di Sutherland: per l'aria i coefficienti sono $T_0 = 288 K$, $C = 110.4 K$.
