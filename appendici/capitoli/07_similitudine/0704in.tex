 \begin{exerciseS}[Getto: codice numerico adimensionale]
 Si vuole studiare con la corrente di aria che esce da un ugello verticale
 di diametro $\tilde{D}=0.01\ m$, nell'intervallo di velocità di riferimento
 $\tilde{U} \in [1,10] \ m/s$. Si ha a disposizione un codice numerico 
 che risolve le equazioni in forma adimensionale, in cui non è possibile
 variare le condizioni al contorno, e una sola griglia di calcolo.
 Si chiede di:
 \begin{itemize}
   \item determinare l'intervallo di numeri di Reynolds $Re$ da inserire
     nel codice, sapendo che la velocità di riferimento nel codice è $U=1$
     e il diametro nella griglia vale $D=1$.
   \item la frequenza $\tilde{f}$ di rilascio di vortici quando
     $\tilde{U}=1\ m/s$, sapendo che la frequenza estratta dai risultati
     numerici è $f=0.2$;
   \item stimare l'errore compiuto dal codice nel trascurare l'effetto
     della gravità.
 \end{itemize}
 \end{exerciseS}
 
\sol

\partone
Similitudine fluidodinamica. Numeri di Reynolds e Froude. Ordini di
 grandezza dei termini.

\begin{equation}
Re = \frac{\rho U L}{\mu} \qquad , \qquad Fr = \dfrac{U}{\sqrt{g L}}
\end{equation}

\parttwo
\begin{itemize}
\item
Affinchè le simulazioni numeriche siano rappresentative della corrente
 incomprimibile che si vuole studiare, è necessario che ci sia similitudine
 fluidodinamica tra i due casi: bisogna imporre l'uguaglianza dei numeri di
 Reynolds
 \begin{equation}
   Re = \dfrac{\tilde{U}\tilde{D}}{\tilde{\nu}} \approx 
    \dfrac{(1 \div 10) m/s \times 10^{-2}m}{10^{-5}m^2/s} = 
    10^3 \div 10^4 \ .
 \end{equation}

\item 
Se la frequenza adimensionale ottenuta dalla simulazione numerica è $f=0.2$,
 la frequenza dimensionale viene ottenuta dall'ugualglianza dei numeri di 
 Strouhal, cioè ``ri-dimensionalizzando'' $f$ con le
 grandezze di riferimento usate per l'adimensionalizzazione
 ($U$,$L$,$\rho$). 
\begin{equation}
  \dfrac{f D}{U} = \dfrac{\tilde{f} \tilde{D}}{\tilde{U}} \quad \Rightarrow
  \quad \tilde{f} = f \dfrac{\tilde{U}}{\tilde{D}} 
  = 0.2 \times \dfrac{1 m/s}{10^{-2} m } = 20 s^{-1} \ .
\end{equation}

\item
Per quantificare l'effetto della gravità, si calcola il valore del numero
 di Froude. Nelle equazioni di Navier-Stokes adimensionali, compare il
 numero adimensionale $ g D / U^2 = 1 / Fr^2$ davanti ai termini di forze di
 volume. Più questo numero è ``piccolo'', più gli effetti delle forze di
 volume sono ridotti.
 \begin{equation}
 \begin{cases}
 U = 1 m/s  & : \qquad 1/Fr^2 \approx 10^{-1} \\
 U = 10 m/s & : \qquad 1/Fr^2 \approx 10^{-3}
 \end{cases}
 \end{equation}

\end{itemize} 
