\subsection{Alcuni concetti di stabilità}
Prima di indagare il sistema di Lorenz vengono introdotti alcuni concetti utili per 
 studiare la stabilità di un generico sistema dinamico,
\begin{equation}
 \dfrac{d \bm{x}}{d t } = \bm{f}(\bm{x}(t),t) \qquad , \qquad \bm{x}(0) = \bm{x}_0 \ .
\end{equation}
Alcuni di questi concetti non sono nuovi; tutte queste definizioni verranno usate 
 (e risulteranno più chiare) nella sezione successiva dedicata al sistema di Lorenz.

\vspace{0.2cm}
\noindent
\textbf{Spazio delle fasi e stati di un sistema.}
 Lo spazio delle fasi è uno spazio nel quale i punti rappresentano i possibili stati del sistema. 
 Lo stato di un sistema dinamico è identificato dal valore delle sue variabili di stato,
 ovvero le variabili che lo descrivono in maniera esaustiva da poterne prevederne
 l'evoluzione.\footnote{Conoscendo lo stato del sistema \textbf{con esattezza}
 è possibile descrivere l'evoluzione libera del sistema, in assenza di perturbazioni
 e forze esterne. In fondo a  questa sezione, sarà più chiara la necessità di conoscere
\textit{con esattezza} lo stato iniziale del sistema, per prevederne l'evoluzione.}
L'evoluzione libera di un sistema dinamico viene descritta dalle traiettorie nel suo
 spazio delle fasi.

\vspace{0.2cm}
\noindent
\textbf{Equilibri e cicli limite.} Un equilibrio $\overline{\bm{x}}$ del sistema
 è una soluzione stazionaria delle equazioni del sistema dinamico, cioè 
\begin{equation}
 \bm{0} = \bm{f}(\overline{\bm{x}}) \ .
\end{equation}
Un ciclo limite di periodo $T$ è una traiettoria periodica del sistema, tale per cui
\begin{equation}
 \bm{x}(t+T) = \bm{x}(t) \quad , \quad \forall t \ ,
\end{equation}
rappresentata nello spazio delle fasi da un'orbita chiusa (e isolata).

\vspace{0.2cm}
\noindent
\textbf{Stabilità alla Lyapunov.}
Lo studio di stabilità alla Lyapunov riguarda l'evoluzione locale del sistema dinamico
 con \textbf{condizioni iniziali perturbate}. Qualitativamente, un punto di equilibrio è stabile
 se, partendo da uno stato ``vicino'' all'equilibrio, lo stato del sistema rimane per
 sempre ``vicino'' all'equilibrio. Inoltre, l'equilibrio è asinotiticamente stabile se
 lo stato converge verso il punto di equilibrio, $\bm{x} \rightarrow \bm{\overline{x}}$
 per $t \rightarrow \infty$.
\newline
 La stabilità di Lyapunov di un equilibrio può essere
 indagata attraverso l'analisi degli autovalori del sistema linearizzato attorno al punto
 di equilibrio.

\vspace{0.2cm}
\noindent
\textbf{Stabilità strutturale.}
La stabilità strutturale considera l'evoluzione del sistema in seguito a perturbazioni
 del sistema stesso. Un sistema dinamico è strutturalmente stabile se
 le traiettorie nel suo spazio delle fasi non cambiano qualitativamente: ad esempio, in un 
 sistema strutturalmente stabile alla perturbazione di un parametro,
 non cambiano il numero dei punti di equilibrio e cicli limite.

\subsection{Sistema dinamico di Lorenz}
In questa sezione si descrive, senza nessuna pretesa di completezza,
 lo di studio di stabilità del sistema dinamico di Lorenz,
\begin{equation}
    \begin{cases}
      \dot{X} = - \sigma X + \sigma Y \\
      \dot{Y} = - Y + \rho X - X Z \\
      \dot{Z} = - \beta Z + X Y \ ,
    \end{cases}
\end{equation}
 come primo esempio di studio di stabilità di un sistema fluidodinamico.
Si studia la stabilità del sistema di Lorenz al variare del parametro $\rho$,
 mantenendo costante il valore dei parametri $\sigma$ e $\beta$.
Lorenz usò come valori $\sigma = 10$ e $\beta = 8/3$.
 Il numero di Prantdl assume un valore paragonabile a quello dell'acqua alla temperatura
 di $20^\circ C$, che vale circa $Pr \approx 7$. Il numero di Prandtl per l'aria e altri gas
 vale circa $0.7$. Il valore $\beta = 8/3$ corrisponde a
 un numero d'onda fondamentale in direzione $x$ uguale a
 $\frac{k}{2} = \frac{1}{2} \sqrt{\frac{4}{\beta} - 1} = \frac{\sqrt{2}}{2} =
 0.3536$.

\paragraph{Punti di equilibrio}
I punti di equilibrio del sistema di Lorenz soddisfano le equazioni stazionarie
\begin{equation}
    \begin{cases}
      0 = - \sigma X + \sigma Y  & \rightarrow Y = X \\
      0 = - Y + \rho X - X Z & \hspace{2.0cm} \searrow \hspace{2.0cm}\rightarrow  X[X^2-(\rho-1)] = 0 \\
      0 = - \beta Z + X Y  & \hspace{1.0cm} \rightarrow \hspace{1.0cm}  X^2 = \beta Z \hspace{0.5cm} \nearrow 
    \end{cases}
\end{equation}
L'equazione $X(X^2 - \rho) = 0$ ha una sola soluzione reale se $\rho < 1$, tre soluzioni
 per $\rho \geq 1$. Quindi per valori di $\rho < 1$ esiste un unico punto di equilibrio,
\begin{equation}
 \text{\textbf{E1}}: (\overline{X}_1, \overline{Y}_1, \overline{Z}_1) = (0,0,0) \ .
\end{equation}
Per valori di $\rho \geq 1$ esistono tre punti di equilibrio,
\begin{equation}
\begin{aligned}
 \text{\textbf{E1}}: &  (\overline{X}_1, \overline{Y}_1, \overline{Z}_1) = (0,0,0) \ , \\
 \text{\textbf{E2}}: &  (\overline{X}_2, \overline{Y}_2, \overline{Z}_2) =
 (-\sqrt{\beta(\rho-1)},-\sqrt{\beta(\rho-1)},\rho-1) \ , \\
 \text{\textbf{E3}}: &  (\overline{X}_3, \overline{Y}_3, \overline{Z}_3) =
 (+\sqrt{\beta(\rho-1)},+\sqrt{\beta(\rho-1)},\rho-1) \ .
\end{aligned}
\end{equation}
%
Prima di studiare la stabilità locale ``alla Lyapunov'' dei punti di equilibrio, riprendendo
 la definizione di \textit{stabilità strutturale}, si scopre che il sistema di Lorenz
 non è strutturalmente stabile a perturbazioni del valore di $\rho$, quando $\rho = 1$: infatti
 per $\rho < 1$ esiste un solo punto di equilibrio, per $\rho > 1$ esistono tre punti di equilibrio
 e di conseguenza le traiettorie nel piano delle fasi subiscono un cambiamento qualitativo.

\paragraph{Stabilità dell'equilibrio E1}
Si studia la stabilità ``alla Lyapunov'' dell'equilibrio \textbf{E1}: $(0,0,0)$. Linearizzando
 il sistema non lnieare di Lorenz attorno all'equilibrio \textbf{E1}, si ottiene il sistema
 linear(izzato)
\begin{equation}
 \begin{bmatrix} \delta \dot{x} \\ \delta \dot{y} \\ \delta \dot{z} \end{bmatrix} = 
 \begin{bmatrix}-\sigma & \sigma & 0 \\ \rho & -1 & 0 \\ 0 & 0 & - \beta \end{bmatrix}  
 \begin{bmatrix} \delta x \\ \delta y \\ \delta z \end{bmatrix} \quad , \quad
 \delta \dot{\bm{x}} = \bm{J}|_{\bm{E1}} \, \delta \bm{x}
\end{equation}
il cui polinomio caratteristico è
\begin{equation}
\begin{aligned}
 p(\lambda) = det(\bm{J}- \lambda \bm{I}) & = -(\beta+\lambda)[(\sigma+\lambda)(1+\lambda)-\sigma\rho] = \\ 
  & = -(\beta+\lambda) [ \lambda^2 + (\sigma+1)\lambda + \sigma(1-\rho)] \ . 
\end{aligned}
\end{equation}
Gli autovalori del sistema linearizzato attorno al primo equilibrio sono quindi
\begin{equation}
  \lambda^{E1}_1 = - \beta \quad , \quad  
  \lambda^{E1}_{2,3} = - \dfrac{\sigma+1}{2} \mp \dfrac{\sqrt{(\sigma+1)^2-4\sigma(1-\rho)}}{2} \ . 
\end{equation}
Per valori positivi dei parametri, tutti gli autovalori sono reali. Gli autovalori $\lambda^{E1}_1$
 e $\lambda^{E1}_2$ sono negativi per ogni valore di $\rho$, mentre l'autovalore $\lambda^{E1}_3$
 cambia segno per $\rho = 1$. L'analisi lineare di stabilità permette di concludere che l'equilibrio
 \textbf{E1} è linearmente stabile per $\rho<1$ e instabile per $\rho > 1$, mentre non
 permette di affermare nulla sul caso $\rho = 1$.

\paragraph{Stabilità degli equilibri E2, E3}
Per valori di $\rho \geq 1$ esistono i due equilibri \textbf{E2}, \textbf{E3}. Si studia
 la loro stabilità ``alla Lyapunov'' tramite lo studio degli autovalori del sistema linearizzato
 attorno ai punti di equilibrio,
\begin{equation}
 \begin{bmatrix} \delta \dot{x} \\ \delta \dot{y} \\ \delta \dot{z} \end{bmatrix} = 
 \begin{bmatrix}-\sigma & \sigma & 0 \\ 1 & -1 & \mp \sqrt{\beta(\rho-1)} \\
 \pm \sqrt{\beta(\rho-1)} & \pm \sqrt{\beta(\rho-1)} & - \beta \end{bmatrix}  
 \begin{bmatrix} \delta x \\ \delta y \\ \delta z \end{bmatrix} \quad , \quad
 \delta \dot{\bm{x}} = \bm{J}|_{\bm{E2,3}} \, \delta \bm{x} \ .
\end{equation}
Si può dimostrare (con il criterio di Routh-Hurwitz analiticamente, o calcolandone numericamente
 il valore) che i due punti di equilibrio sono stabili se
 $\rho < \frac{\sigma(\sigma+\beta+3)}{\sigma-1-\beta} \approx 24.7368$, utilizzando i valori
 $\sigma = 10$, $\beta = 8/3$.

\paragraph{Biforcazioni, cicli limite e attrattori strani}
\begin{figure}[t]
  \centering
  \begin{tabular}{cc}
  \begin{overpic}[width=0.40\textwidth]{./fig/bif_pitchfork}
  \put(-5,85){(a.1)}
  \put(-5,45){(a.2)}
  \end{overpic}  \hspace{0.8cm}
  \begin{overpic}[width=0.45\textwidth]{./fig/bif_hopf}
  \put(-5,75){(b)}
  \end{overpic}  \hspace{0.8cm}
  \end{tabular}
 \caption{Diagrammi di biforcazione. (a) Biforcazione pitchfrok per $\rho = 1$:
   (a.1) luogo delle radici dell'equilibrio \textbf{E1} e (a.2) diagramma di biforcazione. 
  Per $\rho = 1$ l'equilibrio \textbf{E1} diventa instabile e nascono i due equilibri stabili
  \textbf{E2,3}. (b) Biforcazione di Hopf degli equilibri \textbf{E2,2}:
  diagramma di biforcazione, rappresentato utilizzando  la forma normale.
  Per $\varepsilon = 0$, $\rho = 24.7368$, il ciclo limite instabile collassa sull'equilibrio
  stabile, che diventa instabile.}
\end{figure}
L'analisi degli autovalori del sistema linearizzato attorno ai punti di equilibrio permette di 
 determinarne le caratteristiche locali quando gli autovalori hanno parte reale diversa
 da zero. In corrispondenza del cambio di stabilità di un punto di equilibrio e/o della 
 comparsa/scomparsa di punti di equilibrio (ma non solo!), le traiettorie nello spazio delle
 fasi del sistema subiscono un cambiamento qualitativo: il sistema non è strutturalmente stabile
 e si verifica una \textbf{biforcazione}.
\newline
Per studiare la stabità locale di un equilibrio in presenza di autovalori a parte reale nulla
 è necessario costruire un'approssimazione non lineare del sistema. Si considera un punto di equilibrio
 per il quale il sistema linearizzato non ha autovalori instabili, ha $N_s$ autovalori stabili e 
 $N_c$ autovalori a parte reale nulla e si vuole determinare l'evoluzione del sistema nelle
 vicinanze del punto di equilibrio.
Si può dimostrare che la dinamica del sistema $N=N_s+N_c$-dimensionale si riduce velocemente
 alla dinamica di un sistema $N_c$ dimensionale: le $N_s$ dinamiche asintoticamente stabili
 associate agli autovalori con parte reale negativa tendono asintoticamente ad annullarsi
 nell'intorno dell'equilibrio, mentre rimangono solo le dinamiche associate alle $N_c$ 
 dinamiche marginalmente stabili.
\newline
Si può usare un'espansione polinomiale per approssimare il sistema non lineare originale e
 costruire la \textbf{varietà centrale}, cioè la regione dello spazio delle fasi nella quale
 si svolgono le dinamiche marginalmente stabili.
\newline
\noindent 
Ad esempio, quando $\rho = 1$ il sistema di Lorenz nell'intorno dell'equilibrio \textbf{E1} (e
 dei nascenti equilibri \textbf{E2,3}) può essere ricondotto alla dinamica del sistema monodimensionale
\begin{equation}\label{eqn:lorenz:pitchfork}
 \dot{a}(t) = f(a(t)) = a(t) [ \alpha \varepsilon - \beta a(t)^2 ] \quad , \quad
 \text{con } \varepsilon := \rho-1 \ ,
\end{equation}
con $\alpha \approx 0.909$ e $\beta \approx 0.170$. Questo sistema coincide alla \textbf{forma
 normale} della biforcazione, cioè il sistema più semplice in grado di descrivere il cambiamento
 qualitativo del sistema. Lo studio della forma normale della biforcazione rivela l'esistenza
 di un unico equilibrio stabile $\overline{a}_1 = 0.0$ per $\epsilon \leq 0$, cioé $\rho \leq 1$.
 Per $\rho > 1$ l'equilibrio $\overline{a}_1$ diventa instabile e nascono due equilibri stabili
 $\overline{a}_{2,3} = \mp \sqrt{\alpha \varepsilon / \beta}$.
L'equazione (\ref{eqn:lorenz:pitchfork}) rappresenta la forma normale di una \textit{biforcazione
 pitchfork}. Poiché $\beta > 0$, la biforcazione si definisce \textit{supercritica}.
\newline
Analogamente, quando $\rho \approx 24.7368$ i due equilibri \textbf{E2,3} cambiano stabilità: u
 coppia di autovalori complessi coniugati attraversano l'asse immaginario e la loro parte
 reale diventa positiva. Questo tipo di instabilità strutturale viene definita
 \textit{biforcazione di Hopf}: cambia la stabilità del punto di equilibrio considerato 
 e nasce/sparisce un ciclo limite nel suo intorno (il ciclo limite nasce da o riduce al
 punto di equilibrio).
L'approssimazione alla varietà centrale del sistema attorno a uno dei due equilibri conduce 
 al sistema di equazioni
\begin{equation}\label{eqn:lorenz:hopf}
\begin{cases}
 \dot{r}(t) = \alpha_r r \varepsilon - \beta_r r^3 \\
 \dot{\theta}(t) = \omega + \alpha_i \varepsilon + \beta_i r^2 \ 
\end{cases} \quad , \quad 
 \text{con } \varepsilon := \rho-24.7368 \ ,
\end{equation}
dove è stata utilizzata la rappresentazione polare complessa $a(t) = r(t) e^{i \theta(t)}$
 della variabile $a(t)$ che descrive la dinamica del sistema ridotta alla varietà centrale.
 Il parametro $\omega = 9.6245$ coincide con la parte immaginaria degli autovalori marginalmente
 stabili e gli altri parametri valgono:
\begin{equation}
\begin{aligned}
 & \alpha_r = 0.0302 \qquad , \qquad \beta_r =-0.003 \\
 & \alpha_i = 0.1815 \qquad , \qquad \beta_i =-0.028 \ . 
\end{aligned}
\end{equation}
La prima equazione delle (\ref{eqn:lorenz:hopf}) è identica all'equazione che descrive la 
 biforcazione pitchfork. In questo caso, però, il coefficiente $\beta_r$ è minore di zero. Questo
 tipo di biforcazione si definisce \textit{subcritica}. Si può facilmente dimostrare che
 per $\varepsilon < 0$ esistono due (il raggio $r$ di una rappresentazione polare deve
 essere $\geq 0$) equilibri
\begin{equation}
 \overline{\rho}_1 = 0 \quad , \quad \overline{\rho}_2 = \sqrt{-\alpha_r \varepsilon / \beta_r} \ .
\end{equation}
Il primo equilibrio dell'equazione in $r$ corrisponde a un punto fisso, poichè il raggio è nullo.
 Il secondo equilibrio corrisponde al raggio $\overline{\rho}_2$ del ciclo limite
 esistente per $\varepsilon < 0$.
Si dimostra quindi che un ciclo limite instabile coesiste con ognuno dei due punti di
 equilibri stabili \textbf{E2,3} per $\varepsilon < 0$ (cioè $\rho < 24.7368$), almeno
 in un intervallo finito di valori di $\rho$.
Quando $\varepsilon = 0$ (cioè $\rho < 24.7368$), il ciclo limite instabile si riduce al punto
 di equilibrio. Per $\varepsilon > 0$ il punto di equilibrio diventa instabile, mentre
 scompare il ciclo limite.

\vspace{0.5cm}
\noindent
 Rimangono aperte alcune questioni: è possibile descrivere i cicli limite esistenti per (alcuni)
 valori del parametro $\rho < 24.7368$? Qual è l'evoluzione del sistema per valori
 di $\rho > 24.7368$? Ha senso utilizzare il modello di Lorenz, un brutale troncamento di
 un sistema continuo che dà origine a un sistema tridimensionale, per descrivere l'evoluzione
 del sistema fisico per valori crescenti del numero di Rayleigh $Ra$, e quindi del
 parametro $\rho$?

\vspace{0.3cm}
\noindent
Partendo dall'espressione approssimata del ciclo limite ottenuta dalla forma normale della
 biforcazione di Hopf per $\rho \lesssim 24.7368$ è possibile calcolare la forma del ciclo limite
 per valori inferiori del parametro, tramite tecniche di \textbf{continuazione}: negli algoritmi
 di continuazione la soluzione di un problema, nota per un valore del parametro, viene utilizzata
 per stimare la guess iniziale dello stesso problema per un valore diverso del parametro. 
In particolare, per identificare la traiettoria periodica corrispondente a un ciclo limite
 si può utilizzare una tecnica di \textbf{bilanciamento armonico}: la traiettoria periodica
 viene scritta come serie di Fourier, della quale è necessario determinare i coefficienti.

\begin{figure}[t]
  \centering
  \begin{tabular}{cc}
  \begin{overpic}[width=0.65\textwidth, trim={0 -10 0 0}, clip]{./fig/ic_dependence_xt_rho+28}
  \put(0,35){(a)}
  \end{overpic} % \hspace{0.8cm}
  \begin{overpic}[width=0.35\textwidth, trim={60 80 40 0}, clip]{./fig/ic_dependence_attractor_rho+28}
  \put(0,60){(b)}
% \put(-5,45){(a.2)}
  \end{overpic}  \hfill
  \end{tabular}
\caption{Dinamica caotica del sistema di Lorenz per $\rho = 24.74$: evoluzione del sistema
 con condizioni iniziali $\bm{x}^{(1)}_0 = (-10,10,1)$, in blu, e $\bm{x}^{(2)}_0 = \bm{x}^{(1)}_0 +
 1.0\cdot 10^{-9}$, in arancione.
 (a) Evoluzione temporale
 della variabile $X(t)$: partendo da due condizioni iniziali ``vicine'', le due traiettorie
 del sistema si discostano in maniera ``non banale''. Il sistema dimostra un'evoluzione non 
 periodica, estremamente sensibile alle condizioni iniziali e quindi caotica.
 (b) Attrattore di Lorenz nello spazio delle fasi:
 le traiettorie nello spazio delle fasi rivelano la presenza di un attrattore, ``nelle
 vicinanze'' del quale si svolge la dinamica asintotica del sistema.}\label{fig:lorenz-chaos}
\end{figure}


 
\vspace{0.3cm}
\noindent
Per valori di $\rho > 24.7368$ non esistono punti di equilibrio stabili punti di equilibrio
 stabili e non esistono cicli limite stabili. La dinamica del sistema rimane confinata 
 in una regione limitata dello spazio delle fasi, senza divergere.
L'evoluzione del sistema rappresentata in figura (\ref{fig:lorenz-chaos}) dimostra l'elevata
 sensibilità della soluzione alle condizioni 
 iniziali e l'assenza di equilibri o dinamiche periodiche stabili, caratteristici di un
 \textbf{regime caotico}. L'evoluzione di lungo tempo del sistema avviene ``nelle vicinanze''
 dell'attrattore di Lorenz, del quale si può intuire la forma grazie alle traiettorie
 rappresentate in figura (\ref{fig:lorenz-chaos})(b).
\begin{figure}[t]
  \centering
  \begin{tabular}{cc}
  \begin{overpic}[width=0.45\textwidth, trim={60 40 60 0}, clip]{./fig/lorenz_cm001}
  \put(0,75){(a)}
  \end{overpic} \hfill 
  \begin{overpic}[width=0.45\textwidth, trim={60 40 60 0}, clip]{./fig/lorenz_cm002}
  \put(0,75){(b)}
  \end{overpic}  \\
  \begin{overpic}[width=0.45\textwidth, trim={60 40 60 0}, clip]{./fig/lorenz_cm003}
  \put(0,75){(c)}
  \end{overpic} \hfill 
  \begin{overpic}[width=0.45\textwidth, trim={60 40 60 0}, clip]{./fig/lorenz_cm004}
  \put(0,75){(d)}
  \end{overpic}  \\
  \begin{overpic}[width=0.45\textwidth, trim={60 40 60 0}, clip]{./fig/lorenz_cm005}
  \put(0,75){(e)}
  \end{overpic} \hfill 
  \begin{overpic}[width=0.45\textwidth, trim={60 40 60 0}, clip]{./fig/lorenz_cm006}
  \put(0,75){(f)}
  \end{overpic}  \\
  \end{tabular}
\caption{Evoluzione del sistema di Lorenz per $\rho = 24.7368$ nel piano delle fasi.
Dinamica caotica, equilibri (\textbf{E1} in azzurro, \textbf{E2} in arancione e
 \textbf{E3} in verde) e varietà centrali dei due equilibri marginalmente stabili
 \textbf{E2,3}. La dinamica asintotica del sistema caotica del sistema alterna in
 maniera irregolare delle oscillazioni attorno ai due equilibri instabili sulle 
 ``nelle vicinanze'' delle rispettive varietà centrali.}
\label{fig:lorenz-chaos-cm}
\end{figure}
La figura (\ref{fig:lorenz-chaos-cm}) rappresenta la traiettoria del sistema di Lorenz e le
 \textit{varietà centrali} dei due equilibri $\textbf{E2,3}$ marginalmente stabili 
 per $\rho = 24.7368$. Qualitativamente, lo stato del sistema viene attratto su queste
 superfici, lungo le direzioni stabili. Su queste superfici poi, si può osservare la dinamica
 marginalmente stabile (di dimensione ridotta: per il sistema di Lorenz, di dimensione 2, invece
 della dimensione 3 del sistema completo) del sistema: lo stato del sistema inizialmente
 oscilla attorno all' equilibrio \textbf{E2} (ad esempio),
 prima di essere attratta in maniera ``difficilmente prevedibile''
 dalla varietà centrale dell'equilibrio \textbf{E3} e iniziare ad oscillare attorno a 
 quest'ultimo equilibrio.

\vspace{0.3cm}
\noindent
L'approssimazione di Lorenz di dimensioni ridotte del sistema fisico continuo (e quindi
 di dimensione infinita) perde significato all'aumentare del numero di Rayliegh: all'aumentare
 del numero di Rayleigh infatti si attivano delle dinamiche più complesse, di dimensione
 maggiore, non descrivibili a un sistema tridimensionale.


