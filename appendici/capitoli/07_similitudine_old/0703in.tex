\noindent
\begin{exerciseS}[Similitudine a bassa velocità: sottomarino]
La velocit\`{a} di pattugliamento di un sottomarino vale $V_v = 2.5\ m/s$.
Considerando che il sottomarino si muova in acqua in condizioni standard, a quale velocit\`{a}
deve essere provato un modello in scala $\lambda = 1/10$, avendo a disposizione
rispettivamente:
\begin{itemize}
  \item una galleria ad acqua in condizioni standard,
  \item una galleria ad aria a pressione di $10 \ bar$ e temperatura di 
        $30^\circ  C$?
\end{itemize}

Se la resistenza al vero vale $D_v=6000\ N$, quanto vale la resistenza sui
modelli in scala nei due casi?

(Galleria ad aria: $V_m = 35.17\  m/s$, $D_m = 136.1\ N$. 
 Galleria ad acqua: $V_m = 25\  m/s$, $D_m = 6000\ N$.)
\end{exerciseS}

\sol

\partone
Similitudine fluidodinamica per correnti incomprimibili, numero di Reynolds,\vspace{-0.5cm}
\begin{equation}
 Re = \frac{\rho U L}{\mu} \ .
\end{equation}
Formula di Sutherland per la viscosità dinamica \textbf{dei gas},
\begin{equation}
 \mu(T) = \mu_0 \displaystyle\left(\frac{T}{T_0}\right)^{1.5}
 \frac{C+T_0}{C+T} \ .
\end{equation}
Azioni agenti sul modello e coefficienti di forza.

\vspace{0.5cm}
\parttwo
La velocità sul modello si trova tramite l'uguaglianza dei numeri di Reynolds,
\begin{equation}
 \frac{\rho_v U_v L_v}{\mu_v} = \frac{\rho_m U_m L_m}{\mu_m} \quad 
 \rightarrow \quad U_m = U_v \frac{\rho_v L_v}{\rho_m L_m} \ .
 \frac{\mu_m}{\mu_v}
\end{equation}
Per trovare la viscosità dell'aria viene utilizzata la formula di Sutherland (per l'aria i coefficienti sono $T_0 = 288 K$, $C = 110.4 K$, $\mu_0 = 18.27 \mu Pa s$). Il coefficienti di viscosità dinamica dell'acqua in condizioni standard è dell'ordine di $10^{-3} \ kg / (m \ s)$.
%
\newline
La forza agente aerodinamica agente sul corpo, la cui superficie esterna è indicata con $S$, è la risultante degli sforzi di superficie esterna del corpo ${S_b}$,
\begin{equation}
\begin{aligned}
 \bm{F} = \oint_S \bm{t}_{\bm{n}} =  \oint_S -p \bm{\hat{n}} + \mu [\bm{\nabla} \bm{u} + \bm{\nabla}^{T} \bm{u}] \cdot \bm{\hat{n}} 
\end{aligned}
\end{equation}
Vengono scelte la densità caratteristica del fluido $\rho$, una velocità caratteristica della corrente $U$ e una lunghezza caratteristica del problema $L$, per definire la scala della pressione $P = \rho U^2$. Raccogliendo le dimensioni fisiche fuori dal segno di integrale è quindi possibile scrivere,
\begin{equation}
\begin{aligned} 
 \bm{F} & =  \oint_{S_b} -P p^*\bm{\hat{n}} + \frac{\mu U}{L} [\bm{\nabla}^*
  \bm{u}^* + \bm{\nabla}^{*T} \bm{u}^*] \bm{\hat{n}} = 
  & \qquad \text{($P = \rho U^2$, $dS = L^2 dS^*$)} \\
 & = \rho U^2 L^2 \oint_{S_b^*} - p^*\bm{\hat{n}} + \frac{1}{Re} [\bm{\nabla}^*
  \bm{u}^* + \bm{\nabla}^{*T} \bm{u}^*] \bm{\hat{n}} = \\
 & = \dfrac{1}{2}\rho U^2 S \bm{c}_{\bm{F}}(Re) \ ,
\end{aligned}
\end{equation}
avendo introdotto il coefficiente di forza $\bm{c}_{\bm{F}}$, 
\begin{equation}
 \bm{c}_{\bm{F}} = 2 \dfrac{L^2}{S} \oint_{S_b^*} - p^*\bm{\hat{n}} + \frac{1}{Re} [\bm{\nabla}^* \bm{u}^* + \bm{\nabla}^{*T} \bm{u}^*] \bm{\hat{n}}
\end{equation}
che può dipendere dalle variabili fisiche solo attraverso i numeri adimensionali del problema (in questo caso solo da $Re$, per problemi comprimibili anche da $M$) e che rappresenta la forza agente sul corpo adimensionalizzata con la pressione dinamica $\frac{1}{2}\rho U^2$ e con una supreficie di riferimento del corpo $S$. La superficie di riferimento $S$ scala con $L^2$ ($S = a L^2$, $a$ costante).
\newline
Si può scrivere la risultante delle forze sul modello e al vero come
\begin{equation}
 \begin{cases}
  \bm{F}_m = \dfrac{1}{2}\rho_m U_m^2 S_m^2 \bm{c_F}(Re_m) \\
  \bm{F}_v = \dfrac{1}{2}\rho_v U_v^2 S_v^2 \bm{c_F}(Re_v) \ .
 \end{cases}
\end{equation}
Poichè è soddisfatta la similitudine fluidodinamica, i valori dei coefficienti di forza del modello e ``al vero'' sono uguali. Si può quindi scrivere
\begin{equation}
 \bm{F}_m = \dfrac{\rho_m}{\rho_v} \left( \dfrac{U_m}{U_v} \right)^2
  \left( \dfrac{S_m}{S_v} \right) \bm{F}_v = 
 \dfrac{\rho_m}{\rho_v} \left( \dfrac{U_m}{U_v} \right)^2
  \left( \dfrac{L_m}{L_v} \right)^2 \bm{F}_v = 
 \dfrac{\rho_m}{\rho_v} \left( \dfrac{U_m}{U_v} \right)^2
  \lambda^2 \bm{F}_v \ ,
\end{equation}
Nel caso della galleria ad acqua, nella quale il fluido è lo stesso e nello stesso stato termodinamico della situazione reale ($\rho_m = \rho_v$, $\mu_m = \mu_v$), l'uguaglianza dei numeri di Reynolds si semplifica in
\begin{equation}
 \dfrac{\rho_m U_m L_m}{\mu_m} = \dfrac{\rho_v U_v L_v}{\mu_v} \quad \rightarrow \quad
  U_m L_m = U_v L_v \ .
\end{equation}
Quindi, in questo caso la forza agente sul modello di galleria coincide con la forza agente sul corpo nella situazione reale,
\begin{equation}
 \bm{F}^{H_2O,s}_m = \bm{F}^{H_2O,s}_v \ .
\end{equation}
%%%%%%
%Nel caso della galleria ad acqua, le densità e le viscosità sono uguali; avendo imposto l'uguaglianza del numero di Reynolds, $U_m L_m = U_v L_m$, quindi la forza agente sul modello è uguale a quella agente sul corpo vero.

%Nel caso della galleria ad aria, la forza agente sul modello si ricava come:
%\begin{equation}
% \begin{cases}
%  \bm{F}_a = \rho_a U_m^2 L_m^2 \oint_{S^*} ...\\
%  \bm{F}_v = \rho   U^2   L^2   \oint_{S^*} ...
% \end{cases}
% \Rightarrow \quad
% \bm{F}_a = \bm{F}_v \frac{\rho_a U_m^2}{\rho U_v^2} \lambda^2
%\end{equation}
