%
\section{Teorema di Buckingham}
Il teorema di Buckingham afferma che un problema descritto da $n$ variabili fisiche, le cui dimensioni fisiche coinvolgono $k$ grandezze fondamentali, può essere espresso in funzione di $n-k$ gruppi adimensionali.

\section{Equazioni di Navier--Stokes incomprimibili in forma adimensionale}
Nelle equazioni incomprimibili di Navier--Stokes per un fluido a densità costante
\begin{equation}
\begin{cases}
 \rho \dfrac{\partial \bm{u}}{\partial t} + \rho (\bm{u} \cdot \bm{\nabla}) \bm{u} - \mu \Delta \bm{u} + \bm{\nabla} p = \rho \bm{g} \\
 \bm{\nabla} \cdot \bm{u} = 0 \ ,
\end{cases}
\end{equation}
compaiono 7 variabili fisiche $(\rho,\bm{u},\mu,p,\bm{g};\bm{r},t)$, le 2 variabili indipendenti spaziale $\bm{r}$ e temporale $t$, e le 5 variabili dipendenti rappresentate dalla densità $\rho$, dal campo di velocità $\bm{u}$, dal coefficiente di viscosità dinamica $\mu$, dal campo di pressione $p$ e dal campo di forze di volume $\bm{g}$.
Le dimensioni fisiche delle 7 variabili possono essere costruite con 3 grandezze fondamentali, la massa $M$, la lunghezza $L$ e il tempo $T$. Ad esempio, le dimensioni fisiche della velocità sono $[\bm{u}] = L \ T^{-1}$ e quelle della densità sono $[\rho] = M \ L^{-3}$. Le dimensioni delle 7 variabili fisiche che compaiono nelle equazioni di Navier--Stokes incomprimibili sono raccolte nella tabella \ref{tab:adim-ns-1}.
\begin{table}[h]
\begin{center}
\begin{tabular}{c ccccccc}
   & $\bm{r}$ & $t$ & $\rho$ & $\bm{u}$ & $\mu$ & $p$ & $\bm{g}$ \\ 
 \hline \vspace{-10pt}  \\
 M & 0 & 0 &  1 &  0 &  1 &  1 &  0 \\
 L & 1 & 0 & -3 &  1 & -1 & -1 &  1 \\
 T & 0 & 1 &  0 & -1 & -1 & -2 & -2 \\
\end{tabular}
\end{center}
\caption{Variabili fisiche e grandezze fondamentali.}\label{tab:adim-ns-1}
\end{table}
Per poter formare i $7-3 = 4$ gruppi adimensionali che caratterizzano il problema, è necessario scegliere 3 variabili fisiche (o combinazione di queste) che ``contengano in maniera linearmente indipendente'' tutte le 3 grandezze fondamentali del problema. Facendo riferimento alla tabella \ref{tab:adim-ns-1}, le colonne relative alle variabili scelte per l'adimensionalizzazione devono formare dei vettori linearmente indipendenti tra di loro. Ad esempio, due scelte valide delle variabili da usare per l'adimensionalizzazione del problema sono:
\begin{itemize}
 \item $(\rho,U,L)$, una densità, una velocità e una lunghezza di riferimento,
 \item $(\mu,U,L)$, una viscosità, una velocità e una lunghezza di riferimento,
\end{itemize}
mentre una scelta non accettabile è una terna $(T,U,L)$ formata da un tempo, una velocità e una lunghezza di riferimento, poichè non è possibile costruire dei gruppi adimensionali con le variabili fisiche che contengono la massa come grandezza fisica, come la densità, la presssione e il coefficiente di viscosità.
\newline
Tutte le variabili fisiche vengono espresse come il prodotto di una loro grandezza di riferimento, che contiene le dimensioni fisiche e viene indicata con la tilde, e la loro versione adimensionale, indicata con l'asterisco,
\begin{equation}
\begin{aligned}
\bm{r} & = \tilde{L} \bm{r}^* \quad , \quad t = \tilde{T}\ t^* \quad , \quad \bm{u} = \tilde{U} \bm{u}^* \\
\rho & = \tilde{\rho} \ \rho^* \quad , \quad \mu = \tilde{\mu} \ \mu^* \quad , \quad 
p = \tilde{p} \ p^* \quad , \quad \bm{g} = \tilde{g} \ \bm{g}^* \ .
\end{aligned}
\end{equation}
Per le equazioni di Navier--Stokes incomptimibili a proprietà costanti, è possibile scegliere il valore di riferimento della densità e della viscosità dinamica come il valore stesso delle variabili fisiche, $\tilde{\rho} = \rho$, $\tilde{\mu} = \mu$. In questo modo, il loro valore adimensionale è uguale a 1, $\rho^* = \mu^* = 1$. Nel caso del campo di forze di volume dovuto alla gravità, costante e diretto lungo la verticale, è possibile definire il valore di riferimento $\tilde{g} = |\bm{g}|$, cosicché il vettore $\bm{g}^*$ è uguale e contrario al versore $\bm{\hat{z}}$ orientato in direzione verticale. Anche l'operatore \textit{nabla} viene adimensionalizzato, $\bm{\nabla} = \frac{1}{\tilde{L}} \bm{\nabla}^*$.
Le equazioni di Navier--Stokes possono essere scritte come
\begin{equation}
\begin{cases}
 \dfrac{\rho \tilde{U}}{\tilde{t}} \dfrac{\partial \bm{u}^*}{\partial t^*} + \dfrac{\rho \tilde{U}^2}{\tilde{L}} (\bm{u}^* \cdot \bm{\nabla}^*) \bm{u}^* - \dfrac{\mu \tilde{U}}{\tilde{L}^2} \Delta^* \bm{u}^* + \dfrac{\tilde{p}}{\tilde{L}} \bm{\nabla}^* p^* = -\rho g \bm{\hat{z}} \\
 \dfrac{\tilde{U}}{\tilde{L}}\bm{\nabla}^* \cdot \bm{u}^* = 0 \ .
\end{cases}
\end{equation}

\subsection{Adimensionalizzazione ``ad alti numeri di Reynolds''}
Se si scelgono $(\tilde{\rho},\tilde{U},\tilde{L})$ come grandezze di riferimento, dividendo l'equazione della quantità di moto per $\tilde{\rho} \tilde{U}^2 / \tilde{L}$ e il vincolo di incomprimibilità per $\tilde{U} / \tilde{L}$,
\begin{equation}
\begin{cases}
 \dfrac{\tilde{L}}{\tilde{U}\tilde{t}} \dfrac{\partial \bm{u}^*}{\partial t^*} + (\bm{u}^* \cdot \bm{\nabla}^*) \bm{u}^* - \dfrac{\mu }{ \rho \tilde{U} \tilde{L} } \Delta^* \bm{u}^* + \dfrac{\tilde{p}}{\rho \tilde{U}^2} \bm{\nabla}^* p^* = -\dfrac{g\tilde{L}}{\tilde{U}^2} \bm{\hat{z}} \\
 \bm{\nabla}^* \cdot \bm{u}^* = 0 \ ,
\end{cases}
\end{equation}
si possono riconoscere 4 numeri adimensionali:
\begin{itemize}
 \item il numero di Strouhal, $St = \frac{\tilde{L}}{\tilde{U}\tilde{t}}$, che rappresenta il rapporto tra una scala dei tempi e la scala dei tempi $\tilde{L}/\tilde{U}$ costruita con la lunghezza e la velocità di riferimento;
 \item il numero di Reynolds, $Re = \frac{\rho \tilde{U} \tilde{L} }{\mu}$, che rappresenta il rapporto tra gli effetti di inerzia e quelli viscosi;
 \item il numero di Eulero, $Eu = \frac{\tilde{p}}{\rho \tilde{U}^2}$, che rappresenta il rapporto tra la grandezza di riferimento della pressione e quella di un energia cinetica del fluido;
 \item il numero di Froude, $Fr = \frac{\tilde{U}^2}{g\tilde{L}}$, che rappresenta il rapporto tra gli effetti di inerzia e quelli dovuti al campo di forze di volume.
\end{itemize}
Quando non esiste una scala dei tempi ``indipendente'' dal fenomeno fluidodinamico, è possibile scegliere il valore di riferimento del tempo $\tilde{t} = \tilde{L} / \tilde{U}$, in modo tale da ottenere un numero di Strouhal unitario. Per la natura stessa della ``pressione'' di moltiplicatore di Lagrange introdotto nelle equazioni di Navier--Stokes per imporre il vincolo di incomprimibilità, è frequente che la pressione non abbia una scala indipendente nel regime incomprimibile. \'E possibile quindi scegliere una scala di pressione $\tilde{p} = \rho \tilde{U}^2$, in modo tale da ottenere un numero di Eulero unitario,
\begin{equation}
\begin{cases}
 \dfrac{\partial \bm{u}^*}{\partial t^*} + (\bm{u}^* \cdot \bm{\nabla}^*) \bm{u}^* - \dfrac{1}{Re} \Delta^* \bm{u}^* + \bm{\nabla}^* p^* = -\dfrac{1}{Fr} \bm{\hat{z}} \\
 \bm{\nabla}^* \cdot \bm{u}^* = 0 \ .
\end{cases}
\end{equation}
Se le grandezze di riferimento sono rappresentative del problema, in modo tale da rendere gli ordini di grandezza delle variabili adimensionali paragonabili tra loro, il valore dei numeri adimensionali permette di valutare l'influenza dei termini. Ad esempio, per valori elevati del numero di Froude l'influenza delle forze di volume è ridotta. Per valori elevati del numero di Reynolds, l'influenza degli effetti viscosi diventa trascurabile nelle regioni del campo di moto nelle quali le derivate spaziali del campo di velocità sono piccole. Per applicazionii tipiche aeronautiche ad alti numeri di Reynolds, gli effetti viscosi saranno quindi trascurabili in gran parte del dominio, ad eccezione delle regioni di strato limite, all'interno delle quali la componente della velocità ``parallela'' alla parete ha una variazione elevata in direzione perpendicolare alla parete stessa.
Se gli effetti delle forze di volume sono trascurabili ($Fr \rightarrow \infty$), le equazioni di Navier--Stokes incomprimibili per problemi ad alti numeri di Reynolds ($Re \rightarrow \infty$) si riducono alle equazioni di Eulero incomprimibili nelle regioni del dominio in cui gli effetti viscosi sono trascurabili,
\begin{equation}
\begin{cases}
 \dfrac{\partial \bm{u}^*}{\partial t^*} + (\bm{u}^* \cdot \bm{\nabla}^*) \bm{u}^* + \bm{\nabla}^* p^* = \bm{0} \\
 \bm{\nabla}^* \cdot \bm{u}^* = 0 \ .
\end{cases}
\end{equation}

\subsection{Adimensionalizzazione ``a bassi numeri di Reynolds''}
Se si scelgono $(\tilde{\rho},\tilde{U},\tilde{L})$ come grandezze di riferimento, dividendo l'equazione della quantità di moto per $\tilde{\mu} \tilde{U} / \tilde{L}^2$ e il vincolo di incomprimibilità per $\tilde{U} / \tilde{L}$, le equazioni di Navier--Stokes diventano
\begin{equation}
\begin{cases}
 \dfrac{\rho\tilde{L}^2}{\mu \tilde{t}} \dfrac{\partial \bm{u}^*}{\partial t^*} + \dfrac{\rho \tilde{U} \tilde{L}}{\mu}(\bm{u}^* \cdot \bm{\nabla}^*) \bm{u}^* - \Delta^* \bm{u}^* + \dfrac{\tilde{p}\tilde{L}}{\mu \tilde{U}} \bm{\nabla}^* p^* = -\dfrac{\rho g\tilde{L}^2}{\mu \tilde{U}} \bm{\hat{z}} \\
 \bm{\nabla}^* \cdot \bm{u}^* = 0 \ .
\end{cases}
\end{equation}
Se gli effetti delle forze di volume sono trascurabili rispetto agli effetti viscosi e non ci sono scale indipendenti di tempo e pressione, le equazioni di Navier--Stokes in forma adimensionale diventano
\begin{equation}
\begin{cases}
 \dfrac{\partial \bm{u}^*}{\partial t^*} + Re (\bm{u}^* \cdot \bm{\nabla}^*) \bm{u}^* - \Delta^* \bm{u}^* + \bm{\nabla}^* p^* = \bm{0} \\
 \bm{\nabla}^* \cdot \bm{u}^* = 0 \ .
\end{cases}
\end{equation}
Per correnti nelle quali il numero di Reynolds caratteristico tende a zero, note come \textit{creeping flow}, il termine non lineare diventa trascurabile e le equazioni di Navier--Stokes si riducono alle equazioni di Stokes,
\begin{equation}
\begin{cases}
 \dfrac{\partial \bm{u}^*}{\partial t^*} - \Delta^* \bm{u}^* + \bm{\nabla}^* p^* = \bm{0} \\
 \bm{\nabla}^* \cdot \bm{u}^* = 0 \ .
\end{cases}
\end{equation}

\section{Equazione di continuità e numero di Mach}
La forma adimensionale dell'equazione di continuità permette di valutare i limiti dell'approssimazione di corrente incomprimibile, che soddisfa il vincolo cinematico di incomprimibilità $\bm{\nabla} \cdot \bm{u} = 0$.
L'equazione della massa viene scritta in forma convettiva,
\begin{equation}
 - \bm{\nabla} \cdot \bm{u} = \dfrac{1}{\rho}\dfrac{D \rho}{D t} \ .
\end{equation}
Ricordando che lo stato termodinamico di un sistema monocomponente monofase è definito da due variabili termodinamiche, il campo di pressione $p$ viene espresso in funzione del campo di densità $\rho$ e di entropia $s$, come $p(\rho,s)$. Il differenziale di questa relazione,
\begin{equation}
 d p = \left(\dfrac{\partial p}{\partial \rho}\right)_s d \rho +
       \left(\dfrac{\partial p}{\partial s}\right)_{\rho} d s \ ,
\end{equation}
può essere utilizzato per esprimere la derivata materiale della densità in funzione delle derivate materiali di pressione ed entropia,
\begin{equation}
 \dfrac{D \rho}{D t} = \dfrac{1}{\left(\partial p/\partial \rho\right)_s}\dfrac{D p}{D t} - \dfrac{\left(\partial p/\partial s\right)_{\rho}}{\left(\partial p/\partial \rho\right)_s}\dfrac{D s}{D t} = \dfrac{1}{c^2}\dfrac{D p}{D t} - \dfrac{\left(\partial p/\partial s\right)_{\rho}}{c^2}\dfrac{D s}{D t} \ ,
\end{equation}
avendo riconosciuto il quadrato della velocità del suono $c^2 = \left(\frac{\partial p}{\partial \rho}\right)_s$. L'equazione della massa diventa quindi
\begin{equation}
  - \bm{\nabla} \cdot \bm{u} = \dfrac{1}{\rho c^2}\dfrac{D p}{D t} - \dfrac{\left(\partial p/\partial s\right)_{\rho}}{\rho c^2}\dfrac{D s}{D t} \ .
\end{equation}
Per processi isentropici (o per i quali il secondo termine a destra dell'uguale è trascurabile), l'equazione della massa si riduce a
\begin{equation}
  - \bm{\nabla} \cdot \bm{u} = \dfrac{1}{\rho c^2}\dfrac{D p}{D t} \ .
\end{equation}
Utilizzando i valori di densità $\tilde{\rho}$, velocità $\tilde{U}$ e lunghezza $\tilde{L}$ caratteristici della corrente per costruire la scala dei tempi $\tilde{t} = \tilde{L}/\tilde{U}$ e per la pressione $\tilde{p} = \tilde{\rho} \tilde{U}^2$, si ottiene l'equazione della massa in forma adimensionale,
\begin{equation}
 \bm{\nabla}^* \cdot \bm{u}^* = -  \dfrac{M^2}{\rho^*} \dfrac{D p^*}{D t^*} \ ,
\end{equation}
nella quale si è iconosciuto il numero di Mach caratteristico della corrente, $M = \dfrac{\tilde{U}}{c}$, definito come il rapporto tra una velocità caratteristica e la velocità del suono in uno stato termodinamico di riferimento della corrente.
\'E immediato osservare che l'equazione di continuità della massa si riduce al vincolo di incomprimibilità quando il numero di Mach assume valori ridotti (e il campo di pressione non ha variazioni rapide).

\section{Equazioni di Boussinesq e numeri di Prandtl, Rayleigh e Grashof}

\subsection{Equazioni di Boussinesq}
Le equazioni di Boussinesq sono un modello approssimato delle equazioni complete del moto dei fluidi, ricavato sotto le ipotesi che:
\begin{itemize}
    \item il contributo di dissipazione nell'equazione dell'energia sia trascurabile;
    \item la densità dipenda linearmente dalla temperatura nel termine di forze di volume nell'equazione della quantità di moto.
\end{itemize}
La variazione della densità in funzione della densità diventa quindi
\begin{equation}
    d \rho(P, T) = \left(\dfrac{\partial \rho}{\partial P} \right)_T dP +  \left(\dfrac{\partial \rho}{\partial T} \right)_P dT \approx \left(\dfrac{\partial \rho}{\partial T} \right)_P dT = - \rho_0 \, \alpha \, dT \
\end{equation}
\begin{equation}
    \rightarrow \rho = \rho_0 \left( 1 - \alpha \, (T-T_0) \right) \ ,
\end{equation}
dove le derivate sono calcolate nello stato termodinamico di riferimento, $(\rho_0, \ T_0)$, ed è stato introdotto il coefficiente di dilatazione termica
\begin{equation}
  \alpha = - \dfrac{1}{\rho_0} \left(\dfrac{\partial \rho}{\partial T} \right)_P \ .
\end{equation}
%
% {\color{red} $e = c_v T$} {\color{red} \textbf{(scrivere l'equazione dell'entalpia a partire da quella dell'energia $\rho Dh/Dt = \rho c_P DT/Dt = \dots$)}}
Introducendo le approssimazioni elencate, l'espressione dell'energia interna $e = c_v T$ e la legge di Fourier per il flusso di calore per conduzione, $\bm{q} = -k \bm{\nabla} T$, nelle equazioni complete per una corrente incomprimibile di un fluido newtoniano,
\begin{equation}
    \begin{cases}
      \rho \dfrac{\partial \bm{u}}{\partial t} + \rho
      \left( \bm{u} \cdot \bm{\nabla} \right) \bm{u} -
      \mu \nabla^2 \bm{u} + \bm{\nabla} P = \rho \bm{g} \\
      \bm{\nabla} \cdot \bm{u} = 0 \\
      \rho \dfrac{\partial e}{\partial t} + \rho \bm{u} \cdot 
      \bm{\nabla} e = 2 \mu \mathbb{D}:\mathbb{D} - \bm{\nabla} \cdot \bm{q} \ ,
    \end{cases}
\end{equation}
si ottengono le equazioni di Boussinesq
\begin{equation}
    \begin{cases}
      \dfrac{\partial \bm{u}}{\partial t} + 
      \left( \bm{u} \cdot \bm{\nabla} \right) \bm{u} -
      \nu \nabla^2 \bm{u} + \dfrac{1}{\rho_0}\bm{\nabla} P = \big( 1 - \alpha ( T-T_0 ) \big) \bm{g} \\
      \bm{\nabla} \cdot \bm{u} = 0 \\
      \dfrac{\partial T}{\partial t} + \bm{u} \cdot 
      \bm{\nabla} T =  D \nabla^2 T \ ,
    \end{cases}
\end{equation}
avendo definito il coefficiente di diffusione termica $D = \dfrac{k}{\rho_0 c_v}$.

\subsection{Equazioni di Boussinesq: problema bidimensionale tra due superfici piane}

\paragraph{Condizioni al contorno}
Si considera ora la corrente che si sviluppa tra due superfici piane orizzontali infinite, a distanza $h$ l'una dall'altra, mantenute a temperatura costante: la temperatura vale $T_w$ sulla superficie inferiore e $T_c$ sulla superficie superiore. Viene definita la differenza di temperatura $\Delta T = T_w - T_c$.
\newline
Se le due superfici considerate sono superfici solide, la velocità su di esse è nulla. Se le due superfici sono superfici ``libere'' (di simmetria, a sforzo nullo) si annulla la derivata normale della velocità. Prendendo un sistema di assi ortogonali, con l'origine in corrispondenza della superficie inferiore, con l'asse $x$ parallelo e l'asse $z$ perpendicolare alla superficie, si possono riassumere così le condizioni al contorno,
\begin{equation}
    \text{wall: }
    \begin{cases}
      T(x,z=0) = T_w \\ T(x,z=h) = T_c \\
      \bm{u}(x,z=0) = \bm{0} \\ \bm{u}(x,z=h) = \bm{0}
    \end{cases}  \hspace{0.5cm}
    \text{free: } \left\{
    \begin{aligned}
      T(x,z=0) = T_w \ & \ , \ \ T(x,z=h) = T_c \\
      \dfrac{\partial u}{\partial z}(x,z=0) = 0 \ & \ , \ \ \dfrac{\partial u}{\partial z}(x,z=h) = 0 \\
      w(x,z=0) = 0 \ & \  , \  \  w(x,z=h) = 0 \ .
    \end{aligned} \right.
\end{equation}
Non ci sono condizioni al contorno in $x$, poichè la direzione è omogenea. Considereremo qui solo il problema con le condizioni al contorno ``free''.

\paragraph{Soluzione stazionaria non convettiva}
Esiste una soluzione stazionaria ($\partial / \partial t = 0$) del problema con fluido in quiete ($\bm{u} = \bm{0}$).
Il vincolo di incomprimibilità è soddisfatto identicamente. Sfruttando l'omogeneità della direzione $x$, la soluzione stazionaria indipendente dalla coordinata $x$ soddisfa le equazioni
\begin{equation}
    \begin{cases}
        \dfrac{1}{\rho_0}\dfrac{d P}{d z} = \alpha g (T-T_0)  \vspace{0.2cm} \\
        \dfrac{d^2 T}{d z^2} = 0 \ ,
    \end{cases}
\end{equation}
dotate delle opportune condizioni al contorno. La soluzione stazionaria del problema è
\begin{equation}
\begin{cases}
    \overline{T}(z) = T_w + (T_c-T_w) \dfrac{z}{h} =
    T_w - \Delta T \dfrac{z}{h} \\
    \overline{P}(z) = P_w + \alpha \rho_0 g \left[ (T_w-T_0) z
    - \dfrac{1}{2} \Delta T \dfrac{z^2}{h} \right] \ ,
\end{cases}
\end{equation}
avendo indicato con $P_w$ il valore della pressione in corrispondenza della superficie inferiore a $z = 0$.

\paragraph{Equazione delle fluttuazioni}
Viene definita la fluttuazione di temperatura $\tau(x,z)$,
\begin{equation}
\begin{aligned}
    \tau(x,z) = T(x,z) - \overline{T}(z) & = T(x,z) - T_w + \Delta T \dfrac{z}{h} \\
    \quad \rightarrow \quad T(x,z) - T_w & = \tau(x,z) - \Delta T \dfrac{z}{h} \ .
\end{aligned}
\end{equation}
Scegliendo la superficie inferiore a $z = 0$ per definire la condizione termodinamica di riferimento, $T_0 = T_w$.
le equazioni di Boussinesq diventano
\begin{equation}
    \begin{cases}
      \dfrac{\partial \bm{u}}{\partial t} + 
      \left( \bm{u} \cdot \bm{\nabla} \right) \bm{u} -
      \nu \nabla^2 \bm{u} + \dfrac{1}{\rho_0}\bm{\nabla} P = \left( 1 - \alpha \tau + \Delta T \dfrac{z}{h} \right) \bm{g} \\
      \bm{\nabla} \cdot \bm{u} = 0 \\
      \dfrac{\partial \tau}{\partial t} + \bm{u} \cdot 
      \bm{\nabla} \tau + w \dfrac{\partial \overline{T}}{\partial z}=  D \nabla^2 \tau \ .
    \end{cases}
\end{equation}
Inoltre è possibile raccogliere il primo e il terzo termine delle forze di galleggiamento sotto lo stesso operatore di gradiente che opera sul campo di pressione. Infatti, è possibile scrivere il termine di galleggiamento come
\begin{equation}
\begin{aligned}
    \left( 1 - \alpha \tau + \Delta T \dfrac{z}{h} \right) \bm{g} & = \alpha \tau g \bm{\hat{z}} - \bm{\nabla} \left( gz + \Delta T \dfrac{z^2}{2 h} \right) \ .
\end{aligned}
\end{equation}
Definendo una ``pressione generalizzata'' $P'$,
\begin{equation}
    P' = P + \rho_0 g z + \rho_0 \Delta T \dfrac{z^2}{2 h} \ ,
\end{equation}
le equazioni di Boussinesq diventano
\begin{equation}\label{eqn:Bouss-tau}
    \begin{cases}
      \dfrac{\partial \bm{u}}{\partial t} + 
      \left( \bm{u} \cdot \bm{\nabla} \right) \bm{u} -
      \nu \nabla^2 \bm{u} + \dfrac{1}{\rho_0}\bm{\nabla} P' = \alpha  g \tau \bm{\hat{z}} \\
      \bm{\nabla} \cdot \bm{u} = 0 \\
      \dfrac{\partial \tau}{\partial t} + \bm{u} \cdot 
      \bm{\nabla} \tau -\dfrac{\Delta T}{h} w =  D \nabla^2 \tau \ ,
    \end{cases}
\end{equation}
e le condizioni al contorno della temperatura vengono espresse anch'esse in funzione di $\tau$,
\begin{equation}\label{eqn:Bouss-tau-bc}
    \tau(x,z=0) = \tau(x,z=h) = 0 \ .
\end{equation}

\subsection{Equazioni di Boussinesq in forma adimensionale}
Si ricava la forma adimensionale delle equazioni (\ref{eqn:Bouss-tau}) e delle condizioni al contorno (\ref{eqn:Bouss-tau-bc}) utilizzando il teorema $\pi$ di Buckingham. 
\newline
Nel problema di Boussinesq compaiono 12 grandezze dimensionali 
(13 se si volesse considerare la componente $w$ della velocità $\bm{u}$ in maniera indipendente),
\begin{equation}
    \underbrace{\bm{x}, t}_{\text{tar. indip.}}, 
    \underbrace{\bm{u}, \tau, P'}_{\text{campi } f(\bm{x},t)},
    \underbrace{\rho_0, \nu, D, \alpha, g}_{\substack{ \text{\footnotesize{propr. del fluido}} \\ \text{\footnotesize{e del problema}} } }, 
    \underbrace{h, \Delta T}_{\substack{ \text{\footnotesize{dominio e}} \\ \text{\footnotesize{ condizioni al contorno}} } } \ ,
\end{equation}
e 4 grandezze fisiche fondamentali: massa $M$, lunghezza $L$, tempo $T$ e temperatura $\Theta$.
Secondo il teorema di Buckingham, il problema può quindi essere caratterizzato da 8 numeri adimensionali. Utilizzando la stessa scala di lunghezze per adimensionalizzare $\bm{x}$ e $h$ e la stessa scala di temperature per adimensionalizzare $\tau$ e $\Delta T$, sono sufficienti 6 numeri adimensionali.
\newline
\'E necessario scegliere 4 grandezze fisiche di riferimento indipendenti e, possibilmente, rappresentative del problema con le quali adimensionalizzare le altre. Il problema della convezione non forzata  descritto dalle equazioni di Boussinesq è caratterizzato dalla differenza di temperatura $\Delta T$  e dalla distanza $h$ delle superfici, dal fluido considerato e dall'intensità delle forze di volume. I campi di velocità $\bm{u}$, di ``temperatura'' $\tau$ e di ``pressione'' $P'$ sono un risultato, una conseguenza, delle condizioni al contorno e del fluido impiegato: non esistono scale di velocità e pressione indipendenti, mentre il campo di temperatura può essere scalato sulla differenza $\Delta T$. Non esiste nemmeno una scala indipendente dei tempi, poiché l'evoluzione del sistema è determinata dalle condizioni al contorno e dal fluido utilizzato.
\newline
Come grandezze dimensionali di riferimento indipendenti e  caratteristiche del problema vengono scelte la densità del fluido, il coefficiente di diffusione termica, la distanza tra le superfici e la loro differenza di temperatura:
\begin{equation}
    \tilde{\rho}=\rho_0, \ \tilde{D} = D, \ \tilde{L} = h, \ \tilde{\Theta} = \Delta T \ .
\end{equation}
%
\begin{table}[t]
    \centering
    \begin{tabular}{c|cc ccc ccccc cc}
    & $\bm{x}$ & $t$ & $\bm{u}$ & $\tau$ & $P'$ & $\rho_0$ & $\nu$ & $D$ & $\alpha$ & $g$ & $h$ & $\Delta T$\\ \hline
        M &   &   &   &   & 1 & 1 &   &   &   &   &   &   \\
        L & 1 &   & 1 &   &-1 &-3 & 2 & 2 &   & 1 & 1 &   \\
        T &   & 1 &-1 &   &-2 &   &-1 &-1 &   &-2 &   &   \\
 $\Theta$ &   &   &   & 1 &   &   &   &   &-1 &   &   & 1 \\
    \end{tabular}
    \caption{Teorema di Buckingham. Grandezze dimensionali e unità fisiche.}
    \label{tab:Bouss-pi-thm}
\end{table}
%
\newline
Ora è possibile scrivere ogni grandezza dimensionale come prodotto di una grandezza omogenea di riferimento (dimensionale) e del suo valore adimensionale. Si può quindi scrivere,
\begin{equation}\label{eqn:var-adim}
\begin{aligned}
    \bm{x} = \tilde{L} \bm{x}^*  \quad & , \quad t = \tilde{T} t^* \\
    \bm{u} = \tilde{U} \bm{u}^* \quad , \quad \tau & = \tilde{\Theta} \tau^* \quad , \quad P' = \tilde{P} P^{*'} \\
    \rho_0 = \tilde{\rho} \rho_0^* \quad  , \quad \nu = \tilde{\nu} \nu^* \quad , \quad D & = \tilde{D} D^* \quad , \quad \alpha = \tilde{\alpha} \alpha^* \quad , \quad g = \tilde{g} g^* \\
    h = \tilde{L} h^* \quad & , \quad \Delta T = \tilde{\Theta} \Delta T^* \ ,
\end{aligned}
\end{equation}
avendo utilizzato la stessa scala di lunghezza $\tilde{L}$ come riferimento per la coordinata spaziale indipendente $\bm{x}$ e la distanza $h$ tra le due superifici, e la stessa scala di temperatura $\tilde{\Theta}$ come riferimento per il campo di temperatura $\tau$ e la differenza di temperatura tra le due superfici $\Delta T$, come anticipato in precedenza. Le 12 grandezze dimensionali sono state adimensionalizzate usando 10 scale di riferimento: da queste è possibile ricavare 6 numeri adimensionali con cui descrivere il problema.
Inserendo le espressioni (\ref{eqn:var-adim}) nel problema di Boussinesq (\ref{eqn:Bouss-tau}), si ricava
\begin{equation}\label{eqn:Bouss-tau-adim-1}
    \begin{cases}
      \dfrac{\tilde{U}}{\tilde{T}}\dfrac{\partial \bm{u}^*}{\partial t^*} + \dfrac{\tilde{U}^2}{\tilde{L}}
      \left( \bm{u}^* \cdot \bm{\nabla}^* \right) \bm{u}^* -
      \dfrac{\tilde{\nu} \tilde{U}}{\tilde{L}^2} \nu^* \nabla^{*2} \bm{u}^* + \dfrac{\tilde{P}}{\tilde{\rho} \tilde{L}}\dfrac{1}{\rho_0^*}\bm{\nabla} P^{*'} = \tilde{\alpha} \tilde{g} \tilde{\theta} \alpha^*  g^* \tau^* \bm{\hat{z}} \\
      \dfrac{\tilde{U}}{\tilde{L}}\bm{\nabla}^* \cdot \bm{u}^* = 0 \\
      \dfrac{\tilde{\Theta}}{\tilde{T}}\dfrac{\partial \tau^*}{\partial t^*} + \dfrac{\tilde{U}\tilde{\Theta}}{\tilde{L}}\bm{u}^* \cdot 
      \bm{\nabla}^* \tau^* - \dfrac{\tilde{U}\tilde{\Theta}}{\tilde{L}}\dfrac{\Delta T^*}{h^*} w^* = \dfrac{\tilde{D}\tilde{\Theta}}{\tilde{L}^2} D^* \nabla^{*2} \tau^* \ ,
    \end{cases}
\end{equation}
con le conzioni al contorno ``free''
\begin{equation}
    \text{free: } \left\{
    \begin{aligned}
      \tilde{\Theta}\tau^*(\tilde{L}x^*,\tilde{L}z^*=0) = 0 \quad & , \quad  
      \tilde{\Theta}\tau^*(\tilde{L}x^*,\tilde{L}z^*=\tilde{L}h^*) = 0 \\
      \dfrac{\tilde{U}}{\tilde{L}}\dfrac{\partial u^*}{\partial z^*}(\tilde{L}x^*,\tilde{L}z^*=0) = 0 \quad  & , \quad 
      \dfrac{\tilde{U}}{\tilde{L}}\dfrac{\partial u}{\partial z^*}(\tilde{L}x^*,\tilde{L}z^*=\tilde{L}h^*) = 0 \\
      \tilde{U} w^*(\tilde{L}x^*,\tilde{L}z^*=0) = 0 \quad  & , \quad 
      \tilde{U} w^*(\tilde{L}x^*,\tilde{L}z^*=\tilde{L}h^*) = 0
    \end{aligned} \right.
\end{equation}
Con un abuso di notazione, d'ora in poi si indicano le grandezze adimensionali senza asterisco e i campi adimensionali vengono definiti come funzione delle variabili indipendenti adimensionali,
\begin{equation}
    \bm{u}(\bm{x},t) = \tilde{U} \bm{u}^*(\tilde{L} \bm{x}^*, \tilde{T} t^*) \quad \rightarrow \quad \tilde{U} \bm{u}(\bm{x},  t) \ .
\end{equation}
Le grandezze di riferimento delle grandezze costanti vengono scelte coincidenti con la grandezza stessa, cosicché le grandezze adimensionali relative sono uguali a 1,
\begin{equation}\label{eqn:var-adim-2}
\begin{aligned}
    \rho_0 = \tilde{\rho} \quad  , \quad \nu = \tilde{\nu} \quad , \quad D & = \tilde{D}  \quad , \quad \alpha = \tilde{\alpha}  \quad , \quad g = \tilde{g}  \\
    h = \tilde{L}  \quad & , \quad \Delta T = \tilde{\Theta}  \ .
\end{aligned}
\end{equation}
Dividendo l'equazione della quantità di moto per $\tilde{\nu}\tilde{U}/\tilde{L}^2$, il vincolo di incomprimibilità per $\tilde{U}/\tilde{L}$ e l'equazione dell'energia per $\tilde{D}\tilde{\Theta}/\tilde{L}^2$, il problema di Boussinesq diventa
\begin{equation}\label{eqn:Bouss-tau-adim-2}
    \begin{cases}
      \dfrac{\tilde{L}^2}{\tilde{\nu}\tilde{T}}\dfrac{\partial \bm{u}^*}{\partial t^*} + \dfrac{\tilde{U}\tilde{L}}{\tilde{\nu}}
      \left( \bm{u}^* \cdot \bm{\nabla}^* \right) \bm{u}^* -
      \nabla^{*2} \bm{u}^* + \dfrac{\tilde{P}\tilde{L}}{\tilde{\rho} \tilde{\nu} \tilde{U}}\bm{\nabla} P^{*'} = \dfrac{\tilde{\alpha} \tilde{g} \tilde{\Theta} \tilde{L}^2}{\tilde{\nu} \tilde{U}} \tau^* \bm{\hat{z}} \\
      \bm{\nabla}^* \cdot \bm{u}^* = 0 \\
      \dfrac{\tilde{L}^2}{\tilde{D}\tilde{T}}\dfrac{\partial \tau^*}{\partial t^*} + \dfrac{\tilde{U}\tilde{L}}{\tilde{D}}\bm{u}^* \cdot 
      \bm{\nabla}^* \tau^* - \dfrac{\tilde{U}\tilde{L}}{\tilde{D}} w^* = \nabla^{*2} \tau^* \ ,
    \end{cases}
\end{equation}
con le conzioni al contorno ``free''
\begin{equation}\label{eqn:Bouss-adim-2-bc}
    \text{free: }
    \left\{
    \begin{aligned}
      \tau^*(x^*,z^*=0) = 0 \qquad  & , \qquad 
      \tau^*(x^*,z^*=1) = 0 \\
      \dfrac{\partial u^*}{\partial z^*}(x^*,z^*=0) = 0 \qquad & , \qquad 
      \dfrac{\partial u^*}{\partial z^*}(x^*,z^*=1) = 0 \\
      w^*(x^*,z^*=0) = 0 \qquad & , \qquad w^*(x^*,z^*=1) = 0 \ .
    \end{aligned} \right.
\end{equation}
Nel problema (\ref{eqn:Bouss-tau-adim-2}-\ref{eqn:Bouss-adim-2-bc}) compaiono 6 numeri adimensionali. Siamo arrivati al risultato previsto dal teorema di Buckingham. Prima di andare avanti, conviene comunque fare un'osservazione. Solo 5 dei 6 numeri adimensionali trovati sono tra di loro indipendenti: in particolare solo 3 dei 4 numeri adimensionali
\begin{equation}
    \pi_1 = \frac{\tilde{L}^2}{\tilde{D}\tilde{T}} \ , \quad
    \pi_2 = \frac{\tilde{U}\tilde{L}}{\tilde{D}} \ , \quad
    \pi_3 = \frac{\tilde{L}^2}{\tilde{\nu}\tilde{T}} \ , \quad 
    \hat{\pi}_4 = \frac{\tilde{U}\tilde{L}}{\tilde{\nu}} = \pi_2 \dfrac{\pi_3}{\pi_1}
\end{equation}
sono linearmente indipendenti. Sembra di aver commesso un errore poiché abbiamo trovato una contraddizione del teorema di Buckingham. L'apparente errore si nasconde nel termine adimensionale $\frac{\tilde{\alpha} \tilde{g} \tilde{\theta} \tilde{L}^2}{\tilde{\nu} \tilde{U}}$. Questo termine infatti è il prodotto dei numeri adimensionali $\tilde{\alpha} \tilde{\theta}$ e $\frac{\tilde{g} \tilde{L}^2}{\tilde{\nu} \tilde{U}}$.
I sei numeri adimensionali indipendenti che caratterizzano il problema sono quindi
\begin{equation}
\begin{aligned}
    \pi_1 = \frac{\tilde{L}^2}{\tilde{D}\tilde{T}} \quad ,\quad
    \pi_2 & = \frac{\tilde{U}\tilde{L}}{\tilde{D}} \quad ,\quad
    \pi_3 = \frac{\tilde{L}^2}{\tilde{\nu}\tilde{T}} \\
    \pi_4 = \frac{\tilde{P}\tilde{L}}{\tilde{\rho}\tilde{\nu}\tilde{U}} \quad ,\quad
    \pi_5 & = {\tilde{\alpha}\tilde{\Theta}} \quad ,\quad
    \pi_6 = \frac{\tilde{g} \tilde{L}^2}{\tilde{\nu} \tilde{U}} \ .
\end{aligned}
\end{equation}
%
Poiché il coefficiente di dilatazione termica $\alpha$ e la forza per unità di volume $g$ comapiono sempre attraverso il loro prodotto, questo si può considerare come un'unica variabile, $\alpha g$. In questo caso, i 5 numeri adimensionali che descrivono il problema composto dalle 9 (10-1) scale di riferimento sono
\begin{equation}
  \pi'_1 = \pi_1, \  \pi'_2 = \pi_2, \ \pi'_3 = \pi_3, \ \pi'_4 = \pi_4, \ \pi'_5 = \pi_5 \pi_6 \ .  
\end{equation}
%
Non essendoci scale di velocità, tempo e pressione indipendenti, è possibile definire queste scale a partire dalle 4 grandezze fisiche di riferimento $\tilde{L}$, $\Delta \tilde{T}$, $\tilde{\rho}$, $\tilde{D}$, imponendo il valore unitario di alcuni parametri adimensionali,
\begin{equation}
    \begin{aligned}
      \pi'_1 = 1 & \quad \rightarrow \quad \tilde{T} = \dfrac{\tilde{L}^2}{\tilde{D}} \\
      \pi'_2 = 1 & \quad \rightarrow \quad \tilde{U} = \dfrac{\tilde{D}}{\tilde{L}} \\
      \pi'_4 = 1 & \quad \rightarrow \quad \tilde{P} = \dfrac{\tilde{\rho}\tilde{\nu}\tilde{U}}{\tilde{L}} \ .
    \end{aligned}
\end{equation}
Gli unici due parametri adimensionali caratteristici del problema rimangono
\begin{equation}
\begin{aligned}
 \Pi_1 = \pi'_3 = \dfrac{\tilde{L^2}}{\tilde{\nu} \tilde{L}^2/\tilde{D}} \qquad \rightarrow \qquad \Pi_1  & = \dfrac{\tilde{D}}{\tilde{\nu}} = \dfrac{1}{Pr} \\
 \Pi_5 = \pi'_5 = \dfrac{\tilde{\alpha}\tilde{\Theta} \tilde{g} \tilde{L}^2}{\tilde{\nu} \tilde{D}/\tilde{L}} \qquad \rightarrow \qquad \Pi_5 & = \dfrac{\tilde{\alpha}\tilde{g}\tilde{\Theta}  \tilde{L}^3}{\tilde{\nu} \tilde{D}} = Ra = \\
 & = \dfrac{\tilde{\alpha}\tilde{g}\tilde{\Theta}  \tilde{L}^3}{\tilde{\nu}^2}\dfrac{\tilde{\nu}}{\tilde{D}} = Gr \, Pr \ ,
\end{aligned}
\end{equation}
nei quali si possono riconoscere i numeri di Prandtl, $Pr$, di Rayleigh, $Ra$, e di Grashof, $Gr$.
%
\newline
La forma adimensionale del problema di Boussinesq tra due superfici piane è quindi
\begin{equation}\label{eqn:Bouss-tau-adim-3}
    \begin{cases}
      \dfrac{1}{Pr} \left[ \dfrac{\partial \bm{u}}{\partial t} +
      \left( \bm{u} \cdot \bm{\nabla} \right) \bm{u} \right] -
      \nabla^2 \bm{u} + \bm{\nabla} P' = Ra \, \tau \bm{\hat{z}} \\
      \bm{\nabla} \cdot \bm{u} = 0 \\
      \dfrac{\partial \tau}{\partial t} + \bm{u} \cdot 
      \bm{\nabla} \tau -  w = \nabla^{2} \tau \ ,
    \end{cases}
\end{equation}
con le conzioni al contorno ``free''
\begin{equation}\label{eqn:Bouss-adim-3-bc}
    \text{free: }
    \left\{
    \begin{aligned}
      \tau(x,z=0) = 0 \qquad  & , \qquad 
      \tau(x,z=1) = 0 \\
      \dfrac{\partial u}{\partial z}(x,z=0) = 0 \qquad & , \qquad 
      \dfrac{\partial u}{\partial z}(x,z=1) = 0 \\
      w(x,z=0) = 0 \qquad & , \qquad w(x,z=1) = 0 \ .
    \end{aligned} \right.
\end{equation}

\subsection{Equazione della vorticità e funzione di corrente nell'approssimazione di Boussinesq}
Dall'equazione della quantità di moto in (\ref{eqn:Bouss-tau-adim-3}) è possibile ricavare l'equazione della vorticità, applicandole l'operatore di rotore. Poichè il problema è piano e bidimensionale, il campo di vorticità ha componente non nulla solo fuori dal piano $xz$. Utilizzando un sistema di coordinate cartesiano, il campo di vorticità $\bm{\omega}(x,z,t) = \xi(x,z,t) \bm{\hat{y}}$ soddisfa l'equazione
\begin{equation}
    \dfrac{1}{Pr} \left[ \dfrac{\partial \xi}{\partial t} +
      \bm{u} \cdot \bm{\nabla} \xi \right] -
      \nabla^2 \xi = - Ra \, \dfrac{\partial \tau}{\partial x} \ .
\end{equation}
Si può poi introdurre la funzione di corrente $\psi$,
\begin{equation}
    u =   \dfrac{\partial \psi}{ \partial z} \quad , \quad 
    w = - \dfrac{\partial \psi}{ \partial x} \ ,
\end{equation}
in modo tale da soddisfare identicamente il vincolo di incomprimibilità. La componente $y$ del campo di vorticità diventa
\begin{equation}
    \xi = \dfrac{\partial u}{\partial z} - \dfrac{\partial w}{\partial x} =
    \dfrac{\partial^2 u}{\partial z^2} +
    \dfrac{\partial^2 w}{\partial x^2} = 
    \nabla^2 \psi \ ,
\end{equation}
e il termine advettivo di una funzione $f$ qualsiasi può essere scritta come un determinante,
\begin{equation}
    \bm{u} \cdot \bm{\nabla} f = u \dfrac{\partial f}{\partial x} + w \dfrac{\partial f}{\partial z} =
    \dfrac{\partial \psi}{ \partial z} \dfrac{\partial f}{\partial x} - \dfrac{\partial \psi}{ \partial x} \dfrac{\partial f}{\partial z} = \left| \begin{matrix} f_x & f_z \\ \psi_x & \psi_z \end{matrix} \right| =: \dfrac{\partial(f,\psi)}{\partial(x,z)} \ .
\end{equation}
%
Il sistema di equazioni del problema di Boussinesq diventa quindi
\begin{equation}\label{eqn:Bouss-vort-psi-tau}
    \begin{cases}
      \dfrac{\partial \xi}{\partial t} +
      \dfrac{\partial(\xi,\psi)}{\partial(x,z)} 
      = Pr \, \nabla^2 \xi 
      - Pr \, Ra \, \dfrac{\partial \tau}{\partial x} \\
      \dfrac{\partial \tau}{\partial t} +
      \dfrac{\partial(\tau,\psi)}{\partial(x,z)} =
      \nabla^{2} \tau + w \ .
    \end{cases}
\end{equation}

\subsection{Approssimazione di Fourier--Galerkin del problema di Boussinesq}
Utilizzando la geometria del dominio, è possibile espandere le funzioni che compaiono nelle equazioni (\ref{eqn:Bouss-vort-psi-tau}) come somma di prodotti di funzioni armoniche in $x$ e $z$, la cui ampiezza dipende dal tempo
\begin{equation}\label{eqn:harm-1}
\begin{aligned}
    \psi(x,z,t) & = \sum_m \sum_k a_{m,k}(t) \sin{(m\pi z + \phi^a_m)}\sin{(k\pi x + \phi^a_k)} \\
    \tau(x,z,t) & = \sum_m \sum_k b_{m,k}(t) \sin{(m\pi z + \phi^b_m)}\sin{(k\pi x + \phi^b_k)} \ .
\end{aligned}
\end{equation}
Le condizioni al contorno (\ref{eqn:Bouss-adim-3-bc}) del problema con due superfici infinite ``free'' impongono che la componente $w=-\partial{\psi}/\partial{x}$ e la derivata $\partial u/\partial z = \partial^2 \psi/\partial z^2$ siano nulle per $z = 0, \ 1$ per ogni istante temporale e per ogni valore di $x$. Le condizioni al contorno su $w$ impongono le seguenti condizioni sull'espanzione armonica delle funzioni,
\begin{equation}
    \begin{aligned}
      0 = \dfrac{\partial \psi}{\partial x}\Big|_{z=0} & = \sum_m \sum_k k \pi a_{m,k}(t) \sin{ \phi^a_m }\cos{(k\pi x + \phi^a_k)} \\
      & \hspace{4.0cm} \rightarrow \qquad \phi^a_m = 0 \ , \\
      0 = \dfrac{\partial \psi}{\partial x}\Big|_{z=1} & = \sum_m \sum_k k \pi a_{m,k}(t) \sin{ m \pi }\cos{(k\pi x + \phi^a_k)} \\
      & \hspace{4.0cm} \rightarrow \qquad m \in \mathbb{Z} \ .
    \end{aligned}
\end{equation}
Le stesse condizioni derivano dalle condizioni al contorno su $\partial u/\partial z$.
%
Le condizioni al contorno sulla temperatura in impongono le seguenti condizioni sull'espansione armonica della funzione $\tau$
\begin{equation}
    \begin{aligned}
      0 = \tau \Big|_{z=0} & = \sum_m \sum_k b_{m,k}(t) \sin{ \phi^b_m }\sin{(k\pi x + \phi^b_k)} \\
      & \hspace{4.0cm} \rightarrow \qquad \phi^b_m = 0 \ ,\\
      0 = \tau \Big|_{z=1} & = \sum_m \sum_k b_{m,k}(t) \sin{ m \pi }\sin{(k\pi x + \phi^b_k)} \\
      & \hspace{4.0cm} \rightarrow \qquad m \in \mathbb{Z} \ .
    \end{aligned}
\end{equation}
A causa dell'omogeneità della direzione $x$, nella quale il dominio è infinito, non ci sono condizioni sul numero d'onda $k$, che può assumere tutti i valori $\in \mathbb{R}$, e sulla fase delle armoniche in $x$. Le espansioni (\ref{eqn:harm-1}) possono quindi essere scritte come
\begin{equation}\label{eqn:harm-2}
\begin{aligned}
    \psi(x,z,t) & = \sum_{m \in \mathbb{Z}} \sum_k a^1_{m,k}(t) \sin{(m\pi z)} \sin{(k\pi x )} +  a^2_{m,k}(t) \sin{(m\pi z)} \cos{(k \pi x)} \\
    \tau(x,z,t) & = \sum_{m \in \mathbb{Z}} \sum_k b^1_{m,k}(t) \sin{(m\pi z)} \sin{(k\pi x )} + b^2_{m,k}(t) \sin{(m\pi z)} \cos{(k \pi x)}  \ .
\end{aligned}
\end{equation}

\subsection{Dal problema di Boussinesq al modello di Lorenz}
Le espansioni (\ref{eqn:harm-2}) possono essere \textit{brutalmente} troncate per ottenere un modello dinamico di dimensione $N_d = 3$ partendo dal modello continuo, che ha dimensione infinita. Le espansioni (\ref{eqn:harm-2}) vengono troncate mantenendo solo 3 termini
\begin{equation}\label{eqn:harm-3}
\begin{aligned}
    \psi(x,z,t) & = a(t) \sin{(\pi z)} \sin{(k\pi x )}  \\
    \tau(x,z,t) & = b(t) \sin{(\pi z)} \cos{(k\pi x )} + c(t) \sin{(2 \pi z)}   \ ,
\end{aligned}
\end{equation}
avendo definito $a(t) = a^1_{1,k}(t)$, $b(t) = b^2_{1,k}(t)$, $c(t) = b^1_{2,0}(t)$.
Usando le espansioni (\ref{eqn:harm-3}), la componente $y$ del campo di vorticità $\xi = \nabla^2 \psi$ diventa
\begin{equation}
    \xi = - \pi^2 (1 + k^2) \psi = - \pi^2 (1 + k^2) a(t) \sin{(\pi z)} \sin{(k\pi x )}
\end{equation}
%
I due determinanti che compaiono nelle equazioni (\ref{eqn:Bouss-vort-psi-tau}) valgono
\begin{equation}
\begin{aligned}
     \dfrac{\partial (\xi, \psi)}{\partial (x,z)} = &
     \left[ -\pi^3 k(1+k^2) a(t) \sin{(\pi z)}  \cos{(k \pi x)} \right]
     \left[ a(t) \pi \cos{(\pi z)} \sin{(k \pi x)} \right] + \\
      - & \left[  -\pi^3 (1+k^2) a(t) \cos{(\pi z)}  \sin{(k \pi x)} \right]
      \left[ a(t) \pi k \sin{(\pi z)} \cos{(k \pi x)} \right] = 0 \ ,
\end{aligned}
\end{equation}
e
\begin{equation}
\begin{aligned}
    \dfrac{\partial (\tau, \psi)}{\partial (x,z)} = &
    \left[ - \pi k b(t) \sin{(\pi z )} \sin{(k \pi x)} \right]
    \left[ a(t) \pi \cos{(\pi z)} \sin{(k \pi x)} \right] + \\
    - & \left[ \pi b(t) \cos{(\pi z )} \cos{(k \pi x)} + 2\pi c(t) \cos{(2\pi z)} \right]
    \left[ a(t) \pi k \sin{(\pi z)} \cos{(k \pi x)} \right] = \\
    = & - k \pi^2 a(t) b(t) \dfrac{\sin{( 2 \pi z)}}{2} -
    2 k \pi^2 a(t)c(t) \sin(\pi z) \cos(2\pi z) \cos(k \pi x) \ .
\end{aligned}
\end{equation}
%
I laplaciani che compaiono nelle equazioni (\ref{eqn:Bouss-vort-psi-tau}) valgono
\begin{equation}
    \nabla^2 \xi = -\pi^2 (1+k^2) \xi = \pi^4 (1+k^2)^2 a(t) \sin{(\pi z)} \sin{(k \pi x)} \ , 
\end{equation}
e
\begin{equation}
    \nabla^2 \tau = -\pi^2 (1+k^2) b(t) \sin{(\pi z)}\cos{(k \pi x)} - 4 \pi^2 c(t) \sin{(2\pi x)} \ .
\end{equation}
%
Il numero di Prantl viene indicato come $Pr = \sigma$, il numero di Rayleigh come $Ra = R$.
Inserendo le espansioni (\ref{eqn:harm-3}) all'interno delle equazioni (\ref{eqn:Bouss-vort-psi-tau}), il problema troncato di Boussinesq diventa,
\begin{equation}
    \begin{cases}
- \sigma \pi^2 (1+k^2) \dot{a}(t) \sin{(\pi z)} \sin{(k\pi x )} = \sigma \pi^4 (1+k^2)^2 a(t) \sin{(\pi z)} \sin{(k \pi x)} + \\
      \hspace{6.0cm} + \sigma R \, \pi k \, b(t) \sin{(\pi z)}\sin{(k \pi x)} \\
     \dot{b}(t)\sin{(\pi z)} \cos{(k\pi x )} + \dot{c}(t) \sin{(2\pi x)} + \\
     \hspace{2.0cm} - k \pi^2 a(t) b(t) \dfrac{\sin{(2 \pi z)}}{2} -
    2 k \pi^2 a(t)c(t) \sin(\pi z) \cos(2\pi z) \cos(k \pi x) = \\ 
    \hspace{1.5cm} = -\pi^2 (1+k^2) b(t) \sin{(\pi z)}\cos{(k \pi x)} - 4 \pi^2 c(t) \sin{(2\pi x)} + \\
    \hspace{2.0cm} - \pi k a(t) \sin{(\pi z)} \cos{(k \pi x)} \ .
    \end{cases}
\end{equation}
Raccogliendo il termine $\sin{(\pi z)} \sin{(k \pi x)}$ nell'equazione della vorticità si ottiene l'equazione
\begin{equation}
    - \pi^2 (1+k^2) \dot{a} = \sigma \pi^4 (1+k^2)^2 a(t) +  \sigma R \, \pi k \, b(t) \ .
\end{equation}
L'equazione della temperatura viene ``proiettata'' sulle funzioni di base $\sin{(\pi z)} \cos{(k \pi x)}$ e $\sin{(2 \pi x)}$ e sfruttando l'ortogonalità delle funzioni armoniche.
La proiezione consiste nella moltiplicazione dell'equazione per le funzioni di base $\sin{(\pi z)} \cos{(k \pi x)}$ e nell'integrazione in $(x,z) \in \left[0,\frac{2}{k}\right]\times\left[0,1\right]$. Usando il valore degli integrali,
\begin{equation}
\begin{aligned}
 \int_{x=0}^{2/k} \sin{(k \pi x)}^2 dx & = \dfrac{1}{2} \int_{x=0}^{2/k} \left[ 1 - \cos{(2 k \pi x)} \right] dx = \dfrac{1}{k} \\
 \int_{x=0}^{2/k} \sin{(k \pi x)}\cos{(k \pi x)} dx & = \dfrac{1}{k \pi} \int_{x=0}^{2/k} \sin{(k \pi x)} d \big( \sin{(k \pi x)} \big) = 0 \\
 \int_{x=0}^{2/k} \cos{(k \pi x)}^2 dx & = \dfrac{1}{k} \ ,
\end{aligned}
\end{equation}
e degli integrali
\begin{equation}
\begin{aligned}
 \int_{z=0}^{1} \sin{(\pi z)}^2 dz & = \dfrac{1}{2} \int_{x=0}^{1} \left[ 1 - \cos{(2\pi z)} \right] dz = \dfrac{1}{2} \\
 \int_{z=0}^{1} \sin{(\pi z)} \sin{(\pi z)} \cos{(2\pi z)} dz  & =
 \dfrac{1}{2} \int_{z=0}^{1} \left[ 1 - \cos{(2\pi z)} \right]  \cos{(2\pi z)} dz = \\
 & = - \dfrac{1}{2} \int_{z=0}^{1} \cos^2{(2\pi z)}  dz = 
  - \dfrac{1}{4} \ .
\end{aligned}
\end{equation}
%
La proiezione dell'equazione della vorticità sulla funzione $\sin{(\pi z)} \cos{(k \pi x)}$ è
\begin{equation}
    \dfrac{1}{2}\dot{b}(t) - \dfrac{1}{4} 2 k \pi^2 a(t) c(t) =
    -\dfrac{1}{2} \pi^2 (1+k^2) b(t) - \dfrac{1}{2}\pi k a(t) \ ,
\end{equation}
mentre la proiezione dell'equazione della vorticità sulla funzione $\sin{(2 \pi z)}$ è
\begin{equation}
    \dfrac{1}{k}\dot{c}(t) - \dfrac{1}{k} \dfrac{\pi^2 k}{2} a(t) b(t) =
    - \dfrac{1}{k} 4 \pi^2 c(t) \ .
\end{equation}
%
% L'equazione della temperatura contiene principalemente due tipi di termini: i termini in $\sin{(\pi z)} \cos{(k \pi x)}$ e i termini in $\sin{(2 \pi x)}$. Esiste infine il termine $-2 k \pi^2 a(t)c(t) \sin(\pi z) \cos(2\pi z) \cos(k \pi x) $ che non può essere immediatamente raccolto insieme agli altri termini.
%
% Sfruttando le proprietà delle funzioni armoniche, il prodotto $\sin(2 \pi z) \cos(2\pi z) \cos(k \pi x) $ viene scritto 
% \begin{equation}
%     \sin(\pi z) \cos(2\pi z) \cos(k \pi x) =
%     \sin(\pi z) \left( 1 - 2 \sin^2{(\pi z)} \right) \cos(k \pi x)
% \end{equation}
%
\noindent
Le equazioni diventano quindi
\begin{equation}
    \begin{cases}
   - \pi^2 (1+k^2) \dot{a} = \sigma \pi^4 (1+k^2)^2 a(t) +  \sigma R \, \pi k \, b(t) \\
    \dot{b} = -\pi^2 (1+k^2) b(t) + \pi^2 k a(t)c(t)  - \pi k a(t)  \\
    \dot{c} = \dfrac{\pi^2 k}{2} a(t) b(t) - 4 \pi^2 c(t) \ .
    \end{cases}
\end{equation}
Partendo da queste equazioni, si introduce qualche cambio di variabile per riportarsi all'espressione classica del sistema di Lorenz.
Viene introdotto il tempo $t' = \pi^2 (k^2 + 1) t$, cosicché
\begin{equation}
    \dot{f} = \dfrac{df}{dt} = \dfrac{dt'}{dt}\dfrac{df}{dt'} =
    \pi^2 (k^2 + 1) \dfrac{df}{dt'} \ .
\end{equation}
Con un abuso di notazione, d'ora in poi si indica $\dot{(\ )}$ la derivata rispetto a $t'$. La stessa variabile $t'$ viene indicata con $t$. Le equazioni diventano
\begin{equation}
    \begin{cases}
    \dot{a}(t) = \sigma a(t) +  \sigma R \dfrac{k}{\pi^3 (k^2+1)^2} b(t) \\
    \dot{b}(t) = - b(t) + \dfrac{ k}{k^2+1} a(t)c(t)  - \dfrac{ k}{\pi(k^2+1)} a(t)  \\
    \dot{c}(t) = \dfrac{k}{2(k^2+1)} a(t) b(t) - \dfrac{4}{k^2+1}  c(t) \ .
    \end{cases}
\end{equation}
Viene definito infine il cambio di variabili 
\begin{equation}
    \begin{cases}
     X(t) = \dfrac{k}{\sqrt{2}(k^2+1)} a(t) \\
     Y(t) = \dfrac{k}{\sqrt{2}(k^2+1)}
     \left[-\dfrac{R k}{\pi^3 (k^2+1)^2}\right] b(t) \\
     Z(t) = \left[-\dfrac{R k^2}{\pi^3 (k^2+1)}\right] c(t)
    \end{cases}
\end{equation}
che porta alla forma classica del sistema dinamico di Lorenz
\begin{equation}
    \begin{cases}
      \dot{X} = - \sigma X + \sigma Y \\
      \dot{Y} = - Y + \rho X - X Z \\
      \dot{Z} = - \beta Z + X Y \ ,
    \end{cases}
\end{equation}
avendo definito i parameteri 
\begin{equation}
    \rho = \dfrac{R k^2}{\pi^4 (k^2+1)^2} \qquad , \qquad
    \beta = \dfrac{4}{k^2+1} \ .
\end{equation}

\subsection{Sistema dinamico di Lorenz}
In questa sezione si compie lo di studio di stabilità del sistema dinamico di Lorenz,
 come primo esempio di studio di stabilità di un sistema fluidodinamico.
Si introduce il concetto di \textbf{stabilità strutturale} e di \textbf{spazio
 delle fasi}. Si studia la stabilità del sistema di Lorenz al variare del parametro
 $\rho$, mantenendo costante il valore di $\sigma$ e $\beta$.
Lorenz usò come valori $\sigma = 10$ e $\beta = 8/3$.

