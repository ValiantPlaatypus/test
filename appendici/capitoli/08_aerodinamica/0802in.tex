\noindent
\begin{tabular}{c}
\begin{minipage}[b]{0.95\textwidth}
\begin{exerciseS}[Corrente attorno al cilindro]
Una corrente piana con velocit\`{a} asintotica $U_\infty=10\ m/s$ investe un profilo circolare
di raggio $a=0.1\ m$. Determinare il valore di circolazione $\Gamma$ affinch\'{e} nel 
punto sulla superficie del cilindro posto a $\theta=\pi/3$ la velocit\`{a} aumenti fino al valore 
$2U_\infty$ in modulo.

($\Gamma = -1.68\ m^2 / s$, 23.45$\ m^2 / s$)
\end{exerciseS}
\end{minipage}
\end{tabular}


\sol

\partone
  Flusso non viscoso 2D, incomprimibile e irrotazionale attorno al cilindro. Circolazione

\parttwo
 Una volta scritte (come si ricavano?) le componenti della velocità nel campo di moto, si
impongono le condizioni richieste dal problema per determinare il valore di circolazione necessario.

\begin{equation}
\begin{cases}
  u_r (r,\theta) = U_\infty \displaystyle \left(1 - \frac{a^2}{r^2}\right)\cos{\theta} \\
  u_\theta (r,\theta) = - U_\infty \displaystyle \left(1 + \frac{a^2}{r^2}\right)\sin{\theta} + \frac{\Gamma}{2\pi r}
\end{cases}
\end{equation}

\vspace{0.2cm}
Si impongono ora le condizioni del problema. Sulla superficie del cilindro la componente radiale è nulla (condizioni al contorno). Quindi:

\begin{equation}
  |\bm{u}(a,\theta)| = |u_\theta(a,\theta)| = \Big| - 2 U_\infty \sin{\theta} + \frac{\Gamma}{2 \pi a} \Big|
\end{equation}

E quindi
\vspace{0.2cm}
\begin{equation}
\begin{aligned}
  2 U_\infty & = \Big| - 2 U_\infty \sin{\frac{\pi}{3}} + \frac{\Gamma}{2 \pi a} \Big| \\
  \pm 2 U_\infty & = - 2 U_\infty \frac{\sqrt{3}}{2} + \frac{\Gamma}{2 \pi a} \\
  \\
  \Rightarrow \Gamma  & = 2 \pi a U_\infty (\sqrt{3} \pm 2)
\end{aligned}
\end{equation}

\begin{equation}
  \Rightarrow \Gamma = 
  \begin{cases}
    -1.684 \ m^2/s \\
    23,449 \ m^2/s
  \end{cases}
\end{equation}

