\noindent
\begin{tabular}{c}
\begin{minipage}[b]{0.95\textwidth}
\begin{exerciseS}[Separazione su parete piana]
Si assuma che il profilo di velocit\`{a} $u(x,y)$ dello strato limite sulla superficie di un corpo
sia approssimabile con la seguente legge
$$
 u = 1-e^{-y/\sqrt{x}},
$$
dove $u$ \`{e} la velocit\`{a} adimensionalizzata rispetto alla velocit\`{a} esterna, $x$ \`{e} la 
coordinata adimensionale di parete localmente rettilinea e $y$ la coordinata adimensionale in direzione
normale alla parete stessa. Determinare l'andamento dello spessore di spostamento $\delta$ in funzione 
della coordinata $x$ lungo la parete. Lo strato limite in questione separa? Se si per quale valore di $x$?
\vspace{0.5cm}

($\delta(x)=\sqrt{x}$, lo strato limite non separa mai se non nel limite di $x\rightarrow\infty$)
\end{exerciseS}
\end{minipage}
\end{tabular}



\sol

\partone
 Separazione. Spessori di strato limite.

\begin{equation}
  \delta(x) = \int_{0}^{\infty} \left( 1 - \frac{u(y)}{U_e} \right) dy
\end{equation}

\parttwo

\begin{itemize}
\item Spessore di spostamento. Il profilo di velocità è già adimensionalizzato sulla "velocità esterna". Utilizzando la definizione di spessore di spostamento, si ottiene:
\begin{equation}
\begin{aligned}
  \delta(x) & = \int_{0}^{\infty} \left( 1 - u(y) \right) dy = \\
  & = \int_{0}^{\infty} (1 - (1-e^{-y/\sqrt{x}}))dy  = \\
  & = \int_{0}^{\infty} e^{-y/\sqrt{x}} dy = \\
  & = -\sqrt{x} [e^{-y/\sqrt{x}}]\big|_{y=0}^{\infty}
\end{aligned}
\end{equation}

E quindi: $\delta(x) = \sqrt{x}$.



\item Separazione. La condizione di separazione è $\frac{\partial u}{\partial y}\big|_{y=0} = 0$.

\begin{equation}
\begin{aligned}
  \frac{\partial u}{\partial y}  = 
  \frac{\partial}{\partial y} \displaystyle \left[   1-e^{-y x^{-1/2}} \right] = 
   x^{-1/2} e^{-y x^{-1/2}}
\end{aligned}
\end{equation}



Si osserva che $\frac{\partial u}{\partial y}\big|_{y=0}$ non si annulla mai:
\begin{equation}
  \displaystyle\frac{\partial u}{\partial y}\displaystyle\Big|_{y=0} = \frac{1}{\sqrt{x}}
\end{equation}

\end{itemize}

