\noindent
\begin{exerciseS}[Spessori di strato limite: strato limite laminare]
Dato il profilo di velocità, determinare il rapporto di forma $H$.
\begin{equation}
  u(x,y) = 
  \begin{cases}
    U(x) \displaystyle\left( 2 \frac{y}{\delta(x)} - \frac{y^2}{\delta^2(x)} \right) &  \qquad y \le \delta(x) \\
    U(x) &  \qquad y > \delta(x)
  \end{cases}
\end{equation}

($H = 5/2$)
\end{exerciseS}

\sol

\partone Spessori di strato limite. Rapporto di forma $H = \delta_1 / \delta_2$.

\begin{equation}
\begin{aligned}
   \delta_1(x) & = \int_{0}^\infty \displaystyle \left( 1 - \frac{u(x,y)}{U(x)} \right) dy \\
   \delta_2(x) & = \int_{0}^\infty \frac{u(x,y)}{U(x)} \displaystyle \left( 1 - \frac{u(x,y)}{U(x)} \right) dy
\end{aligned}
\end{equation}

\parttwo L'esercizio viene risolto calcolando prima gli integrali nelle definizioni degli spessori di strato limite
e poi il loro rapporto.

Lo spessore di spostamento:
\begin{equation}
\begin{aligned}
  \delta_1 & = \int_{0}^\infty \displaystyle \left( 1 - \frac{u(x,y)}{U(x)} \right) dy  = \\
           & = \int_{0}^{\delta(x)} \displaystyle \left( 1 - \frac{u(x,y)}{U(x)} \right) dy +
           \underbrace{\int_{\delta(x)}^\infty \displaystyle \left( 1 - \frac{u(x,y)}{U(x)} \right) dy}_{=0}  =\\
           & = \int_{0}^{\delta(x)} \displaystyle \left( 1 - 2 \frac{y}{\delta(x)} + \frac{y^2}{\delta^2(x)} \right) dy = \\
           & = \frac{1}{3}\delta(x)
\end{aligned}
\end{equation}

Lo spessore di quantità di moto:
\begin{equation}
\begin{aligned}
  \delta_2 & = \int_{0}^\infty \frac{u(x,y)}{U(x)} \displaystyle \left( 1 - \frac{u(x,y)}{U(x)} \right) dy \\
           & = \int_{0}^{\delta(x)} \frac{u(x,y)}{U(x)}\displaystyle \left( 1 - \frac{u(x,y)}{U(x)} \right) dy +
           \underbrace{\int_{\delta(x)}^\infty \frac{u(x,y)}{U(x)}\displaystyle \left( 1 - \frac{u(x,y)}{U(x)} \right) dy}_{=0}  =\\
           & = \frac{2}{15}\delta(x)
\end{aligned}
\end{equation}

Il rapporto di forma vale quindi $H = 5/2 $.


