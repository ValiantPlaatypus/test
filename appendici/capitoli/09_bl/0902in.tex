\noindent
\begin{exerciseS}[Spessori di strato limite: strato limite turbolento]
Dato il profilo di velocità, determinare il rapporto di forma $H$.
\begin{equation}
  \frac{u(x,y)}{U(x)} = 
  \begin{cases}
    \displaystyle\left( \frac{y}{\delta(x)} \right)^{\frac{1}{7}} &  \qquad y \le \delta(x) \\
    1 &  \qquad y > \delta(x)
  \end{cases}
\end{equation}

($H = 9/7$)
\end{exerciseS}

\sol

\partone Spessori di strato limite. Rapporto di forma $H = \delta_1 / \delta_2$.

\begin{equation}
\begin{aligned}
   \delta_1(x) & = \int_{0}^\infty \displaystyle \left( 1 - \frac{u(x,y)}{U(x)} \right) dy \\
   \delta_2(x) & = \int_{0}^\infty \frac{u(x,y)}{U(x)} \displaystyle \left( 1 - \frac{u(x,y)}{U(x)} \right) dy
\end{aligned}
\end{equation}

\parttwo L'esercizio viene risolto calcolando prima gli integrali nelle definizioni degli spessori di strato limite
e poi il loro rapporto.

Lo spessore di spostamento:
\begin{equation}
\begin{aligned}
  \delta_1 & = \int_{0}^\infty \displaystyle \left( 1 - \frac{u(x,y)}{U(x)} \right) dy  = \\
           & = \int_{0}^{\delta(x)} \displaystyle \left( 1 - \frac{u(x,y)}{U(x)} \right) dy +
           \underbrace{\int_{\delta(x)}^\infty \displaystyle \left( 1 - \frac{u(x,y)}{U(x)} \right) dy}_{=0}  =\\
           & = \frac{1}{8}\delta(x)
\end{aligned}
\end{equation}

Lo spessore di quantità di moto:
\begin{equation}
\begin{aligned}
  \delta_2 & = \int_{0}^\infty \frac{u(x,y)}{U(x)} \displaystyle \left( 1 - \frac{u(x,y)}{U(x)} \right) dy \\
           & = \int_{0}^{\delta(x)} \frac{u(x,y)}{U(x)}\displaystyle \left( 1 - \frac{u(x,y)}{U(x)} \right) dy +
           \underbrace{\int_{\delta(x)}^\infty \frac{u(x,y)}{U(x)}\displaystyle \left( 1 - \frac{u(x,y)}{U(x)} \right) dy}_{=0}  =\\
           & = \frac{7}{72}\delta(x)
\end{aligned}
\end{equation}

Il rapporto di forma vale quindi $H = 9/7 $.

\vspace{0.5cm}
\textit{Osservazione.} Questo profilo di velocità viene usato come approssimazione del profilo
di strato limite turbolento. Questo profilo ha $\frac{\partial u}{\partial y}\big|_{y=0}$ infinita, che 
implica sforzo a parete infinito. 
Una formula per lo sforzo di parete, associata a questo profilo di velocità è:
\begin{equation}
  \tau_w = 0.0225 \rho U^2 \displaystyle \left( \frac{\nu}{U \delta} \right) ^ {\frac{1}{4}}
\end{equation}

