Prima di addentrarsi nell'introduzione ai tensori (e quindi all'algebra multilineare) occorre 
 avere abbastanza chiari i concetti di algebra lineare, che riguarda lo studio degli spazi 
 vettoriali. \'E impensabile riportare qui un riassunto soddisfacente di algebra lineare: lo
 scopo della prima sezione è richiamare alcune definizioni, come ad esempio quella di spazio 
 vettoriale e di funzione lineare, e di ricordare che la struttura di spazio vettoriale non è
 esclusiva di una rappresentazione ``a freccina'' dei vettori, ma è una struttura del tutto generale.

\subsection{Prima di iniziare}

 Uno \textbf{spazio vettoriale} è una struttura algebrica composta da un campo $K$, da un insieme $V$
 di elementi e dalle due operazioni di somma e moltiplicazione per uno scalare, rispetto alle quali l'insieme
 $V$ deve essere chiuso (presi qualsiasi elementi di $V$, le due operazioni agenti su di essi devono avere
 come risultato un elemento di $V$). Il campo $K$ può essere ad esempio il campo dei numeri reali o quello dei
 numeri complessi; gli elementi di $K$ vengono definiti \textit{scalari}. Gli elementi di $V$ sono i \textit{vettori}.
 Le due operazioni di somma e moltiplicazione per uno scalare devono soddifare proprietà che qui non verranno
 riportate.
 
 Prima di riportare un esempio, si ricorda la definizione di \textbf{funzione lineare}. Una funzione $f$ è lineare se
 è additiva e omogenea, cioè se
 \begin{equation}
  \begin{cases}
   f(a+b) = f(a) + f(b) \\
   f(\alpha a) = \alpha f(a)
  \end{cases}
 \end{equation}
 Forse non è così scontato che la funzione $y(x) = m x + q$ NON rappresenta un legame lineare tra $x$ e $y$ (la funzione
 è affine, non lineare).
 
 Si riporta quindi un esempio di spazio vettoriale ``non convenzionale'': l'insieme delle funzioni lineari $f: \mathbb{R}^n \rightarrow
  \mathbb{R}$. Una funzione lineare di $n$ variabili può essere definita come:
  \begin{equation}
   f(\bm{x}) = a_1 x_1 + \dots a_n x_n
  \end{equation}
  L'insieme delle funzioni lineari forma uno spazio vettoriale quando vengono definite le operazioni
  di somma e moltiplicazione per uno scalare come
 \begin{equation}
  \begin{cases}
   (f+g)(\bm{x})    = f(\bm{x}) + g(\bm{x}) \\
   f(\alpha \bm{x}) = \alpha f(\bm{x})
  \end{cases}
 \end{equation}
 
 \begin{esercizio}
  Si verifichi che l'insieme delle funzioni lineari $f: \mathbb{R}^n \rightarrow
  \mathbb{R}$ è chiuso rispetto alle due operazioni. Inoltre si verifichi che la dimensione dello spazio è $n$.
  (Hint: quanti sono i coefficienti necessari e sufficienti a definire tutti gli elementi dello spazio?)
  \end{esercizio}
\newpage

Le equazioni che descrivono i fenomeni fisici hanno carattere tensoriale,
 cioè sono indipendenti dal sistema di coordinate nelle quali vengono scritte.
\'E importante capire la natura tensoriale delle leggi fisiche, capirne
 l'indipendenza dal sistema di coordinate ed essere in grado di scrivere
 correttamente le componenti nei sistemi di coordinate che sono più vantaggiosi
 per una descrizione semplice del fenomeno fisico e per la soluzione di problemi.

\vspace{0.2cm}
In questo capitolo verrà usata la notazione di Einstein: è sottointesa la sommatoria
 sugli indici ripetuti in una espressione. Per chiarezza
\begin{equation}
 a_k b_k = \displaystyle\sum_k a_k b_k
\end{equation}

\subsection{Prime definizioni}

 Sia $\mathcal{V}$ uno spazio vettoriale, sia $\bm{v}$ un elemento di $\mathcal{V}$
 e $\{ \bm{b}_k \}_{k=1:N}$ una base di $\mathcal{V}$; è quindi possibile scrivere
 l'elemento $\bm{v}$ in componenti rispetto alla base $\{ \bm{b}_k \}$
\begin{equation}
  \bm{v} = v^k \bm{b}_k
\end{equation}
 Le componenti $v^k$ di $\bm{v}$ sono definite \textbf{componenti contravarianti} del vettore $\bm{v}$.
 
 Si definisce $\mathcal{V}^*$ \textbf{spazio duale} di $\mathcal{V}$, come lo spazio vettoriale
 (indicato anche come $\mathscr{L}(\mathcal{V},K)$)
 formato da tutte le funzioni lineari da $\mathcal{V}$ a un campo $K$, come può essere
 ad esempio il campo dei numeri reali $\mathbb{R}$.
 Un elemento $\bm{f} \in \mathcal{V}^*$ è quindi una funzione $\bm{f}: \mathcal{V} \rightarrow K $ che
 prende un elemento $\bm{u} \in \mathcal{V}$ e resistuisce un valore in $K$. A volte si usa indicare
\begin{equation}
  \bm{f}(\bm{u}) = \langle \bm{f} , \bm{u} \rangle
\end{equation}
 Viene definita la base duale della base $\{ \bm{b}_k \}_{k=1:N}$ di $\mathcal{V}$: la
 \textbf{base duale} $\{ \bm{b}^k \}_{k=1:N}$ è una base dello spazio $\mathcal{V}^*$ definita come:
\begin{equation}
  \bm{b}^i (\bm{b}_k) = \langle \bm{b}^i , \bm{b}_k \rangle = \delta_k^i
\end{equation}
 dove con $\delta_k^i$ è stata indicata la delta di Kronecker, uguale a $1$ quando gli indici sono
 uguali, uguale a $0$ altrimenti.
 Si può scrivere l'elemento $\bm{v}^* \in \mathcal{V}^*$ in componenti rispetto alla base $\{ \bm{b}^k \}$
\begin{equation}
  \bm{v}^* = v_k \bm{b}^k
\end{equation}
 Le componenti $v_k$ di $\bm{v}^*$ sono definite \textbf{componenti covarianti} del covettore $\bm{v}^*$.
 
 \paragraph{Osservazioni.} 
 La base $\{ \bm{b}_k \}_{k=1:N}$ di $\mathcal{V}$ in generale non è ortogonale. 
  In generale, gli spazi $\mathcal{V}$ e $\mathcal{V}^*$ sono tra di
 loro distinti e quindi gli elementi di questi spazi non sono tra di loro sommabili, 
 \textit{come pere e mele}. 
  Risulta fondamentale prestare attenzione alla posizione degli indici e dei pedici.
  In seguito, dopo aver ristretto la trattazione generale a casi più particolari, si ridurrà l'esigenza di prestare attenzione 
  alla posizione degli indici: un esempio di situazione in cui è possibile ``fare meno attenzione'' alla posizione
  degli indici è costituito dalle \underline{componenti fisiche} di tensori espressi in sistemi di \underline{coordinate 
  curvilinee ortogonali}.
  
  \paragraph{Cambio di base e regole di trasformazione: covarianza e contravarianza.}
  I termini covariante o contravariante sono riferiti alla legge di trasformazione di un ``oggetto'' (componente o elemento di una base),
  se confrontata con la legge di trasformazione degli elementi della base $\{ \bm{b}_k \}_{k=1:N}$ di $\mathcal{V}$.
  Si noti che gli apici sono riservati agli ``oggetti'' contravarianti (le componenti $v^k$ del vettore
  $\bm{v}$ e gli elementi della base duale $\{ \bm{b}^k \}_{k=1:N}$ di $\mathcal{V}^*$), mentre i pedici sono riservati 
  agli elementi covarianti (le componenti $v_k$ del covettore $\bm{v}$ e gli elementi della base $\{ \bm{b}_k \}_{k=1:N}$ di $\mathcal{V}$)
  
  Due basi $\{ \bm{b}_k \}_{k=1:N}$
 e $\{ \bm{\hat{b}}_k \}_{k=1:N}$ di $\mathcal{V}$ sono legate dalla trasformazione
\begin{equation}
  \bm{b}_k = \hat{T}^q_k \bm{\hat{b}}_q \quad , \quad \bm{\hat{b}}_k = {T}^q_k \bm{b}_q
\end{equation}
 dove è stata indicata con $\hat{T}$ la trasformazione inversa di $T$ ($\hat{T} = T^{-1}$).

 \noindent
 Le rispettive basi duali $\{ \bm{b}^k \}_{k=1:N}$ e $\{ \bm{\hat{b}}^k \}_{k=1:N}$ di $\mathcal{V}$ sono legate
 dalla trasformazione inversa\footnote{
 Inserendo la trasformazione riportata, si verifica che
 \begin{equation}
 \bm{\hat{b}}^i (\bm{\hat{b}}_k) = \bm{\hat{b}}^i (T^q_k \bm{b}_q) = T^q_k \bm{\hat{b}}^i (\bm{b}_q) =
  \hat{T}^i_l T^q_k \bm{b}^l (\bm{b}_q) = \hat{T}^i_l T^q_k \delta^l_q =  \hat{T}^i_l T^l_k = \delta^i_k
 \end{equation}
 }, mostrando quindi una natura contravariante alla quale vengono riservati gli apici:
 \begin{equation}
  \bm{b}^k = T^k_q \bm{\hat{b}}^q \quad , \quad \bm{\hat{b}}^k = \hat{T}^k_q \bm{b}^q 
 \end{equation}
 
 \noindent
 Le componenti degli elementi (vettori) $\bm{v}$ dello spazio $\mathcal{V}$ si trasformano secondo la trasformazione inversa degli elementi della base
 $\{ \bm{b}_k \}_{k=1:N}$ di $\mathcal{V}$, mostrando carattere contravariante al quale sono associati gli apici;
 da questo il nome \textbf{componenti contravarianti}.
 \begin{equation}
  \bm{v} = v^q \bm{b}_q = v^q \hat{T}^k_q \bm{\hat{b}}_k = \hat{v}^k \bm{\hat{b}}_k 
  \qquad  \Rightarrow \qquad
  \begin{cases}
     \hat{v}^k = \hat{T}^k_q v^q \\
     v^k       = T^k_q \hat{v}^q \\
  \end{cases}
 \end{equation}
 Allo stesso modo le componenti degli elementi (covettori) $\bm{v^*}$ dello spazio $\mathcal{V^*}$ si trasformano con la stessa trasformazione degli elementi
  della base
 $\{ \bm{b}_k \}_{k=1:N}$ di $\mathcal{V}$, mostrando carattere covariante al quale sono associati i pedici;
  da questo il nome \textbf{componenti covarianti}.
 \begin{equation}
  \bm{v^*} = v^*_q \bm{b}^q = v^*_q T^q_k \bm{\hat{b}}^k = \hat{v}^*_k \bm{\hat{b}}^k 
  \qquad  \Rightarrow \qquad
  \begin{cases}
     \hat{v}^*_k = T^q_k v^*_q \\
     v^*_k       = \hat{T}^q_k \hat{v}^*_q \\
  \end{cases}
 \end{equation}
 

% Definizione di secondo spazio duale V**, isomorfismo canonico tra V e V**: conseguenze pratiche ...
%    ...


% Definizione di tensore come funzione multilineare (p,q)
 \paragraph{Tensori come funzioni multilineari.} Un tensore di ordine $(p,q)$ su $\mathcal{V}$ è una funzione
 (p+q)-lineare
\begin{equation}
   T : \underbrace{\mathcal{V}^* \times \dots \times \mathcal{V}^*}_{\text{p volte}} \times
       \underbrace{\mathcal{V}   \times \dots \times \mathcal{V}  }_{\text{q volte}} \rightarrow K
\end{equation}
 
 Si indica con $\mathcal{T}^p_q(\mathcal{V})$ lo spazio vettoriale dei tensori di ordine $(p,q)$ 
 (con moltiplicazione per uno scalare e somma definiti in seguito). %, operazioni rispetto alle quali uno spazio vettoriale deve essere chiuso) 
 Un tensore di ordine $(0,0)$ è uno scalare, $(1,0)$ un vettore (in questa dispensa ci sarà
 solo un breve cenno all'isomorfismo canonico tra 
 $\mathcal{V}$ e il secondo spazio duale $\mathcal{V}^{**}$, che permette di identificare gli elementi di
 $\mathcal{V}^{**}$ e $\mathcal{V}$), $(0,1)$ un covettore. Inoltre un tensore di ordine
 $(p,0)$ è un tensore puro contravariante, un tensore di ordine $(0,q)$ è un tensore puro
 covariante, un tensore di ordine $(p,q)$ è un tensore misto.
 
 \subsection{Alcune operazioni tensoriali (I)}
 
 \paragraph{Somma e moltiplicazione per uno scalare.} Queste due operazioni e la chiusura dello spazio
 $\mathcal{T}^p_q$ rispetto ad esse\footnote{In parole povere, uno spazio vettoriale V è chiuso rispetto
 a un'operazione se l'operazione su ogni elemento di V restituisce un elemento di V.} sono condizioni
 necessarie alla struttura di spazio vettoriale. La somma di due tensori $\bm{A},\bm{B} \in \mathcal{T}^p_q(\mathcal{V})$
 e la moltiplicazione di $\bm{A}$ per uno scalare $\alpha \in K$ sono definite come
\begin{equation}
 (\bm{A}+\bm{B})(\bm{v}^1,\dots,\bm{v}^p,\bm{v}_1,\dots,\bm{v}_q) = 
   \bm{A} (\bm{v}^1,\dots,\bm{v}^p,\bm{v}_1,\dots,\bm{v}_q) +
   \bm{B} (\bm{v}^1,\dots,\bm{v}^p,\bm{v}_1,\dots,\bm{v}_q)
\end{equation}
 e
\begin{equation}
 ( \alpha \bm{A} ) (\bm{v}^1,\dots,\bm{v}^p,\bm{v}_1,\dots,\bm{v}_q) = 
   \alpha \bm{A}   (\bm{v}^1,\dots,\bm{v}^p,\bm{v}_1,\dots,\bm{v}_q)
\end{equation}
 
 \paragraph{Prodotto tensoriale.} Dati p vettori $\bm{v}_1,\dots,\bm{v}_p \in \mathcal{V}$ e q covettori
 $\bm{v}^1,\dots,\bm{v}^q \in \mathcal{V}^*$ si definisce il prodotto vettoriale
 $\bm{v}_1 \otimes \dots \otimes \bm{v}_p \otimes \bm{v}^1 \otimes \dots \otimes \bm{v}^q \in \mathcal{T}^p_q
 (\mathcal{V})$ come
\begin{equation}
  \bm{v}_1 \otimes \dots \otimes \bm{v}_p \otimes \bm{v}^1 \otimes \dots \otimes \bm{v}^q 
  (\bm{u}^1,\dots,\bm{u}^p,\bm{u}_1,\dots,\bm{u}_q) = 
  \langle \bm{u}^1 ,\bm{v}_1 \rangle \dots \langle \bm{u}^p ,\bm{v}_p \rangle
  \langle \bm{v}^1 ,\bm{u}_1 \rangle \dots \langle \bm{v}^q ,\bm{u}_q \rangle
\end{equation}
 per ogni $\bm{u}^1,\dots,\bm{u}^p \in \mathcal{V}^*$, $\bm{u}_1,\dots,\bm{u}_q \in \mathcal{V}$.
 
 \noindent
 Per due tensori $\bm{A} \in \mathcal{T}^p_q(\mathcal{V})$, $\bm{B} \in \mathcal{T}^r_s(\mathcal{V})$
 il prodotto $\bm{A} \otimes \bm{B} \in \mathcal{T}^{p+r}_{q+s}(\mathcal{V})$ è definito come
\begin{equation}
 (\bm{A}\otimes\bm{B})(\bm{v}^1,\dots,\bm{v}^{p+r},\bm{v}_1,\dots,\bm{v}_{q+s}) = 
   \bm{A} (\bm{v}^1,\dots,\bm{v}^p,\bm{v}_1,\dots,\bm{v}_q)
   \bm{B} (\bm{v}^{p+1},\dots,\bm{v}^{p+r},\bm{v}_{q+1},\dots,\bm{v}_{q+s})
\end{equation}

 \noindent
 Il prodotto tensoriale NON è commutativo! $\bm{A} \otimes \bm{B} \neq \bm{B} \otimes \bm{A}$.

  \subsection{Spazio dei tensori di ordine (p,q) $\mathcal{T}^p_q(\mathcal{V})$}
 
 L'insieme dei tensori di ordine (p,q) costituisce uno spazio vettoriale, indicato con $\mathcal{T}^p_q(\mathcal{V})$,
 una volta definite le operazioni chiuse di somma e di moltiplicazione per uno scalare.
 
 \paragraph{Dimensioni e base prodotto di $\mathcal{T}^p_q(\mathcal{V})$.} Se lo spazio $\mathcal{V}$
 ha dimensione $N$, la dimensione dello spazio $\mathcal{T}^p_q(\mathcal{V})$ è $N^{p+q}$.
 La base $\{ \bm{b}_k \}_{k=1:N}$ di $\mathcal{V}$ e la duale $\{ \bm{b}^k \}_{k=1:N}$ di $\mathcal{V}^*$
 inducono una \textbf{base prodotto} di $\mathcal{T}^p_q(\mathcal{V})$, definita come
\begin{equation}
  \left\{ \bm{b}_{i_1} \otimes \dots \otimes \bm{b}_{i_p} \otimes \bm{b}^{j_1} \otimes \dots \otimes \bm{b}^{j_q}   \right\}_{
  i_1,\dots,i_p,j_1,\dots,j_q = 1 : N}
\end{equation}

\noindent
 Rispetto alla base prodotto un tensore $\bm{A} \in \mathcal{T}^p_q(\mathcal{V})$ viene scritto come
 \begin{equation}
  \bm{A} = A^{i_1 \dots i_p}_{\ \ \ j_1 \dots j_q} \bm{b}_{i_1} \otimes \dots \otimes \bm{b}_{i_p} \otimes
     \bm{b}^{j_1} \otimes \dots \otimes \bm{b}^{j_q}
 \end{equation}
 dove $A^{i_1 \dots i_p}_{\ \ \ j_1 \dots j_q}$ sono le componenti del tensore $\bm{A}$ rispetto alla base prodotto.
 Si dimostra\footnote{Infatti
 \begin{equation}
 \begin{aligned}
   \bm{A}(\bm{b}^{k_1},\dots,\bm{b}^{k_p},\bm{b}_{l_1},\dots,\bm{b}_{l_q}) & =
     A^{i_1 \dots i_p}_{\ \ \ j_1 \dots j_q} \langle \bm{b}^{k_1}, \bm{b}_{i_1} \rangle  \dots \langle \bm{b}^{j_q}, \bm{b}_{l_q} \rangle = \\
     & = A^{i_1 \dots i_p}_{\ \ \ j_1 \dots j_q} \delta^{k_1}_{i_1} \dots \delta^{j_q}_{l_q} = A^{k_1 \dots k_p}_{\ \ \ l_1 \dots l_q}
 \end{aligned}
 \end{equation}
 } che le componenti $A^{i_1 \dots i_p}_{\ \ \ j_1 \dots j_q}$ sono
 \begin{equation}
   A^{i_1 \dots i_p}_{\ \ \ j_1 \dots j_q} = \bm{A}(\bm{b}^{i_1},\dots,\bm{b}^{i_p},\bm{b}_{j_1},\dots,\bm{b}_{j_q})
 \end{equation}
 
%  le componenti di un tensore $\bm{A} \in \mathcal{T}^p_q(\mathcal{V})$ sono $N^(p+q)$ scalari 
% definiti come
%\begin{equation}
%  A^{i_1 \dots i_p}_{\ \ \ j_1 \dots j_q} = \bm{A}(\bm{b}^{i_1},\dots,\bm{b}^{i_p},\bm{b}_{j_1},\dots,\bm{b}_{j_q})
%\end{equation}
% e il tensore $\bm{A}$ può esere scritto in componenti come
%\begin{equation}
%  \bm{A} = A^{i_1 \dots i_p}_{\ \ \ j_1 \dots j_q} \bm{b}_{i_1} \otimes \dots \otimes \bm{b}_{i_p} \otimes
%   \bm{b}^{j_1} \otimes \dots \otimes \bm{b}^{j_q}
%\end{equation}


 \paragraph{Cambio di base e regola di trasformazione delle componenti: definizione ``classica'' di tensore.} 
 
% Due basi $\{ \bm{b}_k \}_{k=1:N}$
% e $\{ \bm{\hat{b}}_k \}_{k=1:N}$ di $\mathcal{V}$ sono legate dalla trasformazione
%\begin{equation}
%  \bm{b}_k = \hat{T}^q_k \bm{\hat{b}}_q \quad , \quad \bm{\hat{b}}_k = {T}^q_k \bm{b}_q
%\end{equation}
% dove è stata indicata con $\hat{T}$ la trasformazione inversa di $T$ ($\hat{T} = T^{-1}$).

% \noindent
% Le rispettive basi duali $\{ \bm{b}^k \}_{k=1:N}$ e $\{ \bm{\hat{b}}^k \}_{k=1:N}$ di $\mathcal{V}$ sono legate
% dalla trasformazione inversa\footnote{
% Inserendo la trasformazione riportata, si verifica che
% \begin{equation}
% \bm{\hat{b}}^i (\bm{\hat{b}}_k) = \bm{\hat{b}}^i (T^q_k \bm{b}_q) = T^q_k \bm{\hat{b}}^i (\bm{b}_q) =
%  \hat{T}^i_l T^q_k \bm{b}^l (\bm{b}_q) = \hat{T}^i_l T^q_k \delta^l_q =  \hat{T}^i_l T^l_k = \delta^i_k
% \end{equation}
% }:
% \begin{equation}
%  \bm{b}^k = T^k_q \bm{\hat{b}}^q \quad , \quad \bm{\hat{b}}^k = {T}^k_q \bm{b}^q 
% \end{equation}
% 
% \noindent
% Le componenti dello spazio $\mathcal{V}$ si trasformano con la trasformazione inversa degli elementi della base
% $\{ \bm{b}_k \}_{k=1:N}$ di $\mathcal{V}$: da questo il nome \textbf{componenti contravarianti}.
% \begin{equation}
%  \bm{v} = v^q \bm{b}_q = v^q \hat{T}^k_q \bm{\hat{b}}_k = \hat{v}^k \bm{\hat{b}}_k 
%  \qquad  \Rightarrow \qquad
%  \begin{cases}
%     \hat{v}^k = \hat{T}^k_q v^q \\
%     v^k       = T^k_q \hat{v}^q \\
%  \end{cases}
% \end{equation}
% Allo stesso modo le componenti dello spazio $\mathcal{V^*}$ si trasformano con la stessa trasformazione degli elementi della base
% $\{ \bm{b}_k \}_{k=1:N}$ di $\mathcal{V}$: da questo il nome \textbf{componenti covarianti}.
% \begin{equation}
%  \bm{v^*} = v^*_q \bm{b}^q = v^*_q T^q_k \bm{\hat{b}}^k = \hat{v}^*_k \bm{\hat{b}}^k 
%  \qquad  \Rightarrow \qquad
%  \begin{cases}
%     \hat{v}^*_k = T^q_k v^*_q \\
%     v^*_k       = \hat{T}^q_k \hat{v}^*_q \\
%  \end{cases}
% \end{equation}
% 
 
 \noindent
 Aiutandosi con la legge di trasformazione degli elementi della base $\{ \bm{b}_k \}_{k=1:N}$ di $\mathcal{V}$ 
 e della base duale $\{ \bm{b}^k \}_{k=1:N}$ di $\mathcal{V}^*$, si può verificare\footnote{
 Per ricavare la regola di trasformazione della base prodotto è sufficiente applicare a tutti gli elementi $\bm{b}_{i_\alpha}$ e 
 $\bm{b}^{j_\alpha}$ le regole di trasformazione viste in precedenza
 \begin{equation}
   \begin{cases}
     \bm{b}_k = \hat{T}^q_k \bm{\hat{b}}_q \\ \bm{\hat{b}}_k = T^q_k \bm{b}_q
   \end{cases} \qquad \qquad
   \begin{cases}
     \bm{b}^k = T^k_q \bm{\hat{b}}^q \\ \bm{\hat{b}}^k = \hat{T}^k_q \bm{b}^q
   \end{cases}
 \end{equation}
 Usando la multilinearità del prodotto vettoriale
 \begin{equation}
 \begin{aligned}
  \bm{\hat{b}}_{i_1} \otimes \dots \otimes \bm{\hat{b}}_{i_p} \otimes
   \bm{\hat{b}}^{j_1} \otimes \dots \otimes \bm{\hat{b}}^{j_q} = &
  ( T^{k_1}_{i_1}\bm{b}_{k_1} ) \otimes \dots \otimes ( T^{k_p}_{i_p}\bm{b}_{k_p}) \otimes
  ( \hat{T}^{j_1}_{l_1} \bm{b}^{l_1} ) \otimes \dots \otimes ( \dots \hat{T}^{j_p}_{l_p} \bm{b}^{l_q} ) = \\
  = & T^{k_1}_{i_1}\dots T^{k_p}_{i_p}\hat{T}^{j_1}_{l_1}\dots \hat{T}^{j_p}_{l_p}
   \bm{b}_{k_1} \otimes \dots \otimes \bm{b}_{k_p} \otimes
   \bm{b}^{l_1} \otimes \dots \otimes \bm{b}^{l_q} \\
 \end{aligned}
 \end{equation}
 }
  che la base prodotto dello spazio $\mathcal{T}^p_q(\mathcal{V})$ si trasforma secondo
\begin{equation}
\begin{aligned}
 &  \bm{\hat{b}}_{i_1} \otimes \dots \otimes \bm{\hat{b}}_{i_p} \otimes
   \bm{\hat{b}}^{j_1} \otimes \dots \otimes \bm{\hat{b}}^{j_q} = 
  T^{k_1}_{i_1}\dots T^{k_p}_{i_p}\hat{T}^{j_1}_{l_1}\dots \hat{T}^{j_p}_{l_p}
  \bm{b}_{k_1} \otimes \dots \otimes \bm{b}_{k_p} \otimes
   \bm{b}^{l_1} \otimes \dots \otimes \bm{b}^{l_q} \\
  &  \bm{b}_{i_1} \otimes \dots \otimes \bm{b}_{i_p} \otimes
   \bm{b}^{j_1} \otimes \dots \otimes \bm{b}^{j_q} = 
  \hat{T}^{k_1}_{i_1}\dots \hat{T}^{k_p}_{i_p} T ^{j_1}_{l_1}\dots T^{j_p}_{l_p}
  \bm{\hat{b}}_{k_1} \otimes \dots \otimes \bm{\hat{b}}_{k_p} \otimes
   \bm{\hat{b}}^{l_1} \otimes \dots \otimes \bm{\hat{b}}^{l_q}
\end{aligned}
\end{equation}
 e le componenti di un tensore $\bm{A} \in \mathcal{T}^p_q(\mathcal{V})$ rispetto alle due basi sono legate da\footnote{
 Le componenti si trasformano con la legge inversa agli elementi della base. Il tensore $\bm{A}$ espresso in componenti nelle due
 basi prodotto è
 \begin{equation}
 \bm{A} = 
 \begin{cases}
   A^{i_1 \dots i_p}_{\ \ \ j_1 \dots j_q} \bm{b}_{i_1} \otimes \dots \otimes \bm{b}_{i_p} \otimes
     \bm{b}^{j_1} \otimes \dots \otimes \bm{b}^{j_q} \\
  \hat{A}^{i_1 \dots i_p}_{\ \ \ j_1 \dots j_q} \bm{\hat{b}}_{i_1} \otimes \dots \otimes \bm{\hat{b}}_{i_p} \otimes
     \bm{\hat{b}}^{j_1} \otimes \dots \otimes \bm{\hat{b}}^{j_q}
 \end{cases}
 \end{equation}
 La legge di trasformazione delle componenti si ricava grazie alla legge di trasformazione della base prodotto
 \begin{equation}
 \begin{aligned}
  \bm{A} & =  A^{i_1 \dots i_p}_{\ \ \ j_1 \dots j_q} \bm{b}_{i_1} \otimes \dots \otimes \bm{b}_{i_p} \otimes
     \bm{b}^{j_1} \otimes \dots \otimes \bm{b}^{j_q} = \\
   & = A^{i_1 \dots i_p}_{\ \ \ j_1 \dots j_q} 
     \hat{T}^{k_1}_{i_1}\dots \hat{T}^{k_p}_{i_p} T ^{j_1}_{l_1}\dots T^{j_p}_{l_p}
     \bm{\hat{b}}_{k_1} \otimes \dots \otimes \bm{\hat{b}}_{k_p} \otimes
     \bm{\hat{b}}^{l_1} \otimes \dots \otimes \bm{\hat{b}}^{l_q}= \\ 
   & = \hat{A}^{k_1 \dots k_p}_{\ \ \ l_1 \dots l_q} \bm{\hat{b}}_{k_1} \otimes \dots \otimes \bm{\hat{b}}_{k_p} \otimes
     \bm{\hat{b}}^{l_1} \otimes \dots \otimes \bm{\hat{b}}^{l_q}
 \end{aligned}
 \end{equation}
 }
\begin{equation}
\begin{aligned}
 &  \hat{A}^{k_1 \dots k_p}_{\ \ \ l_1 \dots l_q} = 
  \hat{T}^{k_1}_{i_1}\dots \hat{T}^{k_p}_{i_p} T^{j_1}_{l_1} \dots T^{j_p}_{l_p}
  A^{i_1 \dots i_p}_{\ \ \ j_1 \dots j_q} \\
 &  A^{i_1 \dots i_p}_{\ \ \ j_1 \dots j_q} = 
  T^{i_1}_{k_1}\dots T^{i_p}_{k_p} \hat{T}^{l_1}_{j_1} \dots \hat{T}^{l_p}_{j_p}
  \hat{A}^{k_1 \dots k_p}_{\ \ \ l_1 \dots l_q} \\
\end{aligned}
\end{equation}
 


 \subsection{Alcune operazioni tensoriali (II)}

 \paragraph{Contrazione.} L'operazione di contrazione $\bm{C}^k_l$ agente su un tensore 
 $\bm{A}$ di ordine $(p,q)$ ($i<p$, $k<q$) ha come risultato un tensore di ordine $(p-1,q-1)$. In componenti si ottiene
 \begin{equation}
 \begin{aligned}
  \bm{C}^k_{l}\bm{A} & = \bm{C}^k_{l} A^{i_1 \dots i_p}_{\ \ \ j_1 \dots j_q} \bm{b}_{i_1} \otimes \dots \otimes \bm{b}_{i_p} \otimes
   \bm{b}^{j_1} \otimes \dots \otimes \bm{b}^{j_q} = \\
   & = A^{i_1 \dots i_{k-1} t i_{k+1} \dots i_p}_{\ \ \ \ \ j_1 \dots j_{l-1} t j_{l+1} \dots j_q} 
   \bm{b}_{i_1} \otimes \dots \otimes \bm{b}_{i_{k-1}} \otimes \bm{b}_{i_{k+1}} \otimes \dots \otimes \bm{b}_{i_p} \otimes
   \bm{b}^{j_1} \otimes \dots \otimes \bm{b}^{j_{l-1}} \otimes \bm{b}^{j_{l+1}} \otimes \dots \otimes \bm{b}^{j_q}
 \end{aligned}
 \end{equation}
 Affinché il risultato della contrazione di due indici di un tensore dia come risultato un altro tensore,
 è necessario che la contrazione avvenga tra un indice contravariante e uno covariante.
 
 \paragraph{``Dot'' product.} Siano $\bm{A} \in \mathcal{T}^p_q(\mathcal{V})$, $\bm{B} \in \mathcal{T}^r_s(\mathcal{V})$, il
 prodotto ``dot'' $\bm{A} \cdot \bm{B}$ è un tensore di ordine $(p+r-1,q+s-1)$, definito tramite il prodotto tensoriale e una contrazione.
 In particolare si definisce
 \begin{equation}
  \bm{A} \cdot \bm{B} = \bm{C}^{p+1}_{q} (\bm{A} \otimes \bm{B})
 \end{equation}
 
 \noindent
 Il prodotto ``dot'' NON è commutativo ($\bm{A} \cdot \bm{B} \neq \bm{B} \cdot \bm{A}$)
 Il prodotto ``dot'' NON è un prodotto interno (in generale non restituisce uno scalare).
 
 \paragraph{Doppio ``Dot'' product.}
 Siano $\bm{A} \in \mathcal{T}^p_q(\mathcal{V})$, $\bm{B} \in \mathcal{T}^r_s(\mathcal{V})$, il
 prodotto ``dot'' $\bm{A} : \bm{B}$ è un tensore di ordine $(p+r-2,q+s-2)$ definito tramite il prodotto tensoriale e una doppia contrazione.
  In particolare si definisce
 \begin{equation}
  \bm{A} : \bm{B} = \bm{C}^{p+1,p+2}_{q-1,q} (\bm{A} \otimes \bm{B})
 \end{equation}
 
 
% \paragraph{Invarianti.} \dots
 
 


% Tensori in uno spazio con prodotto interno
 \subsection{Tensori su uno spazio $\mathcal{V}$ dotato di prodotto interno}
 Per uno spazio vettoriale $\mathcal{V}$ dotato di prodotto interno, una volta scelto e fissato il prodotto
 interno, esiste un \textbf{isomorfismo} tra gli spazi $\mathcal{V}$ e $\mathcal{V}^*$ che permette di identificare
 gli elementi di $\mathcal{V}$ con quelli di $\mathcal{V}^*$.
 Viene definito l'isomorfismo $\bm{G}: \mathcal{V} \rightarrow \mathcal{V}^*$ tale che
 \begin{equation}
   \langle \bm{G}\bm{v}, \bm{w} \rangle = \bm{v} \cdot \bm{w} \qquad \forall \bm{v},\bm{w} \in \mathcal{V}
 \end{equation}
 
 Solo a questo punto, e prestando ancora comunque molta attenzione alla posizione degli indici, è lecito 
 considerare gli elementi di $\mathcal{V}^*$ come elementi di $\mathcal{V}$. Ad esempio, la base duale
 di $\mathcal{V}^*$ corrisponde alla \textbf{base reciproca} $\{ \bm{b}^k \}_{k=1:N}$ di $\mathcal{V}$
\begin{equation}
  \bm{b}^i \cdot \bm{b}_k  = \delta_k^i
\end{equation}
 

 \paragraph{Osservazioni.} Come già osservato, la base $\{ \bm{b}_k \}$ in generale non è
 ortogonale, né tantomeno ortonormale. Questo in generale complica la scrittura in componenti di un vettore.
 Come vedremo in seguito, ci viene in aiuto la base reciproca.
 
 Anche se è possibile identificare $\mathcal{V}^*$ con $\mathcal{V}$, è fondamentale non confondere
 le componenti contravarianti (indice come apice) e covarianti (pedice) di vettori e tensori.
 In un'equazione vettoriale o tensoriale (come le leggi fisiche) scritte in componenti, è possibile sommare 
 senza ulteriori trasformazioni termini che hanno gli stessi indici contravarianti e covarianti. 
% Per essere chiari,
% la somma di due tensori $\bm{A} + \bm{B} = \bm{C}$ può essere scritta in com
% \begin{equation}
%&   A^{ij}_{k} + B^{ij}_k  = C^{ij}_k \\
% \end{equation}
% mentre $A^{ij}_k + B^{i}_{jk} = C^{ij}_k$.
 
 \paragraph{Cosa non è stato detto. Riferimenti.} Viene solo accennata l'esistenza di un 
 \textit{secondo spazio duale} $\mathcal{V}^{**}$, come il duale del duale.
 Si dimostra che esiste un \textit{isomorfismo} (canonico) tra lo spazio $\mathcal{V}$ e $\mathcal{V}^{**}$,
 cioè che è possibile identificare univocamente ogni elemento di $\mathcal{V}^{**}$ con
 un elemento di $\mathcal{V}$ e viceversa ed è quindi possibile confondere questi due spazi.
 Il testo di Bowen e Wang, \textit{Introduction to vectors and tensors. Linear and multilinear
 algebra} può essere considerato un valido e completo riferimento, anche per il futuro.
 Questo testo è estremamente completo, di lettura non sempre agevole e contiene sicuramente molto più
 di quanto sia indispensabile presentare in una prima e breve introduzione ai tensori, come è questa.
 Il testo può essere comunque un buon riferimento per il futuro, qualora aveste questa necessità.
 Se i collegamenti sono ancora validi, i due volumi sono disponibili in rete. 
 
 \href{http://oaktrust.library.tamu.edu/bitstream/handle/1969.1/2502/IntroductionToVectorsAndTensorsVol1.pdf}
 {Vol. 1: Linear and Multilinear Algebra}
 
 \href{http://oaktrust.library.tamu.edu/bitstream/handle/1969.1/3609/IntroductionToVectorsAndTensorsVol2.pdf}
 {Vol. 2: Vector and Tensor Analysis}
 
% http://www.mat.unimi.it/users/carati/didattica/dispense/tensori.pdf
 
 
 \paragraph{Cosa è utile ripassare.} Forse questa è una buona occasione per ripassare
 alcuni concetti, tra i quali quello di spazio vettoriale (definizione e proprietà, dimensione e base, \dots),
 prodotto interno, linearità (e la differenza con l'essere ``affine''), e in generale rispoleverare e approfondire
 la propria conoscenza di algebra lineare (o almeno intuirne l'importanza e l'alta probabilità di 
 imbattercisi nel futuro).
% Può essere educativo iniziare ad avere una visione ``più astratta'' di questi oggetti, legata
% maggiormente alle definizioni rispetto a una rappresentazione più concreta (un vettore non è solo
% una ``freccia'', ma si scopre che anche una funzione può essere un elemento di un particolare spazio vettoriale \dots)
 
% \subsubsection{Definizione come applicazione multilineare.}
% 
% \paragraph{Definizione classica.}
% 

% \paragraph{Coordinate rispetto a una base}
% Base e base duale...
% \paragraph{Covarianza e controvarianza}
% \paragraph{Cambiamento di base}

% \paragraph{Equivalenza delle due definizioni.}
% 


%\subsection{Operazioni elementari}
% \paragraph{Somma e moltiplicazione per uno scalare}
%  Struttura di spazio vettoriale \dots
% \paragraph{Composizione}
% 
% \paragraph{Trasposto}
% 
% \paragraph{Contrazione}
% 
% \paragraph{Invarianti}
%  Determinante, traccia, \dots

%\subsection{Campi tensoriali e coordinate curvilinee}
% \paragraph{Metrica.}
%\begin{itemize}
%  \item Tensore metrico e $A_i = g_{ij} A^j$.
%  \item $g_{ij}$ in sistemi di riferimento diversi.
%  \item $ds^2 = g_{ij} dx^i dx^j$
%\end{itemize}

% \paragraph{Base naturale}
%  Vettori tangenti alle linee coordinate.
%  
%  \noindent
%  Componenti:
%\begin{itemize}
%  \item Componenti controvarianti
%  \item Componenti covarianti
%  \item Componenti fisiche
%\end{itemize}
% 

 



