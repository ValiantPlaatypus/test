\begin{exerciseS}[Linee di corrente, traiettorie, linee di fumo: non stazionario]
 Sia dato il campo di moto
\begin{equation}
 \bm{u}(x,y,z) = 3y \bm{\hat{x}} - 3x \bm{\hat{y}} +t\bm{\hat{z}}
\end{equation}
Calcolare l'equazione delle linee di corrente, delle traiettorie e delle linee di fumo (curve di emissione) e disegnarle.
\end{exerciseS}


\sol

\partone Definizione di linee di corrente, traiettorie, linee di fumo, tracce. Soluzione di sistemi di equazioni differenziali.

%\begin{itemize}

%\item
%Le linee di corrente sono curve $\bm{S}$ tangenti al campo vettoriale $\bm{u}(\bm{r},t)$ in ogni punto dello spazio $\bm{r}$ e per ogni istante temporale $t$. Essendo curve (1 dimensione), possono essere espresse in forma parametrica, come funzioni di un parametro scalare $p$. La 'traduzione' della definizione in formula è quindi:
%\begin{equation}
% \frac{d\bm{S}(p)}{dp} = \bm{u}(\bm{S}(p),t)
%\end{equation}
%Il vettore tangente ${d\bm{S}(p)}{dp}$ alla curva $\bm{S}(p)$ nel punto ${\bm{S}(p)}$ è parallelo al vettore 
%velocità $\bm{u}$ nello stesso punto $\bm{S(p)}$, al tempo considerato $t$.

%\item
%Le traiettorie descrivono il moto della singola particella fluida e sono descritte dall'equazione:
%\begin{equation}
%\begin{cases}
% \frac{d\bm{R}(t)}{dt} = \bm{u}(\bm{R(t)},t) \\
% \bm{R}(t_0) = \bm{R_0}
%\end{cases}
%\end{equation}
%La traiettoria descritta sopra è quella della particella che all'istante $t_0$ passa per il punto $\bm{R_0}$.
%Interpretazione della formula: la velocità ${d\bm{R}(t)}/{dt}$ della particella (derivata della posizione della
%particella $R(t)$ nel tempo) è uguale alla velocità del fluido nella posizione $R(t)$ nella quale si trova la particella all'istante $t$.

%Fissati $t_0$ e $\bm{R_0}$, si osserva la traiettoria della particella al variare del tempo $t$.

%\item
%Le linee di fumo sono un modo per tracciare tutte le particelle di fluido passate per un determinato punto nello spazio a diversi istanti temporali. La loro equazione è:
%\begin{equation}
%\begin{cases}
% \frac{d\bm{R}(t)}{dt} = \bm{u}(\bm{R},t) \\
% \bm{R}(\tau) = \bm{\bar{R}}
%\end{cases}
%\end{equation}
%L'equazione è identica all'equazione delle traiettorie.
%Cambia la variabile che descrive la curva: si considerano fissi il punto di emissione $\bm{\bar{R}}$ e il tempo
%$t$ al quale viene osservata la curva di emissione; la variabile che descrive la curva di emissione è il tempo
%$\tau$ al quale le particelle passano da $\bm{\bar{R}}$.

%Nel caso di campi stazionari linee di corrente, traiettorie e linee di fumo coincidono.

%\item
%Tracce:
%\begin{equation}
%\begin{cases}
% \frac{d\bm{R}(t)}{dt} = \bm{u}(\bm{R},t) \\
% \bm{R}(\tau) = \bm{\bar{R}}
%\end{cases}
%\end{equation}
%L'equazione è identica all'equazione delle traiettorie e delle curve di emissione.
%Cambia la variabile che descrive la curva: si considerano fissi il tempo $\tau$ e il tempo
%$t$ al quale viene osservata la curva di emissione; la variabile che descrive la curva di emissione è la posizione $\bm{\bar{R}}$ dalle quali passano le particelle.

%\end{itemize}

%\textit{Osservazione}. Non c'è nessuna differenza formale tra $\tau$ e $t_0$ e $\bm{R_0}$ e $\bm{\bar{R}}$.

%\clearpage

\parttwo
\begin{itemize}
\item Linee di corrente.
\begin{equation}
 \begin{cases}
  \frac{dX}{dp} = \lambda(p) 3Y \\
  \frac{dY}{dp} = -\lambda(p) 3X \\
  \frac{dZ}{dP} = \lambda(p) t
 \end{cases}
 \quad\Rightarrow\quad
 ...
 \quad\Rightarrow\quad
 \begin{cases}
  X(p) = R \cos(p) \\
  Y(p) = R \sin(p) \\
  Z(p) = Z_0 + t p
 \end{cases} \text{se $\lambda(p) = \frac{1}{3}$}
\end{equation}

\item Traiettorie.
\begin{equation}
 \begin{cases}
  \frac{dx}{dt} = 3y \\
  \frac{dy}{dt} = -3x \\
  \frac{dz}{dt} = t \\
  x(0) = x_0 , \quad y(0) = y_0 , \quad z(0) = z_0
 \end{cases}
 \quad\Rightarrow\quad
 \begin{cases}
  \frac{dx}{dt} = 3y \\
  \frac{d^2y}{dt^2} + 9 y = 0 \\
  z(t) = \frac{t^2}{2} + C
 \end{cases}
 \quad\Rightarrow\quad
 \begin{cases}
  x(t) = A \cos(3t) + B \sin(3t) \\
  y(t) = - A \sin(3t) + B \cos(3t) \\
  z(t) = \frac{t^2}{2} + C
 \end{cases}
\end{equation}
Inserendo le condizioni iniziali si ottiene
\begin{equation}
 \begin{cases}
  x(t) = x_0 \cos(3t) + y_0 \sin(3t) \\
  y(t) = - x_0 \sin(3t) + y_0 \cos(3t) \\
  z(t) = \frac{t^2}{2} + z_0
 \end{cases}
\end{equation}

\textit{Osservazione.} Ottenere $y$ in funzione di $x$ e confrontare con le linee di corrente. Commentare...

\item Linee di fumo.
Inserendo nelle soluzioni generali trovate per le traiettoire, le condizioni 
\begin{equation}
 \begin{cases}
  x(\tau) = x_0 \\ y(\tau) = y_0 \\ z(\tau) = z_0
 \end{cases}
\end{equation}
si ottengono le linee di fumo
\begin{equation}
\begin{cases}
  x(t) = x_0 \cos(3(t-\tau)) + y_0 \sin(3(t-\tau)) \\
  y(t) = - x_0 \sin(3(t-\tau)) + y_0 \cos(3(t-\tau)) \\
  z(t) = \frac{t^2-\tau^2}{2} + z_0
\end{cases}
\end{equation}



\end{itemize}
