\subsection{Attività sperimentali: difetto di scia e volume di controllo.}

L'esercizio svolto in precedenza risulta propedeutico ad alcune attività sperimentali
 che svolgerete durante il corso di studi o in attività sperimentali future.
Le attività svolte nel mondo reale sono affette da imprecisioni e incertezze; i 
 \textit{datasheet} che accompagnano uno strumento raccolgono anche le informazioni
 sulla sua incertezza di misura, spesso in forma di intervallo di confidenza di 
 probabilità nota o di scarto quadratico medio.
Le grandezze misurate permettono poi di calcolare delle grandezze derivate. L'incertezza
 di misura dei singoli strumenti si propaga sulla grandezza derivata: nell'ipotesi che
 le incertezze siano tra di loro indipendenti e non correlate, è possibile utilizzare
 la formula RSS (\textit{root-sum-squares}). Se la grandezza derivata $f$ è funzione
 delle $N$ grandezze misurate $x_i$, sulle quali si ha incertezza $\sigma_{x_i}$, è possibile
 stimare l'incertezza su $f$
\begin{equation}
  \sigma_f^2 = \sum_{i=1}^{N} \left( \dfrac{\partial f}{\partial x_i} \right)^2 \sigma_{x_i}^2
\end{equation}

L'incertezza $\sigma_f$ è parte integrante del risultato di una attività sperimentale
 e un indicatore della bontà dell'apparato sperimentale pensato per tale attività.
 In generale, l'incertezza sulla grandezza desiderata deve essere ``molto minore'' della
 grandezza stessa: in caso contrario, l'apparato sperimentale risulterebbe indatto 
 all'attività da svolgere.
Essendo parte integrante del risultato, è buona regola indicare l'incertezza sulla grandezza
 misurata, ad esempio indicandone il valore numerico, il valore relativo alla misura,
 gli intervalli di confidenza sui grafici\dots

\paragraph{Difetto di scia e resistenza del profilo.}
\'E possibile stimare la resistenza di un profilo a partire dalla misura del difetto di scia.
 La misura di velocità in scia viene effettuata (con una sonda di Pitot o altro \dots)
 in punti discreti, più radi lontano dalla scia del profilo,
 fiù fitti in corrispondenza della scia dove i gradienti sono maggiori.
L'integrale nella formula (\ref{eqn:difetto_scia}) viene approssimato con un metodo numerico,
 come ad esempio la formula del punto medio, quella del trapezio, \dots
Utilizzando qui per semplicità la formula del punto medio, è possibile scrivere
\begin{equation}
 F_x = \int_{0}^{H} \rho u(y) (U_\infty - u(y)) dy
    \approx \sum_{k=1}^{N} \rho \left[ u_k (U_\infty - u_k) \right] \Delta s_k
\end{equation}
Le misure ottenute dalla sonda sono affette da incertezza $\sigma_{u_i}$; la derivata $\partial F_x / \partial u_i$ è
\begin{equation}
 \dfrac{\partial F_x}{\partial u_m}  = \rho \left( U_\infty - 2 u_m \right) \Delta s_m
\end{equation}
La stima dell'incertezza sulla resistenza è quindi
\begin{equation}
 \sigma_{F_x}^2 = \sum_{i=1}^{N} \left[ \rho \left( U_\infty - 2 u_i \right) \Delta s_i \right]^2 \sigma_{u_i}^2
\end{equation}


\paragraph{Volume di controllo.}
Esistono metodi sperimentali (es. PIV) che permettono di ottenere il campo di velocità in un determinato istante
 su una griglia di punti.
Si pensi al campo di moto attorno a un'ala allungata, a bassi angoli di incidenza.
 Se il campo di moto è bidimensionale (in buona approssimazione) è possibile 
 utilizzare tecniche sperimentali bidimensionali (PIV-2D) per misurare il campo di velocità su un piano:
 è possibile ottenere una stima delle azioni (per unità di apertura) che esercita il fluido sul profilo
 di ala tagliato dal piano di misura, tramite l'equazione di bilancio di quantità di moto in (\ref{eqn:airfoil_bil_int}).

\vspace{0.2cm}
\textit{(In questa formula compare solo il termine di flusso di quantità di moto, mentre non è presente
 il termine di sforzi di superficie $\oint_S \bm{t}_n$, che include il contributo della pressione, \textbf{non}
 sempre trascurabile.)}
\vspace{0.2cm}

Seguendo il procedimento svolto nel pragrafo precedente applicato all'integrale di superficie, è possibile
 stimare l'incertezza sulla resistenza e sulla portanza ottenute tramite questo metodo.
Si scopre che l'incertezza sulla misura dipende dalla risoluzione della griglia e dalle condizioni
 di prova. Come indicazione generale, l'incertezza su portanza e resistenza sono dello stesso
 ordine di grandezza, e possono raggiungere fino al $30\%$ della misura della resistenza.
In molte applicazioni la portanza è maggiore della resistenza: in una prova ad efficienza del 
 profilo $E=10$, si ottiene un $3\%$ di incertezza sulla portanza.
In questo caso, questo metodo risulta accettabile per una stima della portanza, non per la resistenza.





