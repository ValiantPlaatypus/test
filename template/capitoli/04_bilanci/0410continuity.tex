%
L'equazione di continuità può essere riscritta mettendo in evidenza la derivata materiale
\begin{equation}
 \frac{\partial \rho}{\partial t} + \bm{\nabla} \cdot (\rho \bm{u}) = 0 
  \quad  \rightarrow  \quad 
  \frac{D\rho}{Dt} = -\rho \bm{\nabla} \cdot \bm{u}
\end{equation}
%
\'E possibile dimostrare\footnote{I più curiosi, cerchino ``fornmula di Jacobi''.} la relazione $DJ/Dt = J \bm{\nabla} \cdot \bm{u}$, dove
 $J$ indica il determinante del gradiente $\partial \bm{x}/\partial 
 \bm{x}_0$, si può scrivere l'equazione in coordinate lagrangiane,
 dopo averla moltiplicata per $J$ ($\ne 0$)
\begin{equation}
 J \frac{D\rho}{Dt} = - \rho \frac{DJ}{Dt} \Rightarrow
 \frac{D (J\rho)}{Dt} = 0 \Rightarrow J \rho = \rho_0
\end{equation}
%
La variazione della densità di una particella
 materiale è legata alla variazione del volume della stessa (ricordare
 che $dv = J dV$). Questa conclusione è ragionevole se si pensa che
 la massa della particella materiale si conserva ($dm = \rho dv = 
 \rho_0 dV$).
%
\begin{remark}
Il vincolo di incomprimibilità rappresenta la costanza del volume della 
 particella materiale. Il volume $dv$ coincide con il volume di riferimento $dV$, implicando $J \equiv 1$ e quindi  $\bm{\nabla} \cdot \bm{u} = 0$.
\end{remark}

