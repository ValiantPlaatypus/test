
L'equazione della vorticità in coordinate euleriane è
\begin{equation}
 \frac{\partial \bm{\omega}}{\partial t}
   + (\bm{u}\cdot\bm{\nabla}) \bm{\omega} =
 (\bm{\omega}\cdot\bm{\nabla}) \bm{u} + \nu \bm{\Delta} \bm{\omega}
\end{equation}

Se viene fatta l'ipotesi di viscosità nulla, il termine contenente il 
 laplaciano della vorticità non compare nell'equazione: questo termine
 è il responsabile della diffusione (isotropa per come è scritto) della
 vorticità.
 
L'equazione può essere quindi riscritta come:
 \begin{equation}
  \frac{D\bm{\omega}}{Dt} = (\bm{\omega} \cdot \bm{\nabla}) \bm{u}
 \end{equation}

\noindent 
Scritta in componenti
\begin{equation}
\begin{aligned}
  \frac{D \omega_i}{D t} = \omega_k \frac{\partial u_i}{\partial x_k} \\
\end{aligned}
\end{equation}

\noindent
Il termine di destra può essere riscritto come
\begin{equation}
\begin{aligned}
 \omega_k \frac{\partial u_i}{\partial x_k} & = 
 \omega_k \frac{\partial u_i}{\partial x_{0 l}}
      \frac{\partial x_{0 l}}{\partial x_k} = \qquad \qquad
      \left(u_i = \frac{D x_i}{D t}\right)  \\
 & = \omega_k \frac{D}{Dt} \left( \frac{\partial x_i}{\partial x_{0 l}}
    \right)\frac{\partial x_{0 l}}{\partial x_k} 
\end{aligned}
\end{equation}
Vale la relazione
\begin{equation}
 \frac{\partial x_i}{\partial x_{0 l}}
   \frac{\partial x_{0 l}}{\partial x_k} = \delta_{ik}
\end{equation}
Il termine di sinistra può essere riscritto come
\begin{equation}
 \frac{D \omega_i}{Dt} = \frac{D}{Dt} \left(\delta_{ik} \omega_k \right) =
 \frac{D}{Dt} \left( \frac{\partial x_i}{\partial x_{0 l}}
   \frac{\partial x_{0 l}}{\partial x_k} \omega_k \right)
\end{equation}

Inserendo nell'equazione della vorticità e sfruttando le proprietà
 della derivata del prodotto:
\begin{equation}
\begin{aligned}
 &  \frac{D}{Dt} \left(  \frac{\partial x_i}{\partial x_{0 l}}
   \frac{\partial x_{0 l}}{\partial x_k} \omega_k  \right)  - 
  \omega_k \frac{D}{Dt} \left( \frac{\partial x_i}{\partial x_{0 l}}
    \right)\frac{\partial x_{0 l}}{\partial x_k} = 0 \\
 &  \frac{\partial x_i}{\partial x_{0 l}} 
  \frac{D}{Dt} \left( \frac{\partial x_{0 l}}{\partial x_k} \omega_k
    \right) = 0
\end{aligned}
\end{equation}
Se la trasformazione non è singolare, risulta quindi
\begin{equation}
  \frac{D}{Dt} \left( \frac{\partial x_{0 l}}{\partial x_k} \omega_k
    \right) = 0  \quad \Rightarrow \quad
  \frac{\partial x_{0 l}}{\partial x_k} \omega_k = \omega_{l 0}
\end{equation}
e in conclusione, invertendo il gradiente della trasformazione delle
 coordinate
\begin{equation}\label{eqn:bilanci:vorticitàLagrange}
   \omega_k = \frac{\partial x_k}{\partial x_{0 l}}\omega_{l 0}
   \qquad , \qquad \bm{\omega} = \frac{\partial \bm{x}}{\partial \bm{x}_0}
      \bm{\omega}_0
\end{equation}

Si può quindi notare che la vorticità segue la stessa evoluzione
 di un segmento infinitesimo materiale, per il quale vale:
\begin{equation}
  d\bm{x} = \frac{\partial \bm{x}}{\partial \bm{x}_0}
      d\bm{x}_0
\end{equation}









