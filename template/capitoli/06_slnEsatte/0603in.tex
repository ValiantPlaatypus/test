\noindent
\begin{tabular}{cc}
\begin{minipage}[b]{0.95\textwidth}
\begin{exerciseS}[Corrente libera su parete verticale]
Si consideri una corrente d'acqua a pelo libero, laminare e stazionaria, che
scorre su una parete verticale piana di lunghezza e apertura infinita.
Si ipotizzi che la pressione atmosferica che agisce sul pelo libero sia
uniforme. Si ipotizzi inoltre che lo sforzo tangenziale fra acqua e aria in
corrispondenza del pelo libero sia nullo.

Assegnata la portata in massa per unit\`a di apertura 
$\overline{Q}=0.5\ kg/(ms)$, determinare
\begin{enumerate}
  \item lo spessore $h$ della corrente d'acqua;
  \item lo sforzo tangenziale a parete;
  \item la velocit\`a in corrispondenza del pelo libero;
  \item la velocit\`a media e il numero di Reynolds basato su tale velocit\`a
        media e sullo spessore della corrente.
\end{enumerate}
Si sostituisca poi al pelo libero una parete solida.
Si determini quale dovrebbe essere la velocit\`a di tale parete per ottenere
una portata nulla.

Dati: $\overline{\rho}= 999\ kg/m^3$, 
$\overline{\mu}= 1.15\ 10^{-3} kg/(ms)$.

($h=5.61\, 10^{-4}\  m$, $\tau = 5.494\ Pa$, $u(h)=1.339\ m/s$,
$\overline{U}=0.893\  m/s$, $ Re=434.8$, $U=-0.4464\ m/s$.)
\end{exerciseS}
\end{minipage}
\end{tabular}

\sol

\partone
 Semplificazione delle equazioni di NS in casi particolari. 
Soluzioni esatte in coordinate cartesiane.

\parttwo
Si scelga un sistema di riferimento cartesiano con l'asse x orientato lungo la parete verso il basso e l'asse y perpendicolare ed uscente ad essa.

\noindent
Sulla corrente di questo problema agisce la forza di volume dovuta alla gravità.  

L'ipotesi che la pressione sia uniforme sulla superficie di interfaccia
 tra acqua e aria implica che la pressione è costante in tutto il fluido:
 si vedrà che $\frac{\partial p}{\partial y}=0$; se sulla superficie libera
 la pressione è costante e non varia nello spessore, allora la pressione
 è costante in tutto il fluido.

\begin{itemize}

  \item Scrittura delle equazioni di NS in 2 dimensioni.
  
 \begin{equation}
\begin{cases}
  \frac{\partial u}{\partial t} + u \frac{\partial u}{\partial x}
  + v \frac{\partial u}{\partial y} - \nu \left( 
  \frac{\partial^2 u}{\partial x^2} +
  \frac{\partial^2 u}{\partial y^2} \right)
   + \frac{1}{\rho} \frac{\partial p}{\partial x} = f_x \\
  \frac{\partial v}{\partial t} + u \frac{\partial v}{\partial x}
  + v \frac{\partial v}{\partial y} - \nu \left( 
  \frac{\partial^2 v}{\partial x^2} +
  \frac{\partial^2 v}{\partial y^2} \right)
  + \frac{1}{\rho}  \frac{\partial p}{\partial y} = f_y \\
  \frac{\partial u}{\partial x} + \frac{\partial v}{\partial y} = 0
\end{cases}
\end{equation}

  \item Semplificazione delle equazioni di NS per il problema da affrontare.
  
Ipotesi: 
\begin{itemize}
\item problema stazionario: $\frac{\partial}{\partial t} = 0$;
\item direzione x omogenea (canale infinito in direzione x): $\frac{\partial u}{\partial x} = \frac{\partial v}{\partial x} = 0$; 
la pressione nelle equazioni di NS incomprimibili è un moltiplicatore di Lagrange per imporre il vincolo di incomprimibilità; inoltre non appare mai, se non nelle condizioni al contorno, come $p$ ma solo con le sue derivate spaziali: quindi non è corretto imporre $= \frac{\partial v}{\partial x} = 0$, nonostante la direzione $x$ sia omogenea;
\item $\frac{\partial u}{\partial x} = 0$ inserito nel vincolo di incomprimibilità ($\frac{\partial u}{\partial x}+\frac{\partial v}{\partial y}=0$) implica $\frac{\partial v}{\partial y}=0$; poichè $\frac{\partial v}{\partial x}=\frac{\partial v}{\partial y}=0$ e $v = 0$ a parete per la condizione al contorno di adesione, segue che $v = \text{cost} = 0$;
\item forze di volume solo in direzione verticale: per come sono stati orientati gli assi, $\bm{f} = g \hat{\bm{x}}$.
\end{itemize}
  
\begin{equation}
\begin{cases}
  - \mu \frac{\partial^2 u}{\partial y^2} = - \frac{\partial p}{\partial x} + \rho g\\
  \frac{\partial p}{\partial y} = 0  
\end{cases}
\end{equation}

Dalla seconda segue che la pressione può essere funzione solo di $x$. Come già detto in precedenza, la pressione sulla superficie libera è costante e uguale alla pressione ambiente $P_a$: se la pressione non può variare nello spessore, allora è costante ovunque. La derivata parziale $\frac{\partial p}
 {\partial x}=0$, il suo gradiente è nullo e quindi la pressione è costante
 in tutta la corrente di acqua.

 Nella prima, il termine a sinistra dell'uguale è funzione solo di $y$; quello di destra è costante e uguale a $\rho g$. Le condizioni al contorno sono
 di adesione a parete e di sforzo di taglio nullo all'interfaccia tra aria
 ed acqua: $0=\tau(H)=\mu \frac{\partial u}{\partial y}(H)=\mu u'(H)$, dove 
 la derivata parziale in $y$ è stata sostituita da quella ordinaria, poichè
 la velocità è solo funzione di $y$.
%  \begin{equation}
%  \end{equation}
  
  \begin{equation}
  \begin{cases}
    - \mu u''(y) = \rho g \ , \ y \in[0,H] \\
    u(0) = 0  \\ u'(H) = 0
  \end{cases}
  \end{equation}
  
  
  
  \item Soluzione dell'equazione differenziale (semplice) con dati al contorno.
  
  Risulta:
  \begin{equation}
    \Rightarrow u(y) = - \frac{\rho g}{2 \mu} y^2 + \frac{\rho g}{\mu} H y
  \end{equation}
  
  \item Calcolo della portata come integrale della velocità; si trova così la relazione tra Q ed H.
  
  \begin{equation}
    Q = \int_{0}^{H} \rho u(y) dy = \frac{1}{3}\frac{\rho^2 g}{\mu} H^3 
  \end{equation}
  E quindi
  \begin{equation}
     H = \left( \frac{3 Q \mu}{\rho^2 g} \right) ^ {\frac{1}{3}}
     \quad \Rightarrow \quad H = 5.61 \cdot 10^{-4} m
  \end{equation}
  
  \item Calcolo dello sforzo a parete
  \begin{equation}
    \tau = \mu u'|_{y=0} = \rho g H \quad \Rightarrow \quad \tau = 5.494 Pa
  \end{equation}
  \textit{Osservazione.} Equilibrio con la forza di gravità (problema stazionario).
  
  \item Calcolo di $u(H)$.
  \begin{equation}
    u(H) = \frac{1}{2}\frac{\rho g}{\mu} H^2 
    \quad \Rightarrow \quad u(H) = 1.342 m/s
  \end{equation}
  
  \item Calcolo velocità media e numero di Reynolds.
  \begin{equation}
    \bar{U} = \frac{1}{H}\int_{0}^{H} u(y) dy = \frac{Q}{\rho H}
    \quad \Rightarrow \quad \bar{U} = \frac{Q}{\rho H}
                                    = \frac{2}{3}u(H) = 0.895 m/s
  \end{equation}
  
  \begin{equation}
    Re = \frac{\rho \bar{U} H}{\mu}
    \quad \Rightarrow \quad Re = 434.8
  \end{equation}
 
  
\end{itemize}

L'ultima parte del problema chiede di sostiutire alla superficie libera,
 una parete infinita. L'equazione trovata in precedenza è ancora valida;
 è necessario però sostituire la condizione di sforzo tangenziale nullo
 con adesione su una parete mobile con velocità costante $U$.

  \begin{equation}
  \begin{cases}
    - \mu u''(y) = \rho g \ , \ y \in[0,H] \\
    u(0) = 0  \\ u(H) = U
  \end{cases}
  \end{equation}

Il profilo di velocità è:
\begin{equation}
 u(y) = \dfrac{\rho g}{2 \mu}(-y^2 + yH) +\dfrac{U}{H}y
\end{equation}
dove la velocità $U$ è ancora incognita. Per trovarne il valore, si 
calcola la portata e la si pone uguale a zero. La portata è uguale a
\begin{equation}
 Q = \int_0^H u(y) dy = \dots = \dfrac{1}{12}\dfrac{\rho g H^3}{\mu}
 + \dfrac{1}{2}UH
\end{equation}
Imponendo $Q=0$,
\begin{equation}
 U = - \dfrac{\rho g H^2}{6 \mu} \quad \Rightarrow \quad
 U = - 0.4474\ m/s
\end{equation}

