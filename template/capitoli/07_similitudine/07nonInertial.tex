
\subsubsection{Equazioni di Navier--Stokes}
In un sistema di riferimento inerziale\footnote{
  Per noi, un \textbf{sistema di riferimento inerziale} è un sistema di riferimento rispetto al quale le equazioni della fisica assumono la loro forma più semplice, senza che compaiano termini di trascinamento, come ad esempio i termini di Coriolis.
}
le equazioni di Navier--Stokes governano la dinamica di una corrente incomprimibile di un fluido a proprietà costanti 
\begin{equation}
\begin{cases}
 \rho \Dt{\vel} = \rho \g - \grad p + \mu \lapl \vel \\
 \dive \vel = 0 \ ,
\end{cases}
\end{equation}
dove l'operatore di derivata materiale applicata al campo di velocità coincide con il campo di accelerazione delle particelle,
\begin{equation}
 \acc = \Dt{\vel} = \pt{\vel} + ( \vel \cdot \grad ) \vel \ .
\end{equation}

\subsubsection{Cinematica relativa}
Sia dato un sistema di riferimento ortonormale inerziale $I$ e un sistema di riferimento non inerziale $J$, con origine $O_I$ e $O_J$ rispettivamente. La velocità e l'accelerazione di $O_J$ definite nel sistema di riferimento inerziale sono $\vel_{O_J}$, $\acc_{O_J}$, mentre vengono indicate con $\bm{\Omega}_{J/I}$ e $\bm{\alpha}_{J/I}$ la velocità e l'accelerazione angolare del sistema di riferimento $J$ rispetto al sistema $I$. Si può esprimere la posizione di un punto $P$ nel sistema di riferimento $I$ come
\begin{equation}\label{eqn:rel:pos}
 \pos = \pos_{O_J} + \pos^J \ ,
\end{equation}
avendo indicato \dots
Dalla derivata temporale della posizione di $P$, si ottiene la velocità
\begin{equation}\label{eqn:rel:vel}
 \vel = \vel_{O_J} + \vel^J + \bm{\Omega}_{J/I} \times \pos^J \ ,
\end{equation}
avendo indicato \dots
Dalla derivata temporale della velocità di $P$, si ottiene l'accelerazione
\begin{equation}\label{eqn:rel:acc}
 \acc = \acc_{O_J} + \acc^J + \bm{\alpha}_{J/I} \times \pos^J + 2~\bm{\Omega}_{J/I} \times \vel^J + 
   \bm{\Omega}_{J/I} \times \left( \bm{\Omega}_{J/I} \times \pos^J \right) \ ,
\end{equation}
avendo indicato \dots

\subsubsection{Equazioni di Navier--Stokes in un sistema di riferimento non inerziale}
Si possono quindi riscrivere le equazioni di Navier--Stokes, utilizzando le grandezze cinematiche riferite al sistema di riferimento non inerziale, mettendo in evidenza i contributi non inerziali e lasciando cadere i pedici $_{J/I}$ della velocità e dell'accelerazione angolare
\begin{equation}
\begin{cases}
\begin{aligned}
 \rho \acc^J = &  \rho \g - \grad p + \mu \lapl \vel + \\
               & - \rho \left[
   \acc_{O_J} + \bm{\alpha} \times \pos^J + 2~\bm{\Omega} \times \vel^J + 
   \bm{\Omega} \times \left( \bm{\Omega} \times \pos^J \right) 
             \right] \ ,
\end{aligned} \\
\dive \vel = 0 \ .
\end{cases}
\end{equation}
In questa espressione delle equazioni compare ancora la velocità $\vel$ riferita al sistema di riferimento inerziale, nel contributo viscoso della quantità di moto e nel vincolo di incomprimibilità.
Per riportartci a un'espressione delle equazioni di Navier--Stokes dove compaiano solo grandezze cinematiche riferite al sistema non inerziale, utilizziamo le formule (\ref{eqn:rel:pos}, \ref{eqn:rel:vel}) nel laplaciano e nella divergenza di $\vel$, ricordando che la velocità $\vel_{O_J}(t)$ dell'origine e la velocità angolare $\bm{\Omega}(t)$ del sistema non inerziale sono solo funzioni del tempo che non dipendono dallo spazio, e che quindi le loro derivate spaziali sono nulle.
%
%\noindent
La divergenza della velocità diventa quindi
\begin{equation}
\begin{aligned}
 \dive \vel & = \dive \left( \vel_{O_J} + \vel^J + \bm{\Omega} \times \pos^J \right) \\
            & = \dive \left( \vel^J + \bm{\Omega} \times ( \pos - \pos_{O_J} ) \right) \\
\end{aligned}
\end{equation}
Il termine $\dive \left( \bm{\Omega} \times \pos_{O_J} \right)$ è nullo poiché il contenuto della parentesi non dipende dallo spazio, ma solo dal tempo. Anche il termine $\dive \left( \bm{\Omega} \times \pos \right)$ risulta nullo, come si può dimostrare facilmente utilizzando un sistema di coordinate ortogonali e i simboli di Ricci per esprimere il prodotto vettoriale,\footnote{
 Chi non ci crede, o non è familiare con i simboli di Ricci, può calcolare il prodotto vettoriale $\bm{\Omega} \times \pos$ utilizzando un sistema cartesiano
 \begin{equation}
   \bm{\Omega} \times \pos = 
     \left( \Omega_y \, z - \Omega_z \, y \right) \bm{\hat{x}} +
     \left( \Omega_z \, x - \Omega_x \, z \right) \bm{\hat{y}} +
     \left( \Omega_x \, y - \Omega_y \, x \right) \bm{\hat{z}} \ ,
 \end{equation}
 e verificare immediatamente che $\dive (\bm{\Omega} \times \pos) = 0$, poichè nessuna componente del vettore dipende dalla coordinata spaziale corrispondente.
}
\begin{equation}
  \dive \left( \bm{\Omega} \times \pos \right) =
  \partial_i \left( \epsilon_{ijk} \Omega_j r_k \right) =
  \epsilon_{ijk} \Omega_j \partial_i r_k =
  \epsilon_{ijk} \Omega_j \delta_{ik} = 
  \epsilon_{iji} \Omega_j = 0 \ ,
\end{equation}
poiché i simboli di Ricci con indici ripetuti sono identicamente nulli.
Di conseguenza, la divergenza del campo di velocità $\vel^J$ riferito al sistema di riferimento non inerziale è uguale alla divergenza del campo di velocità $\vel$ riferito al sistema di riferimento inerziale,
\begin{equation}
  \dive \vel = \dive \vel^J \ .
\end{equation}
%
Non dovrebbe essere difficile trovare l'espressione del laplaciano, una volta osservato il grado della dipendenza dallo spazio nell'espressione (\ref{eqn:rel:pos}): il termine $\vel_{O_J}$ non dipende dallo spazio, mentre il termine $\bm{\Omega} \times \pos^J = \bm{\Omega} \times ( \pos - \pos_{O_J} )$ ha una dipendenza \textit{affine} (e non \textit{lineare}, poiché c'è un termine costante) dalla posizione nello spazio. Ricordandosi che il laplaciano è la somma delle derivate spaziali del secondo ordine (in un sistema di coordinate cartesiane), dovrebbe essere chiaro che il laplaciano di questi due termini è nullo e che quindi vale,
\begin{equation}
\begin{aligned}
 \lapl \vel & = \lapl \left( \vel_{O_J} + \vel^J + \bm{\Omega} \times \pos^J \right) = \\
            & = \lapl \left( \vel_{O_J} + \vel^J + \bm{\Omega} \times ( \pos - \pos_{O_J} ) \right) = \lapl \vel^J
\end{aligned}
\end{equation}
Utilizzando le due relazioni ottenute per la divergenza e il laplaciano del campo di velocità, si può ora riscrivere le equazioni di Navier--Stokes riferite a un sistema di riferimento non inerziale, utilizzando unicamente le grandezze cinematiche relative a questo sistema di riferimento,
\begin{equation}
\begin{cases}
\begin{aligned}
 \rho \acc^J = &  \rho \g - \grad p + \mu \lapl \vel^J + \\
               & - \rho \left[
   \acc_{O_J} + \bm{\alpha} \times \pos^J + 2~\bm{\Omega} \times \vel^J + 
   \bm{\Omega} \times \left( \bm{\Omega} \times \pos^J \right) 
             \right] \ ,
\end{aligned} \\
\dive \vel^J = 0 \ .
\end{cases}
\end{equation}

\subsubsection{Alcune osservazioni sulla natura tensoriale delle equazioni}
{\color{red} Tranne le grandezze cinematiche che descrivono il moto del sistema di riferimento non inerziale e le grandezze cinematiche che descrivono il moto del fluido relativo a questo sistema di riferimento, tutti gli oggetti che compaiono nelle equazioni vettoriali sono indipendenti dal sistema di riferimento utilizzato per descrivere il problema. In una notazione vettoriale ha quindi poco senso indicare in maniera diversa questi termini, a partire dagli operatori differenziali.

TODO: Relazione tra le coordinate e tra i vettori delle basi ortogonali

TODO: Variabili indipendenti delle funzioni

TODO: Calcolo in componenti del gradiente e del laplaciano ``espressi nel sistema di riferimento non inerziale''

}

\subsection{Equazioni in forma adimensionale}
Si ricava ora la forma adimensionale delle equazioni, in un sistema di riferimento con origine fissa, $\vel_{O_J}=\bm{0}, \acc_{O_J} = \bm{0}$, e in rotazione con velocità angolare costante, $\bm{\alpha} = \bm{0}$,
\begin{equation}
\begin{cases}
\begin{aligned}
 \rho \acc^J = \rho \g - \grad p + \mu \lapl \vel^J 
  - 2~\rho \bm{\Omega} \times \vel^J
  - \rho \bm{\Omega} \times \left( \bm{\Omega} \times \pos^J \right)
\end{aligned} \\
\dive \vel^J = 0 \ .
\end{cases}
\end{equation}
Lasciamo cadere l'apice $^J$ su tute le grandezze cinematiche relative al sistema di riferimento non inerziale. Aggiungendo il modulo $\Omega$ della velocità angolare del sistema di riferimento, $\bm{\Omega} = \Omega \bm{\hat{\Omega}}$, alle grandezze di riferimento $( \tilde{\rho}, \, \tilde{U}, \, \tilde{L} )$, usate per creare anche le scale \textit{non indipendenti} di tempo $\tilde{T} = \tilde{L}/\tilde{U}$, di accelerazione $\tilde{A} = \tilde{U}^2 / \tilde{L}$ e di pressione $\tilde{p} = \tilde{\rho}\, \tilde{U}^2$, si può scrivere
\begin{equation}
 \f{\rho \tilde{U}^2}{\tilde{L}} \acc^* = \rho g \bm{\hat{g}}
   - \f{\tilde{\rho}\tilde{U}^2}{\tilde{L}} \grad^* p^*
   + \f{\mu \tilde{U}}{\tilde{L}^2} \Delta^* \bm{u}^*
   - 2 \rho \Omega \tilde{U} \bm{\hat{\Omega}} \times \bm{u}^*
   +   \rho \Omega^2 \tilde{L} \bm{\hat{\Omega}} \times \left( \bm{\hat{\Omega}} \times \bm{r}^* \right) \ ,
\end{equation}
e, dividendo tutti i termini per $\rho \tilde{U}^2/\tilde{L}$,
\begin{equation}
 \acc^* = \f{g \tilde{L}}{\tilde{U}^2} \bm{\hat{g}}
   -  \grad^* p^*
   + \f{\mu }{\rho \tilde{U} \tilde{L}} \Delta^* \bm{u}^*
   - 2 \f{\Omega \tilde{L}}{\tilde{U}} \bm{\hat{\Omega}} \times \bm{u}^*
   +   \f{\Omega^2 \tilde{L}^2}{\tilde{U}^2} \bm{\hat{\Omega}} \times \left( \bm{\hat{\Omega}} \times \bm{r}^* \right) \ ,
\end{equation}
si possono riconoscere i numeri adimensionali caratteristici del problema,
\begin{itemize}
 \item il numero di Reynolds $Re = \frac{\rho \tilde{U} \tilde{L}}{\mu}$, che rappresenta il rapporto tra gli effetti inerziali e gli effetti viscosi;
 \item il numero di Froude, $Fr = \frac{\tilde{U}^2}{g \tilde{L}}$, che rappresenta il rapporto tra gli effetti inerziali e quelli legati al campo di forze di volume (ad esempio, gli effetti gravitazionali ss $\bm{g}$ è il campo di forze gravitazionale);
 \item il numero di Rossby, $Ro = \frac{\tilde{U}}{\Omega \tilde{L}}$ rappresenta il rapporto tra gli effetti inerziali e gli effetti ``di trascinamento'' dovuti al moto del sistema di riferimento non inerziale;
\end{itemize}
e riscrivere le equazioni in forma adimensionale,
\begin{equation}
\begin{cases}
 \acc^* = \f{1}{Fr} \bm{\hat{g}}
   -  \grad^* p^*
   + \f{1}{Re} \Delta^* \bm{u}^*
   - \f{2}{Ro} \bm{\hat{\Omega}} \times \bm{u}^*
   + \f{1}{Ro^2} \bm{\hat{\Omega}} \times \left( \bm{\hat{\Omega}} \times \bm{r}^* \right) \\ \\
 \dive \vel^* = 0 \ .
\end{cases}
\end{equation}

{\color{red} TODO. Utilizzando il valore della velocità angolare della Terra, $\Omega_T = 7.27 \cdot 10^{-5} s^{-1}$, è possibile valutare gli effetti di Coriolis sul moto di una corrente e riconoscere la scala di tempo $\tilde{L}/\tilde{U} \sim 1/\Omega$ dei fenomeni fisici per i quali gli effetti di Coriolis sono rilevanti.\footnote{
 La scala dei tempi deve essere circa uguale all'inverso della velocità angolare del sistema non inerziale, $\tilde{L}/\tilde{U} \sim 1/\Omega$, affinché il numero di $Rossby$ sia circa uguale a $1$ e che quindi i termini di ``trascinamento'' siano dello stesso ordine dei termini di pressione. Se il numero di Rossby è molto maggiore di $1$ gli effetti non inerziali saranno irrilevanti, mentre se il numero di Rossby è molto minore di $1$, il problema sarà dominato da tali effetti.
}
}
