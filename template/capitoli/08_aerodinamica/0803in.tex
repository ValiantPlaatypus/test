\noindent
\begin{tabular}{c}
\begin{minipage}[b]{0.95\textwidth}
\begin{exerciseS}[Corrente attorno al cilindro]
Una corrente piana con velocit\`{a} asintotica orizzontale (parallela all'asse x) $U_\infty=1$ viene perturbata introducendo 
nell'origine del piano un vortice in modo tale da accelerare la corrente nel semipiano superiore e rallentarla 
in quello inferiore per valori positivi di $\Gamma$. Determinare la circolazione necessaria ad ottenere una differenza di componente x della velocit\`{a} pari a 1 
tra i due punti di coordinate (polari) $R=1$ e $\theta=\pm\pi/2$. 
($\Gamma = -\pi$)
\end{exerciseS}
\end{minipage}
\end{tabular}

\sol

\partone
 Flusso non viscoso 2D, incomprimibile e irrotazionale. Circolazione. Corrente indisturbata. Vortice. Sovrapposizione delle cause e degli effetti.

\parttwo
 Una volta scritte (come si ricavano?) le componenti della velocità nel campo di moto, si
impongono le condizioni richieste dal problema per determinare il valore di circolazione necessario.

\begin{equation}
\begin{cases}
  u_r (r,\theta) = U_\infty \cos{\theta} \\
  u_\theta (r,\theta) = - U_\infty \sin{\theta} + \frac{\Gamma}{2\pi r}
\end{cases}
\end{equation}

\vspace{0.2cm}
Si impongono ora le condizioni del problema. Per $\theta_1 = \pi/2$ e $\theta_2 = -\pi/2$, la componente radiale è nulla.

\begin{equation}
  \begin{cases}
    u_{\theta}(R,\theta_1) = -U_\infty + \frac{\Gamma}{2\pi R} \\
    u_{\theta}(R,\theta_2) = U_\infty + \frac{\Gamma}{2\pi R}
  \end{cases}
\end{equation}

Si vuole determinare la differenza delle componenti in direzione x $u_x(R,\theta_1) - u_x(R,\theta_2)$; questa è uguale a $-(u_\theta(R,\theta_1) + u_\theta(R,\theta_2))$. Quindi, per $R=1$:

\begin{equation}
  -\frac{\Gamma}{\pi} = 1 \quad \Rightarrow \quad \Gamma = -\pi
\end{equation}

