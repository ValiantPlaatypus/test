Per correnti irrotazionali ($\omega = \bm{0}$) in un dominio semplicemente connesso ($\bm{u} = \bm{\nabla} \phi$) di fluidi incomprimibili ($\bm{\nabla} \cdot \bm{u} = 0$), il potenziale cinetico soddisfa l'equazione di Laplace $\Delta \phi = 0$. Infatti, inserendo nel vincolo di incomprimibilità la relazione che lega il potenziale cinetico alla velocità si ottiene
\begin{equation}
 0 = \bm{\nabla} \cdot \bm{u} = \bm{\nabla} \cdot (\bm{\nabla} \phi) = \nabla^2 \phi = \Delta \phi .
\end{equation}
Come nel caso della seconda e della terza forma del teorema di Bernoulli per fluidi viscosi, vedi introduzione al capitolo \S\ref{ch:bernoulli}, l'ipotesi di fluido non viscoso non è direttamente necessaria per ottenere l'equazione di Laplace per il potenziale. L'ipotesi di fluido non viscoso rientra però nel requisito che la corrente sia irrotazionale. L'equazione della vorticità per fluido incomprimibile è
\begin{equation}
 \p{\bm{\omega}}{t} + (\bm{u} \cdot \bm{\nabla}) \bm{\omega} = (\bm{\omega} \cdot \bm{\nabla}) \bm{u} + \nu \Delta \bm{\omega} ,
\end{equation}
che per un fluido non viscoso, si riduce a 
\begin{equation}
 \p{\bm{\omega}}{t} + (\bm{u} \cdot \bm{\nabla}) \bm{\omega} = (\bm{\omega} \cdot \bm{\nabla}) \bm{u}  \qquad , \qquad
 \dfrac{D \bm{\omega}}{D t} = (\bm{\omega} \cdot \bm{\nabla}) \bm{u} ,
\end{equation}
dove è stata messa in evidenza la derivata materiale della vorticità, che rappresenta la variazione della vorticità di una particella fluida, che si muove con la velocità del fluido. Se si considera un problema in cui un corpo aerodinamico è investito da una corrente che è uniforme all'infinito a monte, la vorticità all'infinito a monte è nulla: si può dimostrare facilmente allora che $D\bm{\omega} / D t = \bm{0}$, e quindi la vorticità si mantiene costante e nulla, sulle linee di corrente che partono dall'infinito a monte\footnote{\'E immediato convincersi del fatto, utilizzando la descrizione lagrangiana (\ref{eqn:bilanci:vorticitàLagrange}) della vorticità per un fluido non viscoso.}. Per correnti ad alto numero di Reynolds attorno a corpi affusolati, nelle quali non si verificano separazioni, gli effetti viscosi e la vorticità sono confinati in strati limite ``sottili'' attorno ai corpi solidi e in scie ``sottili'' che si staccano da essi. 

\noindent
\'E quindi possibile descrivere una corrente di un fluido incomprimibile ad alto numero di Reynolds, all'\textit{esterno} di queste sottili regioni vorticose, con un modello di fluido non viscoso. Partendo dalle equazioni di Navier--Stokes che governano la dinamica di un fluido viscoso, per le quali vale la condizione al contorno di adesione a parete ($\bm{u} = \bm{b}$), si arriva a un modello che permette di calcolare il campo di velocità dal potenziale cinetico, che soddisfa l'equazione di Laplace $\Delta \phi = 0$ nel dominio e la condizione al contorno di non penetrazione ($\bm{u} \cdot \bm{\hat{n}} = \bm{b} \cdot \bm{\hat{n}}$) in corrispondenza delle pareti solide, e in seguito di calcolare la pressione utilizzando il teorema di Bernoulli.
\begin{equation}
 \begin{cases}
  \p{\bm{u}}{t} + (\bm{u} \cdot \bm{\nabla}) \bm{u} - \nu \Delta \bm{u} + \bm{\nabla}P = \bm{g} \\
  \bm{\nabla} \cdot \bm{u} = 0 \\
  \bm{u}\big|_{wall} = \bm{b}  \qquad + \textit{altre b.c}
 \end{cases}
 \xrightarrow[{\small \begin{aligned} \nu = 0 , & \ \omega = \bm{0} \\ \bm{u} & = \bm{\nabla}\phi \end{aligned} }]{}
  \begin{cases}
  \p{\phi}{t} + \f{|\bm{\nabla}\phi|^2}{2} + P + \chi = C(t) \\
  \Delta \phi = 0 \\
  \p{\phi}{n} = \bm{u}\cdot\bm{\hat{n}} \big|_{wall} = \bm{b}\cdot\bm{\hat{n}} \qquad +  \textit{altre b.c} \\
 \end{cases}
\end{equation}
Il problema di Laplace è lineare ed è quindi valido il principio di sovrapposizione di cause ed effetti, se la geometria del dominio è fissata. Questa considerazione può sembrare strana, ma è determinata dalla possibile presenza di scie che si distaccano dai corpi solidi e che possono evolvere (per problemi non stazionari) all'interno del dominio.
L'equazione di Lapalce può rappresentare anche problemi non stazionari, nonostante non compaia esplicitamente nessuna derivata temporale nell'equazione. La dipendenza temporale può comparire all'interno delle condizioni al contorno e la soluzione si adatta immediatamente ad esse. Memoria della soluzione agli istanti di tempo precendenti è contenuta all'interno delle scie, la cui vorticità è legata al valore di circolazione attorno al corpo (e quindi di portanza) e la cui dinamica è determinata dalle equazioni di governo della vorticità.




