
L'equazione integrale di Von Karman per lostrato limite viene ricavata integrando in $y$ tra $0$ e $\infty$ la componente $x$ della quantità di moto delle equazioni di Prandtl per lo strato limite
\begin{equation}\label{eqn:VKint}
 \underbrace{\int_{y=0}^{\infty} u \dfrac{\partial u}{\partial x} dy}_{(a)} +
 \underbrace{\int_{y=0}^{\infty} v \dfrac{\partial u}{\partial y} dy}_{(b)} -
 \underbrace{\int_{y=0}^{\infty} \nu \dfrac{\partial^2 u}{\partial y^2} dy}_{(c)} -
 \underbrace{\int_{y=0}^{\infty} U U'(x)dy}_{(d)} = 0
\end{equation}
dove è stata indicata con $U(x)$ la velocità della corrente esterna allo strato limite. Si calcolano ora i termini (c), (b). Da (c) si ricava un termine nel quale compare lo sforzo tangenziale a parete $\tau_w$
 \begin{equation}\label{eqn:VKc}
  - \int_{y=0}^{\infty} \nu \dfrac{\partial^2 u}{\partial y^2}(x,y) dy = 
  - \nu \left[ \dfrac{\partial u}{\partial y}  \right]\Bigg|_{y=0}^{\infty} =
    \nu \dfrac{\partial u}{\partial y} (x,0) = \dfrac{\tau_w(x)}{\rho}
 \end{equation}
Il termine (b) richiede un po' di lavoro e attenzione in più (IxP indica l'integrazione per parti).
\begin{equation}\label{eqn:VKb}
\begin{aligned}
 & \int_{y=0}^{\infty} v (x,y) \dfrac{\partial u}{\partial y}(x,y) dy = \\
 & \hspace{7.0cm}  \left(\text{IxP}: \int_{0}^{\infty} v \dfrac{\partial u}{\partial y} = \left[ v u \right]\Bigg|_{y=0}^{\infty} - \int_{0}^{\infty} \dfrac{\partial v}{\partial y} u \right) \\
 & \quad = v(x,\infty) u(x,\infty) - \underbrace{v(x,0) u(x,0)}_{=0} - \int_{0}^{\infty} \dfrac{\partial v}{\partial y} u = \\
 & \hspace{7.5cm}
   \qquad \left( u(x,\infty) = U(x) ; \dfrac{\partial v}{\partial y} = -\dfrac{\partial u}{\partial x}\quad \right) \\
 & \quad = v(x,\infty) U(x) + \int_{y=0}^{\infty}  \dfrac{\partial u}{\partial x} u = \\ 
 & \hspace{4.5cm} \qquad \left( v(x,\infty) = \int_{y=0}^{\infty} \dfrac{\partial v}{\partial y}(x,y) dy =-\int_{y=0}^{\infty} \dfrac{\partial u}{\partial x}(x,y) dy\right) \\
 & \quad =-\int_{y=0}^{\infty} U(x) \dfrac{\partial u}{\partial x} dy + \int_{y=0}^{\infty}  \dfrac{\partial u}{\partial x}(x,y) u(x,y) dy \ . 
\end{aligned}
\end{equation}
Inserendo le espressioni (\ref{eqn:VKc}), (\ref{eqn:VKb}) nell'equazione (\ref{eqn:VKint}), si ottiene
 \begin{equation}
  \begin{aligned} 
  0 & = \int_{0}^{\infty} \left[ 2 u \dfrac{\partial u}{\partial x} -  U \dfrac{\partial u}{\partial x} - U \dfrac{d U}{d x} \right] dy + \dfrac{\tau_w}{\rho} = \\
    & = \int_{0}^{\infty} \left[ \dfrac{\partial u^2}{\partial x}
        - \dfrac{\partial (U u)}{\partial x} + u \dfrac{d U}{d x} - u \dfrac{d U}{d x} \right] dy + \dfrac{\tau_w}{\rho} = \\
    & = \int_{0}^{\infty}  \dfrac{\partial}{\partial x} \left[ u^2 - U u \right] dy
    - \dfrac{dU}{dx} \int_{0}^{\infty} \left[ U - u \right]dy + \dfrac{\tau_w}{\rho} = \\
   & = \dfrac{d}{d x} \int_{0}^{\infty}   \left[ u^2 - U u \right] dy
    - \dfrac{dU}{dx} \int_{0}^{\infty} \left[ U - u \right]dy + \dfrac{\tau_w}{\rho} = \\
   & = - \dfrac{d}{d x} \left( U^2(x) \int_{0}^{\infty}  \dfrac{u}{U}  \left[ 1 -  \dfrac{u}{U} \right] dy \right)
    - \dfrac{dU}{dx} U \int_{0}^{\infty} \left[ 1 - \dfrac{u}{U} \right] dy + \dfrac{\tau_w}{\rho} = \\
   & = - \dfrac{d}{d x} \left[ U^2(x) \delta_2(x) \right] - U (x)U' (x)\delta_1(x)+ \dfrac{\tau_w(x)}{\rho} \\
  \end{aligned}
 \end{equation}
e quindi
 \begin{equation}
  \dfrac{d}{d x} \left[ U^2(x) \delta_2(x) \right] + U (x)U' (x)\delta_1(x) = \dfrac{\tau_w(x)}{\rho}
 \end{equation}
 Infine espandendo i termini, ricordando la definizione di rapporto di forma $H = \delta_1 / \delta_2$, coefficiente di attrito $c_f = \dfrac{\tau_w}{\frac{1}{2}\rho U^2}$
 \begin{equation}
 \begin{aligned}
  & 2 U U' \delta_2 + U^2 \delta'_2 + U U' \delta_1(x) = \dfrac{\tau_w(x)}{\rho} \\
  & [ 2 \delta_2 + \delta_1] U U' + U^2 \delta'_2 = \dfrac{\tau_w(x)}{\rho} \\
  & [ 2 + H ]\delta_2 \dfrac{U'}{U}  + \delta'_2 = \dfrac{\tau_w}{\rho U^2} = \dfrac{c_f}{2}
 \end{aligned}
 \end{equation}

