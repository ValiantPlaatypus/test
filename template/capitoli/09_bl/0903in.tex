\noindent
\begin{exerciseS}[Equazione integrale di Von Karman]
Dati $\rho = 1.225 kg/m^3$, $\nu = 10^{-5} m^2/s$ e velocità esterna $U(x) = 1.45 m/s$, utilizzando le formule per il profilo di velocità e sforzo a parete per lo strato limite turbolento, calcolare lo spessore $\delta(x)$ dello strato limite.
\begin{equation}
  \frac{u(x,y)}{U(x)} = 
  \begin{cases}
    \displaystyle\left( \frac{y}{\delta(x)} \right)^{\frac{1}{7}} &  \qquad y \le \delta(x) \\
    1 &  \qquad y > \delta(x)
  \end{cases}
\end{equation}
\begin{equation}
  \tau_w = 0.0225 \rho U^2 \displaystyle \left( \frac{\nu}{U \delta} \right) ^ {\frac{1}{4}}
\end{equation}

\end{exerciseS}

\sol

\partone Spessori di strato limite. Rapporto di forma. Coefficiente di attrito. Equazione integrale di Von Karman.
\begin{equation}
  c_f = \frac{\tau_w}{\frac{1}{2}\rho U^2 }
\end{equation}

\begin{equation}\label{eqn:VK}
  \frac{d \delta_2}{dx} + \frac{\delta_2}{U(x)}\frac{d U(x)}{dx} (2+H) = \frac{c_f}{2}
\end{equation}

\parttwo Si calcolano gli spessori di strato limite $\delta_1$ e $\delta_2$, il raporto di forma $H$ e il coefficiente di attrico $c_f$; poi si inseriscono nell'equazione integrale di Von Karman.
Poichè la velocità esterna non varia in x, il secondo termine si annulla.

Gli spessori di strato limite e il rapporto di forma hanno valore $\delta_1 = \frac{1}{8}\delta$,
 $\delta_2 = \frac{7}{72}\delta$, $H = \frac{9}{7}$.

Il coefficiente di attrito vale:
\begin{equation}
\begin{aligned}
  c_f & = \frac{\tau_w}{\frac{1}{2}\rho U^2 } = \\
  & = \frac{2}{\rho U^2} 0.0225 \rho U^2 
  \displaystyle \left( \frac{\nu}{U \delta} \right)^{\frac{1}{4}} = \\
      & = 0.045 \displaystyle \left( \frac{\nu}{U \delta} \right)^{\frac{1}{4}}
\end{aligned}
\end{equation}

Inserendo nell'equazione di Von Karman:

\begin{equation}
 \frac{d \delta_2}{dx}  = \frac{c_f}{2} \qquad\Rightarrow \qquad
  \frac{7}{72}\frac{d \delta(x)}{d x} = 0.0225 \displaystyle \left( \frac{\nu}{U \delta} \right)^{\frac{1}{4}}
\end{equation}

Integrando tra 0 e x, avendo imposto $\delta(0) = 0$, si ottiene:

\begin{equation}
  \delta(x) = 0.0225 \frac{90}{7} \displaystyle \left( \frac{\nu}{U } \right)^{\frac{1}{4}} x^{\frac{4}{5}}
\end{equation}

\vspace{3.0cm}
\paragraph{Dalle equazioni di Prantdl per lo strato limite all'equazione
 integrale di VK.} L'equazione integrale di VK (\ref{eqn:VK}) viene ricavata
 integrando in $y$ tra $0$ e $\infty$ la componente $x$ della quantità di
 moto delle equazioni di Prandtl per lo strato limite
\begin{equation}\label{eqn:VKint}
 \underbrace{\int_{y=0}^{\infty} u \dfrac{\partial u}{\partial x} dy}_{(a)} +
 \underbrace{\int_{y=0}^{\infty} v \dfrac{\partial u}{\partial y} dy}_{(b)} -
 \underbrace{\int_{y=0}^{\infty} \nu \dfrac{\partial^2 u}{\partial y^2} dy}_{(c)} -
 \underbrace{\int_{y=0}^{\infty} U U'(x)dy}_{(d)} = 0
\end{equation}
dove è stata indicata con $U(x)$ la velocità della corrente esterna allo strato 
 limite. Si calcolano ora i termini (c), (b). Da (c) si ricava un termine 
 nel quale compare lo sforzo tangenziale a parete $\tau_w$
 \begin{equation}\label{eqn:VKc}
  - \int_{y=0}^{\infty} \nu \dfrac{\partial^2 u}{\partial y^2}(x,y) dy = 
  - \nu \left[ \dfrac{\partial u}{\partial y}  \right]\Bigg|_{y=0}^{\infty} =
    \nu \dfrac{\partial u}{\partial y} (x,0) = \dfrac{\tau_w(x)}{\rho}
 \end{equation}
Il termine (b) richiede un po' di lavoro e attenzione in più (IxP indica l'integrazione per parti).
\begin{equation}\label{eqn:VKb}
\begin{aligned}
% \int_{y=0}^{\infty} v (x,y) \dfrac{\partial u}{\partial y}(x,y) dy 
%  & = \qquad & \qquad \left( v(x,y) = \int_{\xi=0}^{y} \dfrac{\partial v}{\partial y}(x,\xi) d\xi \right) \\
%  & \qquad & = 
 & \int_{y=0}^{\infty} v (x,y) \dfrac{\partial u}{\partial y}(x,y) dy =
      & \qquad \left(\text{IxP}: \int_{0}^{\infty} v \dfrac{\partial u}{\partial y} = \left[ v u \right]\Bigg|_{y=0}^{\infty} - \int_{0}^{\infty} \dfrac{\partial v}{\partial y} u \right) \\
 & \quad = v(x,\infty) u(x,\infty) - \underbrace{v(x,0) u(x,0)}_{=0} - \int_{0}^{\infty} \dfrac{\partial v}{\partial y} u =  & 
   \qquad \left( u(x,\infty) = U(x) ; \dfrac{\partial v}{\partial y} = -\dfrac{\partial u}{\partial x}\quad \right) \\
 & \quad = v(x,\infty) U(x) + \int_{y=0}^{\infty}  \dfrac{\partial u}{\partial x} u = 
 & \qquad \left( v(x,\infty) = \int_{y=0}^{\infty} \dfrac{\partial v}{\partial y}(x,y) dy =-\int_{y=0}^{\infty} \dfrac{\partial u}{\partial x}(x,y) dy\right) \\
 & \quad =-\int_{y=0}^{\infty} U(x) \dfrac{\partial u}{\partial x} dy + \int_{y=0}^{\infty}  \dfrac{\partial u}{\partial x}(x,y) u(x,y) dy & \\ 
\end{aligned}
\end{equation}
Inserendo le espressioni (\ref{eqn:VKc}), (\ref{eqn:VKb}) nell'equazione
 (\ref{eqn:VKint}), si ottiene:
 \begin{equation}
  \begin{aligned} 
  0 & = \int_{0}^{\infty} \left[ 2 u \dfrac{\partial u}{\partial x} -  U \dfrac{\partial u}{\partial x} - U \dfrac{d U}{d x} \right] dy + \dfrac{\tau_w}{\rho} = \\
    & = \int_{0}^{\infty} \left[ \dfrac{\partial u^2}{\partial x}
        - \dfrac{\partial (U u)}{\partial x} + u \dfrac{d U}{d x} - u \dfrac{d U}{d x} \right] dy + \dfrac{\tau_w}{\rho} = \\
    & = \int_{0}^{\infty}  \dfrac{\partial}{\partial x} \left[ u^2 - U u \right] dy
    - \dfrac{dU}{dx} \int_{0}^{\infty} \left[ U - u \right]dy + \dfrac{\tau_w}{\rho} = \\
   & = \dfrac{d}{d x} \int_{0}^{\infty}   \left[ u^2 - U u \right] dy
    - \dfrac{dU}{dx} \int_{0}^{\infty} \left[ U - u \right]dy + \dfrac{\tau_w}{\rho} = \\
   & = - \dfrac{d}{d x} \left( U^2(x) \int_{0}^{\infty}  \dfrac{u}{U}  \left[ 1 -  \dfrac{u}{U} \right] dy \right)
    - \dfrac{dU}{dx} U \int_{0}^{\infty} \left[ 1 - \dfrac{u}{U} \right] dy + \dfrac{\tau_w}{\rho} = \\
   & = - \dfrac{d}{d x} \left[ U^2(x) \delta_2(x) \right] - U (x)U' (x)\delta_1(x)+ \dfrac{\tau_w(x)}{\rho} \\
  \end{aligned}
 \end{equation}
 Riassumendo,
 \begin{equation}
  \dfrac{d}{d x} \left[ U^2(x) \delta_2(x) \right] + U (x)U' (x)\delta_1(x) = \dfrac{\tau_w(x)}{\rho}
 \end{equation}
 Infine espandendo i termini, ricordando la definizione di rapporto di forma $H = \delta_1 / \delta_2$, coefficiente di attrito $c_f = \dfrac{\tau_w}{\frac{1}{2}\rho U^2}$
 \begin{equation}
 \begin{aligned}
  & 2 U U' \delta_2 + U^2 \delta'_2 + U U' \delta_1(x) = \dfrac{\tau_w(x)}{\rho} \\
  & [ 2 \delta_2 + \delta_1] U U' + U^2 \delta'_2 = \dfrac{\tau_w(x)}{\rho} \\
  & [ 2 + H ]\delta_2 \dfrac{U'}{U}  + \delta'_2 = \dfrac{\tau_w}{\rho U^2} = \dfrac{c_f}{2}
 \end{aligned}
 \end{equation}









