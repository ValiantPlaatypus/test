\noindent
\begin{exerciseS}[Strato limite di Blasius e sforzo a parete]
Ricavare le equazioni di Prandtl dello strato limite laminare (hp...). Ricavare poi l'equazione
 di Blasius per lo strato limite laminare (hp...). Ricavare la soluzione con un metodo numerico:
  in particolare, ricavare il valore di $\frac{d^2 g}{d\eta^2}\big|_{\eta=0}$ da inserire nella formula
  dello sforzo viscoso a parete (Shooting method ed iterazioni di Newton).
  
($g''(0) = 0.332$)
\end{exerciseS}

\sol

\partone Equazioni di Prandtl. Equazione di Blasius. Soluzione in similitudine. 
Shooting method.

\parttwo Le equazioni di Prandtl dello strato limite possono essere ricavate tramite 
ragionamenti sugli ordini di grandezza delle grandezze fisiche.

\begin{equation}
  \begin{cases}
    u\frac{\partial u}{\partial x} + v\frac{\partial u}{\partial y} - \nu \frac{\partial^2 u}{\partial y^2} = - \frac{1}{\rho}P(x) \\
    \frac{\partial u}{\partial x} + \frac{\partial x}{\partial y} = 0
  \end{cases}
\end{equation}

Sfruttando la definizione di funzione di corrente, l'ipotesi che $U(x)$ sia costante, si cerca una soluzione
in similitudine delle equazioni di Prandtl.
Siano $\eta = y/\delta(x)$ e $\psi = U(x) \delta(x) g(\eta)$, si ricava l'andamento dello spessore dello strato 
limite $\delta(x) = \sqrt{\frac{\nu x}{U}}$ e l'equazione di Blasius
\begin{equation}
 g''' + \frac{1}{2} g g'' = 0
\end{equation}
con le condizioni al contorno
\begin{equation}
 \begin{cases}
 g(0) = 0 \\
 g'(0) = 0 \\
 \lim_{\eta\to \infty}
 g'(\eta) = 1
 \end{cases}
\end{equation}

La formula per lo sforzo viscoso a parete è:
\begin{equation}
  \tau_w = \mu \frac{\partial u}{\partial y}\big|_{y=0} = \mu \frac{\partial^2 \psi}{\partial y^2}\big|_{\eta=0} = 
  \mu \frac{U}{\delta(x)} g''(0)
\end{equation}

Per risolvere l'equazione di Blasius con metodi numerici, si può incontrare qualche difficoltà
nell'imporre la condizione al contorno per $\eta \to \infty$.
Tramite uno \textit{shooting method} si può risolvere il problema ai valori al contorno, tramite 
la soluzione di problemi ai valori iniziali insieme a un metodo per trovare gli zeri di una
funzione (es. Newton).
Il dominio semi-infinito viene troncato. Il dominio numerico è quindi $[0,\bar{\eta}]$. L'equazione 
scalare di terzo ordine, viene scritta come sistema del primo ordine. Invece di imporre la condizione
all'infinito, viene imposto il valore di $g''(0)=\alpha$. Si risolve l'equazione. Si trova il valore di $g'_n$in $\bar{\eta}$.
Si itera fino a quando il valore assoluto di $F(\alpha) = g'_n(\bar{\eta};\alpha) - \lim_{\eta\to \infty} g'(\eta)$ non è inferiore 
a una tolleranza stabilita.

Per esempio, partendo da $\alpha=0.1$, con una tolleranza $tol = 1E-09$:
\begin{center}
\begin{verbatim}
nIter    g"(0)    res 
  1     0.1000  5.508e-01 
  2     0.2836  9.975e-02 
  3     0.3308  2.575e-03 
  4     0.3320  1.660e-06 
  5     0.3320  1.804e-12
\end{verbatim}
\end{center}
    


