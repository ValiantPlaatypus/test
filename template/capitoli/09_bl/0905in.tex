\noindent
\begin{exerciseS}[Strato limite di Blasius: resistenza di una lamina]
    Una lamina piana molto sottile viene investita parallelamente su entrambe le sue facce da una corrente di velocità uniforme $U$ di un fluido di densità $\rho$ e viscosità dinamica $\mu$.
Nell'ipotesi che il problema possa essere approssimato con le equazioni bidimensionali
dello strato limite laminare con velocità asintotica costante (Blasius):
\begin{itemize}
 \item determinare l'espressione della resistenza di attrito della lamina rettangolare in funzione della sua lunghezza $\ell$ (lati paralleli alla velocità della corrente esterna) e della sua larghezza $b$;
 \item determinare la forma della lamina rettangolare di area data $A = \ell \, b$ che ha resistenza minima.
\end{itemize}
\end{exerciseS}

\sol 

\partone Strato limite laminare. Equazione di Blasius. 

\parttwo Nell'ipotesi in cui si possa utilizzare la soluzione di Blasius, considerata 
omogenea in apertura, la resistenza di attrito della lamina è
\begin{equation}
\begin{aligned}
    D & = 2 b \int_0^{\ell} \tau_w(x) dx = & \displaystyle\left(\tau_w = \mu \frac{U}{\delta(x)} g''(0)\right)\\
    & = 2 g''(0) \mu b U \int_0^{\ell} \frac{1}{\delta(x)} dx = & \displaystyle\left(\delta(x) = \sqrt{\frac{\nu x}{U}}\right) \\
    & = 2 g''(0) \frac{\mu b U^{\frac{3}{2}} }{\sqrt{\nu}} \int_0^{\ell} \frac{1}{\sqrt{x}} dx = & \\
    & = 4 g''(0) \sqrt{\rho \mu}  U^{\frac{3}{2}}  \, b \sqrt{\ell} \ , 
\end{aligned}
\end{equation}
o raccogliendo nel coefficiente $K$ tutti i parametri costanti del problema, $D(\ell, \, b) = K \, b \sqrt{\ell}$.

\paragraph{Resistenza minima, ad area $A$ fissata.}
Per trovare il valore minimo della resistenza di una lamina, si inserisce il vincolo $A = \ell \, b $ all'interno dell'espressione della resistenza per ottenere l'espressione in funzione di una sola variabile,
\begin{equation}
    D_A(\ell) = K \, A \, \dfrac{1}{\sqrt{\ell}} = \tilde{K} \dfrac{1}{\sqrt{\ell}} \ .
\end{equation}
Si osserva che la resistenza minima si ottiene per una lamina di lunghezza $\ell$ infinita e di spessore $b$ infinitesimo, $D_A(\ell \rightarrow \infty) \rightarrow 0$.
% \textit{Alcune osservazioni.} 
% \begin{itemize}
% \item A parità di superficie, conviene una lamina lunga o larga? Sempre?
% \item Avviene transizione?
% \item Commentare risultati, in caso avvenga transizione a turbolenza.
% \end{itemize}

\paragraph{Commento.}
Il problema considera un modello laminare dello strato limite su tutta la lunghezza della lamina, senza considerare la transizione a un regime di moto turbolento. In generale la transizione a un regime di moto turbolento avviene a una determinata distanza dal bordo di attacco della lamina. Se avviene la transizione a un regime di moto turbolento, aumentando la lunghezza della lamina aumenta la sua superficie soggetta a un regime turbolento, nel quale gli sforzi a parete sono generalmente maggiori, rispetto a uno strato limite laminare.
