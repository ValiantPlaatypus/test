\section{Stabilità idrodinamica}

In questa sezione viene data una breve introduzione allo studio della stabilità idrodinamica, viene accennato alla possibile inadeguatezza dell'analisi modale e viene introdotta l'analisi lineare non modale, in grado di spiegare la dinamica di alcune correnti che non viene colta dall'analisi modale.
%
\newline
La stabilità idrodinamica si occupa delle prime instabilità delle correnti laminari, di solito descritte con l'approccio deterministico della teoria dei sistemi dinamici. Le equazioni di Navier--Stokes dotate delle appropriate condizioni iniziali, condizioni al contorno (ed eventualmente condizioni di compatibilità) descrivono il moto di una corrente incomprimibile di un fluido newtoniano. Nella forma adimensionale delle equazioni,
\begin{equation}
 \begin{cases}
  \dfrac{\partial \bm{u}}{\partial t} + (\bm{u} \cdot \bm{\nabla} ) \bm{u}
  -\dfrac{1}{Re} \nabla^2 \bm{u} + \bm{\nabla} p = \bm{0} \ , \\
  \bm{\nabla} \cdot \bm{u} = 0 \ ,
 \end{cases}
\end{equation}
il numero di Reynolds $Re$ è un parametro che rappresenta il rapporto tra la ``scala'' degli effetti inerziali e quella degli effetti viscosi. Fin dai primi studi di Reynolds (1883) sulla transizione al regime turbolento nelle correnti in canali, questo numero adimensionale è stato riconosciuto come il principale parametro che determina il regime della corrente, laminare o turbolento.
%
\vspace{0.3cm}
\newline
La turbolenza è frequentemente considerata come l'ultimo problema non risolto della meccanica classica, e solo recentemente sono stati spiegati alcuni fenomeni riguardanti la stabilità idrodinamica. L'approccio tradizionale all'analisi di stabilità dei sistemi dinamici consiste nella ricerca dei punti di equilibrio, ossia le soluzioni stazionarie del sistema, e nella soluzione del problema agli autovalori del sistema linearizzato attorno all'equilibrio. L'approccio modale, ``alla Lyapunov'', permette di ottenere alcune informazioni locali sulla stabilità di un equilibrio in seguito a condizioni iniziali perturbate.
\'E noto che l'analisi modale delle correnti stazionarie, che rappresentano equilibri delle equazioni di Navier--Stokes, riesce a descrivere il comportamento di alcune correnti, come ad esempio la convezione di Rayliegh--Benard, la corrente di Taylor--Couette tra due cilindri coassiali rotanti o la scia di Von K\'arm\'an dietro un cilindro, mentre non è in grado di descrivere le osservazioni sperimentali di altre correnti, come la corrente di Newton--Couette e la corrente di Poiseuille, sia in canali piani sia in canali cilindrici, e alcune correnti a getto.
%
\vspace{0.3cm}
\newline
\subsection{Fallimento dell'analisi modale: correnti amplificatrici di disturbi}
Per le \textit{correnti aperte}, per le quali le particelle fluide rimangono nel dominio euleriano considerato in un intervallo di tempo finito, vengono definiti i concetti di \textit{instabilità assoluta} e \textit{instabilità convettiva}. Queste definizioni sono legate alla risposta impulsiva del sistema e determinano la classificazione storica delle correnti aperte in \textbf{oscillatori fluidodinamici} e \textbf{correnti amplificatrici del rumore}: gli oscillatori fluidodinamici mostrano una dinamica oscillatoria intrinseca (auto-sostenuta) a una precisa frequenza e sono poco influenzate da disturbi esterni; nelle correnti amplificatrici del rumore, invece, disturbi con un ampio spettro di frequenze mostrano un'enorme amplificazione mentre vengono trasportate lungo il dominio.
%
\newline
Per spiegare il fallimento dell'analisi modale, alcuni studiosi si sono concentrati sull'inadeguatezza dello studio di stabilità del sistema linearizzato e hanno incluso alcuni termini non lineari per lo studio di stabilità dei sistemi con la cosiddetta \textit{teoria dell'instabilità secondaria}. Comunque, è emerso che la principale ragione della discrepanza tra i risultati dell'analisi modale e le osservazioni sperimentali riguardanti correnti amplificatrici di rumore, dominate dalla convezione, è dovuta alla \textbf{non-normalità} del sistema linearizzato di Navier--Stokes. In particolare, il termine convettivo rende il problema linearizzato di Navier--Stokes non-normale, cioè descritto da un operatore linearizzato (l'equivalente della matrice jacobiana) non simmetrico, che in generale ha modi (autofunzioni, autovettori) non ortogonali. I sistemi non-normali hanno una dinamica lineare attorno a un equilibrio stabile più complicata rispetto alla dinamica dei sistemi lineari normali, caratterizzati da autovettori ortogonali.
%
\newline
I sistemi non-normali possono ad esempio mostrare il fenomeno del \textbf{crescita transitoria}: partendo nelle vicinanze di un equilibrio stabile, lo stato del sistema può momentaneamente allontanarsi dall'equilibrio (ad esempio, misurando la distanza con una norma energetica) per poi tendere asintoticamente ad esso.
Quando sono forzati, i sistemi non-normali possono mostrare una \textbf{pseudorisonanza}, una grande amplificazione di disturbi con frequenza lontana da quella del autovalore meno stabile, e una grande amplificazione di disturbi stocastici. A causa della loro natura non-normale, sia la risposta deterministica sia la risposta stocastica delle correnti amplificatrici di rumore attorno a equilibri stabili sono state studiate con analisi di risposta lineari non-modali, sia nel dominio del tempo sia nel dominio della frequenza. L'analisi della risposta di questi sistemi viene affrontato come un problema di ottimizzazione: nel dominio del tempo, tecniche classiche di algebra lineare vegono usare insieme a un'integrazione ``avanti e indietro'' nel tempo per risolvere il problema di ottimizzazione e ricavare la condizione iniziale ottimale, cioè quella associata alla massima crescita transitoria; nel dominio delle frequenze, tecniche basate su una decomposizione ai valori singolari (SVD) della trasformata di Fourier del \textit{resolvent operator}\footnote{
La risposta armonica di un sistema lineare può essere studiata nel dominio delle frequenze. La trasformata di Fourier del sistema lineare % \vspace{-0.2cm}
%\begin{equation}
$\qquad \underline{\dot{x}} = \underline{\underline{A}} \, \underline{x} +
                      \underline{\underline{B}} \, \underline{f} \qquad $
%\end{equation} \vspace{-0.3cm}
è %\vspace{-0.2cm}
%\begin{equation} 
$\qquad i\omega\underline{\hat{x}} = \underline{\underline{A}} \, \underline{\hat{x}} +
                      \underline{\underline{B}} \, \underline{\hat{f}} \ $.
%\end{equation}
Il \textit{resolvent operator} $\mathcal{R}(\omega)$ del sistema nel dominio delle frequenze è l'operatore che lega la trasformata di Fourier $\underline{\hat{f}}(\omega)$ della forzante $\underline{f}(t)$ alla trasformata di Fourier $\underline{\hat{x}}(\omega)$ dello stato (o dell'uscita) $\underline{x}(t)$ del sistema,
\begin{equation}
    \underline{\hat{x}}(\omega) 
    = [-i\omega \underline{\underline{I}} + \underline{\underline{A}}]^{-1}
    \underline{\underline{B}} \, \underline{\hat{f}}(\omega) = \mathcal{R}(\omega) \underline{\hat{f}}(\omega) \qquad \rightarrow \qquad 
    \mathcal{R}(\omega) =[-i\omega \underline{\underline{I}} + \underline{\underline{A}}]^{-1} \ .
\end{equation}
}
per determinare la perturbazione armonica maggiormente amplificata dal sistema e la sua amplificazione; la risposta stocastica del sistema può infine essere determinata dalla risposta armonica del sistema in tutto lo spettro di frequenze.

{\color{red} \paragraph{Approccio variazionale e metodi dell'aggiunto} }
{\color{red} \paragraph{Analisi di sensitività} }
{\color{red} \paragraph{Tecniche di controllo} }

\subsection{Sistemi dipendenti da un parametro e stabilità strutturale}
Spesso la dinamica di un sistema è influenzata dal valore di alcuni suoi parametri caratteristici. Ad esempio la dinamica di una corrente incomprimibile di un fluido viscoso nuewtoniano dipende dal numero di Reynolds $Re$, come facilmente intuibile dalla forma adimensionale delle equazioni di Navier--Stokes. Tipicamente, il numero di Reynolds determina il regime di moto della corrente: qualitativamente, in alcune correnti quando il numero di Reynolds supera un valore critico la soluzione laminare stabile lascia spazio a un regime caotico, turbolento.
\newline
Lo studio della stabilità di un sistema dipendente dal valore di uno o più parametri è l'ogetto della \textbf{teoria delle biforcazioni}, che si occupa di studiare i cambiamenti qualitativi della dinamica di un sistema al variare dei suoi parametri. Di seguito, viene fornito lo studio di stabilità del sistema di Lorenz in funzione del valore di un suo parametro, come primo esempio di studio di stabilità di un sistema fluidodinamico ridotto: infatti il sistema tridimensionale di Lorenz può essere ricavato da un'approssimazione di Galerkin con base armonica della soluzione delle equazioni di Boussinesq per il problema della convezione di un fluido tra due superfici parallele. La stabilità del sistema viene studiata al variare di un parametro che è proporzionale al numero di Rayliegh $Ra$ caratteristico del problema,
\begin{equation}
 \rho \sim Ra = \dfrac{\alpha g \Delta T h^3}{\nu D} \ .
\end{equation}


\vspace{0.2cm}
\noindent
Prima di indagare il sistema di Lorenz vengono introdotti alcuni concetti utili per 
 studiare la stabilità di un generico sistema dinamico,
\begin{equation}
 \dfrac{d \bm{x}}{d t } = \bm{f}(\bm{x}(t),t) \qquad , \qquad \bm{x}(0) = \bm{x}_0 \ .
\end{equation}
Alcuni di questi concetti non sono nuovi; tutte queste definizioni verranno usate 
 (e risulteranno più chiare) nella sezione successiva dedicata al sistema di Lorenz.

\vspace{0.2cm}
\noindent
\textbf{Spazio delle fasi e stati di un sistema.}
 Lo spazio delle fasi è uno spazio nel quale i punti rappresentano i possibili stati del sistema. 
 Lo stato di un sistema dinamico è identificato dal valore delle sue variabili di stato,
 ovvero le variabili che lo descrivono in maniera esaustiva da poterne prevederne
 l'evoluzione.\footnote{Conoscendo lo stato del sistema \textbf{con esattezza}
 è possibile descrivere l'evoluzione libera del sistema, in assenza di perturbazioni
 e forze esterne. In fondo a  questa sezione, sarà più chiara la necessità di conoscere
\textit{con esattezza} lo stato iniziale del sistema, per prevederne l'evoluzione.}
L'evoluzione libera di un sistema dinamico viene descritta dalle traiettorie nel suo
 spazio delle fasi.

\vspace{0.2cm}
\noindent
\textbf{Equilibri e cicli limite.} Un equilibrio $\overline{\bm{x}}$ del sistema
 è una soluzione stazionaria delle equazioni del sistema dinamico, cioè 
\begin{equation}
 \bm{0} = \bm{f}(\overline{\bm{x}}) \ .
\end{equation}
Un ciclo limite di periodo $T$ è una traiettoria periodica del sistema, tale per cui
\begin{equation}
 \bm{x}(t+T) = \bm{x}(t) \quad , \quad \forall t \ ,
\end{equation}
rappresentata nello spazio delle fasi da un'orbita chiusa (e isolata).

\vspace{0.2cm}
\noindent
\textbf{Stabilità alla Lyapunov.}
Lo studio di stabilità alla Lyapunov riguarda l'evoluzione locale del sistema dinamico
 con \textbf{condizioni iniziali perturbate}. Qualitativamente, un punto di equilibrio è stabile
 se, partendo da uno stato ``vicino'' all'equilibrio, lo stato del sistema rimane per
 sempre ``vicino'' all'equilibrio. Inoltre, l'equilibrio è asinotiticamente stabile se
 lo stato converge verso il punto di equilibrio, $\bm{x} \rightarrow \bm{\overline{x}}$
 per $t \rightarrow \infty$.
\newline
 La stabilità di Lyapunov di un equilibrio può essere
 indagata attraverso l'analisi degli autovalori del sistema linearizzato attorno al punto
 di equilibrio.

\vspace{0.2cm}
\noindent
\textbf{Stabilità strutturale.}
La stabilità strutturale considera l'evoluzione del sistema in seguito a perturbazioni
 del sistema stesso. Un sistema dinamico è strutturalmente stabile se
 le traiettorie nel suo spazio delle fasi non cambiano qualitativamente: ad esempio, in un 
 sistema strutturalmente stabile alla perturbazione di un parametro,
 non cambiano il numero dei punti di equilibrio e cicli limite.


\subsubsection{Sistema dinamico di Lorenz}
In questa sezione si descrive, senza nessuna pretesa di completezza,
 lo di studio di stabilità del sistema dinamico di Lorenz,
\begin{equation}
    \begin{cases}
      \dot{X} = - \sigma X + \sigma Y \\
      \dot{Y} = - Y + \rho X - X Z \\
      \dot{Z} = - \beta Z + X Y \ ,
    \end{cases}
\end{equation}
 come primo esempio di studio di stabilità di un sistema fluidodinamico.
Si studia la stabilità del sistema di Lorenz al variare del parametro $\rho$,
 mantenendo costante il valore dei parametri $\sigma$ e $\beta$.
Lorenz usò come valori $\sigma = 10$ e $\beta = 8/3$.
 Il numero di Prantdl assume un valore paragonabile a quello dell'acqua alla temperatura
 di $20^\circ C$, che vale circa $Pr \approx 7$. Il numero di Prandtl per l'aria e altri gas
 vale circa $0.7$. Il valore $\beta = 8/3$ corrisponde a
 un numero d'onda fondamentale in direzione $x$ uguale a
 $\frac{k}{2} = \frac{1}{2} \sqrt{\frac{4}{\beta} - 1} = \frac{\sqrt{2}}{2} =
 0.3536$.

\paragraph{Punti di equilibrio}
%
\begin{figure}[t]
  \centering
  \begin{tabular}{cc}
  \begin{overpic}[width=0.45\textwidth, trim={40 00 60 0}, clip]{./fig/eq_rho+1_01}
  \put(-5,62){(a) $\rho = 1.01$}
  \put(-10,50){E1}  \put(-10,32){E2}  \put(-10,14){E3}
  \end{overpic} \hfill 
  \begin{overpic}[width=0.45\textwidth, trim={40 00 60 0}, clip]{./fig/eq_rho+1_1}
  \put(-5,62){(b) $\rho = 1.1$}
  \end{overpic}  \\
  \begin{overpic}[width=0.45\textwidth, trim={40 00 60 0}, clip]{./fig/eq_rho+1_5}
  \put(-5,62){(c) $\rho = 1.5$}
  \put(-10,50){E1}  \put(-10,32){E2}  \put(-10,14){E3}
  \end{overpic} \hfill 
  \begin{overpic}[width=0.45\textwidth, trim={40 00 60 0}, clip]{./fig/eq_rho+5_0}
  \put(-5,62){(d) $\rho = 5.0$}
  \end{overpic}  \\
  \begin{overpic}[width=0.45\textwidth, trim={40 00 60 0}, clip]{./fig/eq_rho+15_0}
  \put(-5,62){(e) $\rho =15.0$}
  \put(-10,50){E1}  \put(-10,32){E2}  \put(-10,14){E3}
  \end{overpic} \hfill 
  \begin{overpic}[width=0.45\textwidth, trim={40 00 60 0}, clip]{./fig/eq_rho+24_7}
  \put(-5,62){(f) $\rho =24.7368$}
  \end{overpic}  \\
  \end{tabular}
\caption{Equilibri del sistema di Lorenz per diversi valori del parametro $\rho > 1$.
    Il campo vettoriale della velocità è sovrapposto al campo di temperatura ``scalato''
    $\tilde{T} = \frac{T(x,z)-T_w}{\delta T}$, i cui valori al contorno sono
    $\tilde{T}_w=0$ sulla superficie inferiore a $z=0$ e $\tilde{T}_c = -1$ sulla
    superficie superiore a $z=1$.}
\label{fig:lorenz-equil}
\end{figure}
I punti di equilibrio del sistema di Lorenz soddisfano le equazioni stazionarie
\begin{equation}
    \begin{cases}
      0 = - \sigma X + \sigma Y  & \rightarrow Y = X \\
      0 = - Y + \rho X - X Z & \hspace{2.0cm} \searrow \hspace{2.0cm}\rightarrow  X[X^2-(\rho-1)] = 0 \\
      0 = - \beta Z + X Y  & \hspace{1.0cm} \rightarrow \hspace{1.0cm}  X^2 = \beta Z \hspace{0.5cm} \nearrow 
    \end{cases}
\end{equation}
L'equazione $X(X^2 - \rho) = 0$ ha una sola soluzione reale se $\rho < 1$, tre soluzioni
 per $\rho \geq 1$. Quindi per valori di $\rho < 1$ esiste un unico punto di equilibrio,
\begin{equation}
 \text{\textbf{E1}: } (\overline{X}_1, \overline{Y}_1, \overline{Z}_1) = (0,0,0) \ .
\end{equation}
Per valori di $\rho \geq 1$ esistono tre punti di equilibrio,
\begin{equation}
\begin{aligned}
 \text{\textbf{E1}: } &  (\overline{X}_1, \overline{Y}_1, \overline{Z}_1) = (0,0,0) \ , \\
 \text{\textbf{E2}: } &  (\overline{X}_2, \overline{Y}_2, \overline{Z}_2) =
 (-\sqrt{\beta(\rho-1)},-\sqrt{\beta(\rho-1)},\rho-1) \ , \\
 \text{\textbf{E3}: } &  (\overline{X}_3, \overline{Y}_3, \overline{Z}_3) =
 (+\sqrt{\beta(\rho-1)},+\sqrt{\beta(\rho-1)},\rho-1) \ .
\end{aligned}
\end{equation}
%
I campi di velocità e temperatura degli equilibri del sistema fisico
 corrispondenti ai punti di equilibrio del sistema di Lorenz sono raffigurati in
 figura \ref{fig:lorenz-equil}. L'equilibrio \textbf{E1} rappresenta la soluzione
 statica, il cui il campo di velocità è nullo: non sono presenti moti convettivi
 e la trasmissione della temperatura avviene solo per conduzione (diffusione).
Questo equilibrio è stabile per valori del parametro $\rho<1$. Per valori $\rho > 1$
 questo equilibrio diventa instabile e nascono i due equilibri ``simmetrici''
 \textbf{E2,3} che rappresentano dei moti convettivi stabili, traslati tra di loro
 di metà della lunghezza d'onda $\frac{k}{2}$: i moti convettivi tendono a portare il
 fluido caldo dalla parete inferiore (a temperatura maggiore, per $\rho>0$, corrispondente
 a $Ra > 0$, e quindi $\Delta T = T_w - T_c >0$) verso la parete superiore.
 La nascita di due equilibri stabili
 in corrispondenza del un cambio di stabilità di un equilibrio esistente è
 caratteristico dei sistemi dotati di simmetria.\footnote{
Un esempio strutturale è quello della trave caricata di punta a compressione. Per
 valori limitati del carico esiste un'unica soluzione, stabile, rappresentata dalla
 trave senza freccia. Quando il carico di compressione supera il valore critico,
 questa soluzione diventa instabile. Nel problema piano, nascono due configurazioni
 di equilbirio stabili del sistema strutturale: la trave può inflettersi 
 (in maniera indifferente in assenza di imperfezioni) verso destra o verso sinistra.
} Questo cambiamento qualitativo nel piano delle fasi, corrisponde alla
 \textit{biforcazione pitchfork} che verrà descritta, almeno brevemente, nelle
 sezioni successive.
\newline
I moti convettivi rappresentati dagli equilibri \textbf{E2,3} contribuiscono al mescolamento
 del fluido e a una maggiore trasimissione del calore tra le due superfici.
Calcolando il flusso di calore trasmesso attraverso le superfici che delimitano
 il dominio in $z=0$ e $z=1$, si può verificare che la convezione è un fenomeno fisico
 più efficiente per la trasmissione del calore nei fluidi rispetto alla conduzione, come mostrato in figura \ref{fig:lorenz-equil-heat-flux}.
\begin{figure}[t]
  \centering
  \begin{tabular}{cc}
  \begin{overpic}[width=0.45\textwidth, trim={0 0 30 0}, clip]{./fig/eq1_rho+24p7_heat_flux}
  \put(5,85){(a) \textbf{E1}}
  \end{overpic} \hspace{1.0cm} 
  \begin{overpic}[width=0.45\textwidth, trim={0 0 30 0}, clip]{./fig/eq2_rho+24p7_heat_flux_manyl}
  \put(5,85){(b) \textbf{E2}}
  \put(24,58){\footnotesize $x_i\!\!=\!\!0$}
  \put(25,53){/}
  \put(8,45){\footnotesize $x_i\!\!=\!\!k/2$}
  \put(20,40){\textbackslash}
  \put(0,25){\footnotesize $x_i\!\!=\!\!k/4$}
  \put(15,27){\small \_\_}
% \put(10,55){\footnotesize $x_i=0$}
  \end{overpic} 
  \end{tabular}
    \caption{Distribuzione della tempreatura nel dominio $T(x,z)$, andamento della temperatura $T(x_i,z)$ in funzione della coordinata $z$, flusso di calore $q = - k \, \partial T / \partial z(x,1)$ per conduzione sulla superficie superiore, normalizzato sul valore del flusso dell'equilibrio ``statico'' \textbf{E1}: (a) equilibrio instabile \textbf{E1}, (b) equilibrio marginalmente stabile \textbf{E2} per $\rho = 24.7368$.}
\label{fig:lorenz-equil-heat-flux}
\end{figure}
 Per esempio, il problema della trasmissione del calore tra due superfici parallele
 separate da un fluido si trova nella costruzione di infissi con \textbf{doppi vetri}:
 lo scopo dei doppi vetri separati da una sottile intercapedine d'aria è quello di
 sfruttare l'aria (ferma!) come ottimo isolante termico. L'intercapedine tra i due
 vetri deve essere sufficiente piccola da impedire la nascita dei moti convettivi,
 che ridurrebbero l'efficienza dell'infisso (e l'efficienza energetica della casa).
Le soluzioni convettive nel problema di Lorenz nascono quando il parametro $\rho$ 
 supera il valore critico $\rho_{cr} = 1$. Il parametro $\rho$ è proporzionale 
 al numero di Rayleigh,
\begin{equation}
 \rho \sim Ra = \frac{\alpha g \Delta T h^3}{\nu D} \ ,
\end{equation}
e quindi proporzionale al cubo della distanza tra le due superfici, $\rho \sim h^3$.
 Si ricava la stessa conclusione utilizzata nella costruzione dei doppi vetri: per
 distanze $h$ tra le due superfici limitate, il parametro $\rho$ è inferiore del
 valore critico e l'unica soluzione stabile esistente è quella isolante di fluido
 in quiete.

\vspace{0.3cm}
\noindent
Prima di studiare la stabilità locale ``alla Lyapunov'' dei punti di equilibrio, riprendendo
 la definizione di \textit{stabilità strutturale} si scopre che il sistema di Lorenz
 non è strutturalmente stabile a perturbazioni del valore di $\rho$, quando $\rho = 1$: infatti
 per $\rho < 1$ esiste un solo punto di equilibrio, per $\rho > 1$ esistono tre punti di equilibrio
 e di conseguenza le traiettorie nel piano delle fasi subiscono un cambiamento qualitativo.

%
\begin{figure}[h]
  \centering
  \begin{tabular}{cc}
  \begin{overpic}[width=0.40\textwidth]{./fig/bif_pitchfork}
  \put(-5,85){(a.1)}
  \put(-5,45){(a.2)}
  \end{overpic}  \hspace{0.8cm}
  \begin{overpic}[width=0.45\textwidth]{./fig/bif_hopf}
  \put(-5,75){(b)}
  \end{overpic}  \hspace{0.8cm}
  \end{tabular}
 \caption{Diagrammi di biforcazione. (a) Biforcazione pitchfrok per $\rho = 1$:
   (a.1) luogo delle radici dell'equilibrio \textbf{E1} e (a.2) diagramma di biforcazione. 
  Per $\rho = 1$ l'equilibrio \textbf{E1} diventa instabile e nascono i due equilibri stabili
  \textbf{E2,3}. (b) Biforcazione di Hopf degli equilibri \textbf{E2,2}:
  diagramma di biforcazione, rappresentato utilizzando  la forma normale.
  Per $\varepsilon = 0$, $\rho = 24.7368$, il ciclo limite instabile collassa sull'equilibrio
  stabile, che diventa instabile.}\label{fig:lorenz-eig}
\end{figure}
%
\paragraph{Stabilità dell'equilibrio E1}
Si studia la stabilità ``alla Lyapunov'' dell'equilibrio \textbf{E1}: $(0,0,0)$. Linearizzando
 il sistema non lineare di Lorenz attorno all'equilibrio \textbf{E1}, si ottiene il sistema
 linear(izzato)
\begin{equation}
 \begin{bmatrix} \delta \dot{x} \\ \delta \dot{y} \\ \delta \dot{z} \end{bmatrix} = 
 \begin{bmatrix}-\sigma & \sigma & 0 \\ \rho & -1 & 0 \\ 0 & 0 & - \beta \end{bmatrix}  
 \begin{bmatrix} \delta x \\ \delta y \\ \delta z \end{bmatrix} \quad , \quad
 \delta \dot{\bm{x}} = \bm{J}|_{\bm{E1}} \, \delta \bm{x}
\end{equation}
il cui polinomio caratteristico è
\begin{equation}
\begin{aligned}
 p(\lambda) = det(\bm{J}- \lambda \bm{I}) & = -(\beta+\lambda)[(\sigma+\lambda)(1+\lambda)-\sigma\rho] = \\ 
  & = -(\beta+\lambda) [ \lambda^2 + (\sigma+1)\lambda + \sigma(1-\rho)] \ . 
\end{aligned}
\end{equation}
Gli autovalori del sistema linearizzato attorno al primo equilibrio sono quindi
\begin{equation}
  \lambda^{E1}_1 = - \beta \quad , \quad  
  \lambda^{E1}_{2,3} = - \dfrac{\sigma+1}{2} \mp \dfrac{\sqrt{(\sigma+1)^2-4\sigma(1-\rho)}}{2} \ . 
\end{equation}
Per valori positivi dei parametri, tutti gli autovalori sono reali. Gli autovalori $\lambda^{E1}_1$
 e $\lambda^{E1}_2$ sono negativi per ogni valore di $\rho$, mentre l'autovalore $\lambda^{E1}_3$
 cambia segno per $\rho = 1$, come mostrato in figura \ref{fig:lorenz-eig}(a.1). L'analisi lineare di stabilità permette di concludere che l'equilibrio
 \textbf{E1} è linearmente stabile per $\rho<1$ e instabile per $\rho > 1$, mentre non
 permette di affermare nulla sul caso $\rho = 1$.
%
\paragraph{Stabilità degli equilibri E2, E3}
Per valori di $\rho \geq 1$ esistono i due equilibri \textbf{E2}, \textbf{E3}. Si studia
 la loro stabilità ``alla Lyapunov'' tramite lo studio degli autovalori del sistema linearizzato
 attorno ai punti di equilibrio,
\begin{equation}
 \begin{bmatrix} \delta \dot{x} \\ \delta \dot{y} \\ \delta \dot{z} \end{bmatrix} = 
 \begin{bmatrix}-\sigma & \sigma & 0 \\ 1 & -1 & \mp \sqrt{\beta(\rho-1)} \\
 \pm \sqrt{\beta(\rho-1)} & \pm \sqrt{\beta(\rho-1)} & - \beta \end{bmatrix}  
 \begin{bmatrix} \delta x \\ \delta y \\ \delta z \end{bmatrix} \quad , \quad
 \delta \dot{\bm{x}} = \bm{J}|_{\bm{E2,3}} \, \delta \bm{x} \ .
\end{equation}
Si può dimostrare (con il criterio di Routh-Hurwitz analiticamente, o calcolandone numericamente
 il valore) che i due punti di equilibrio sono stabili se
 $\rho < \frac{\sigma(\sigma+\beta+3)}{\sigma-1-\beta} \approx 24.7368$, utilizzando i valori
 $\sigma = 10$, $\beta = 8/3$.

\begin{figure}[t]
  \centering
  \begin{tabular}{cc}
  \begin{overpic}[width=0.65\textwidth, trim={0 -10 0 20}, clip]{./fig/ic_dependence_xt_rho+28}
  \put(0,35){(a)}
  \end{overpic} % \hspace{0.8cm}
  \begin{overpic}[width=0.35\textwidth, trim={60 80 40 40}, clip]{./fig/ic_dependence_attractor_rho+28}
  \put(0,60){(b)}
% \put(-5,45){(a.2)}
  \end{overpic}  \hfill
  \end{tabular}
\caption{Dinamica caotica del sistema di Lorenz per $\rho = 24.74$: evoluzione del sistema
 con condizioni iniziali $\bm{x}^{(1)}_0 = (-10,10,1)$, in blu, e $\bm{x}^{(2)}_0 = \bm{x}^{(1)}_0 +
 1.0\cdot 10^{-9}$, in arancione.
 (a) Evoluzione temporale
 della variabile $X(t)$: partendo da due condizioni iniziali ``vicine'', le due traiettorie
 del sistema si discostano in maniera ``non banale''. Il sistema dimostra un'evoluzione non 
 periodica, estremamente sensibile alle condizioni iniziali e quindi caotica.
 (b) Attrattore di Lorenz nello spazio delle fasi:
 le traiettorie nello spazio delle fasi rivelano la presenza di un attrattore, ``nelle
 vicinanze'' del quale si svolge la dinamica asintotica del sistema.}\label{fig:lorenz-chaos}
\end{figure}
\paragraph{Biforcazioni, cicli limite e attrattori strani}
L'analisi degli autovalori del sistema linearizzato attorno ai punti di equilibrio permette di 
 determinarne le caratteristiche locali quando gli autovalori hanno parte reale diversa
 da zero. In corrispondenza del cambio di stabilità di un punto di equilibrio e/o della 
 comparsa/scomparsa di punti di equilibrio (ma non solo!), le traiettorie nello spazio delle
 fasi del sistema subiscono un cambiamento qualitativo: il sistema non è strutturalmente stabile
 e si verifica una \textbf{biforcazione}.
\newline
Per studiare la stabità locale di un equilibrio in presenza di autovalori a parte reale nulla
 è necessario costruire un'approssimazione non lineare del sistema. Si considera un punto di equilibrio
 per il quale il sistema linearizzato non ha autovalori instabili, ha $N_s$ autovalori stabili e 
 $N_c$ autovalori a parte reale nulla e si vuole determinare l'evoluzione del sistema nelle
 vicinanze del punto di equilibrio.
Si può dimostrare che la dinamica del sistema $N=N_s+N_c$-dimensionale si riduce velocemente
 alla dinamica di un sistema $N_c$ dimensionale: le $N_s$ dinamiche asintoticamente stabili
 associate agli autovalori con parte reale negativa tendono asintoticamente ad annullarsi
 nell'intorno dell'equilibrio, mentre rimangono solo le dinamiche associate alle $N_c$ 
 dinamiche marginalmente stabili.
\newline
Si può usare un'espansione polinomiale per approssimare il sistema non lineare originale e
 costruire la \textbf{varietà centrale}, cioè la regione dello spazio delle fasi nella quale
 si svolgono le dinamiche marginalmente stabili.
\newline
\noindent 
Ad esempio, quando $\rho = 1$ il sistema di Lorenz nell'intorno dell'equilibrio \textbf{E1} (e
 dei nascenti equilibri \textbf{E2,3}) può essere ricondotto alla dinamica del sistema monodimensionale
\begin{equation}\label{eqn:lorenz:pitchfork}
 \dot{a}(t) = f(a(t)) = a(t) [ \alpha \varepsilon - \beta a(t)^2 ] \quad , \quad
 \text{con } \varepsilon := \rho-1 \ ,
\end{equation}
con $\alpha \approx 0.909$ e $\beta \approx 0.170$. Questo sistema coincide alla \textbf{forma
 normale} della biforcazione, cioè il sistema più semplice in grado di descrivere il cambiamento
 qualitativo del sistema. Lo studio della forma normale della biforcazione rivela l'esistenza
 di un unico equilibrio stabile $\overline{a}_1 = 0.0$ per $\epsilon \leq 0$, cioé $\rho \leq 1$.
 Per $\rho > 1$ l'equilibrio $\overline{a}_1$ diventa instabile e nascono due equilibri stabili
 $\overline{a}_{2,3} = \mp \sqrt{\alpha \varepsilon / \beta}$.
L'equazione (\ref{eqn:lorenz:pitchfork}) rappresenta la forma normale di una \textit{biforcazione
 pitchfork}. Poiché $\beta > 0$, la biforcazione si definisce \textit{supercritica}.
%
\begin{figure}[h]
  \centering
  \begin{tabular}{cc}
  \begin{overpic}[width=0.45\textwidth, trim={60 40 60 0}, clip]{./fig/lorenz_cm001}
  \put(0,75){(a)}
  \end{overpic} \hfill 
  \begin{overpic}[width=0.45\textwidth, trim={60 40 60 0}, clip]{./fig/lorenz_cm002}
  \put(0,75){(b)}
  \end{overpic}  \\
  \begin{overpic}[width=0.45\textwidth, trim={60 40 60 0}, clip]{./fig/lorenz_cm003}
  \put(0,75){(c)}
  \end{overpic} \hfill 
  \begin{overpic}[width=0.45\textwidth, trim={60 40 60 0}, clip]{./fig/lorenz_cm004}
  \put(0,75){(d)}
  \end{overpic}  \\
  \begin{overpic}[width=0.45\textwidth, trim={60 40 60 0}, clip]{./fig/lorenz_cm005}
  \put(0,75){(e)}
  \end{overpic} \hfill 
  \begin{overpic}[width=0.45\textwidth, trim={60 40 60 0}, clip]{./fig/lorenz_cm006}
  \put(0,75){(f)}
  \end{overpic}  \\
  \end{tabular}
\caption{Evoluzione del sistema di Lorenz per $\rho = 24.7368$ nel piano delle fasi.
Dinamica caotica, equilibri (\textbf{E1} in azzurro, \textbf{E2} in arancione e
 \textbf{E3} in verde) e varietà centrali dei due equilibri marginalmente stabili
 \textbf{E2,3}. La dinamica asintotica del sistema caotica del sistema alterna in
 maniera irregolare delle oscillazioni attorno ai due equilibri instabili sulle 
 ``nelle vicinanze'' delle rispettive varietà centrali.}
\label{fig:lorenz-chaos-cm}
\end{figure} 
%
\newline
Analogamente, quando $\rho \approx 24.7368$ i due equilibri \textbf{E2,3} cambiano stabilità: una
 coppia di autovalori complessi coniugati attraversa l'asse immaginario e la loro parte
 reale diventa positiva. Questo tipo di instabilità strutturale viene definita
 \textit{biforcazione di Hopf}: cambia la stabilità del punto di equilibrio considerato 
 e nasce/sparisce un ciclo limite nel suo intorno (il ciclo limite nasce da o si riduce al punto di equilibrio).
L'approssimazione sulla varietà centrale del sistema attorno a uno dei due equilibri conduce 
 al sistema di equazioni
\begin{equation}\label{eqn:lorenz:hopf}
\begin{cases}
 \dot{r}(t) = \alpha_r r \varepsilon - \beta_r r^3 \\
 \dot{\theta}(t) = \omega + \alpha_i \varepsilon + \beta_i r^2 \ 
\end{cases} \quad , \quad 
 \text{con } \varepsilon := \rho-24.7368 \ ,
\end{equation}
dove è stata utilizzata la rappresentazione polare complessa $a(t) = r(t) e^{i \theta(t)}$
 della variabile $a(t)$ che descrive la dinamica del sistema ridotta alla varietà centrale.
 Il parametro $\omega = 9.6245$ coincide con la parte immaginaria degli autovalori marginalmente
 stabili e gli altri parametri valgono:
\begin{equation}
\begin{aligned}
 & \alpha_r = 0.0302 \qquad , \qquad \beta_r =-0.003 \\
 & \alpha_i = 0.1815 \qquad , \qquad \beta_i =-0.028 \ . 
\end{aligned}
\end{equation}
La prima equazione delle (\ref{eqn:lorenz:hopf}) è identica all'equazione che descrive la 
 biforcazione pitchfork. In questo caso, però, il coefficiente $\beta_r$ è minore di zero. Questo
 tipo di biforcazione si definisce \textit{subcritica}. Si può facilmente dimostrare che
 per $\varepsilon < 0$ esistono due (il raggio $r$ di una rappresentazione polare deve
 essere $\geq 0$) equilibri
\begin{equation}
 \overline{\rho}_1 = 0 \quad , \quad \overline{\rho}_2 = \sqrt{-\alpha_r \varepsilon / \beta_r} \ .
\end{equation}
Il primo equilibrio dell'equazione in $r$ corrisponde a un punto fisso, poichè il raggio è nullo.
 Il secondo equilibrio corrisponde al raggio $\overline{\rho}_2$ del ciclo limite
 esistente per $\varepsilon < 0$.
Si dimostra quindi che un ciclo limite instabile coesiste con ognuno dei due punti di
 equilibri stabili \textbf{E2,3} per $\varepsilon < 0$ (cioè $\rho < 24.7368$), almeno
 in un intervallo finito di valori di $\rho$.
Quando $\varepsilon = 0$ (cioè $\rho < 24.7368$), il ciclo limite instabile si riduce al punto
 di equilibrio. Per $\varepsilon > 0$ il punto di equilibrio diventa instabile, mentre
 scompare il ciclo limite.
% \clearpage

\clearpage
\vspace{0.5cm}
\noindent
 Rimangono aperte alcune questioni: è possibile descrivere i cicli limite esistenti per (alcuni)
 valori del parametro $\rho < 24.7368$? Qual è l'evoluzione del sistema per valori
 di $\rho > 24.7368$? Ha senso utilizzare il modello di Lorenz, un brutale troncamento di
 un sistema continuo che dà origine a un sistema tridimensionale, per descrivere l'evoluzione
 del sistema fisico per valori crescenti del numero di Rayleigh $Ra$, e quindi del
 parametro $\rho$?

\vspace{0.3cm}
\noindent
Partendo dall'espressione approssimata del ciclo limite ottenuta dalla forma normale della
 biforcazione di Hopf per $\rho \lesssim 24.7368$ è possibile calcolare la forma del ciclo limite
 per valori inferiori del parametro, tramite tecniche di \textbf{continuazione}: negli algoritmi
 di continuazione la soluzione di un problema, nota per un valore del parametro, viene utilizzata
 per stimare la guess iniziale dello stesso problema per un valore diverso del parametro. 
In particolare, per identificare la traiettoria periodica corrispondente a un ciclo limite
 si può utilizzare una tecnica di \textbf{bilanciamento armonico}: la traiettoria periodica
 viene scritta come serie di Fourier, della quale è necessario determinare i coefficienti.

 
\vspace{0.3cm}
\noindent
Per valori di $\rho > 24.7368$ non esistono punti di equilibrio stabili punti di equilibrio
 stabili e non esistono cicli limite stabili. La dinamica del sistema rimane confinata 
 in una regione limitata dello spazio delle fasi, senza divergere.
L'evoluzione del sistema rappresentata in figura \ref{fig:lorenz-chaos} dimostra l'elevata
 sensibilità della soluzione alle condizioni 
 iniziali e l'assenza di equilibri o dinamiche periodiche stabili, caratteristici di un
 \textbf{regime caotico}. L'evoluzione di lungo tempo del sistema avviene ``nelle vicinanze''
 dell'attrattore di Lorenz, del quale si può intuire la forma grazie alle traiettorie
 rappresentate in figura \ref{fig:lorenz-chaos}(b).
La figura \ref{fig:lorenz-chaos-cm} rappresenta la traiettoria del sistema di Lorenz e le
 \textit{varietà centrali} dei due equilibri $\textbf{E2,3}$ marginalmente stabili 
 per $\rho = 24.7368$. Qualitativamente, lo stato del sistema viene attratto su queste
 superfici, lungo le direzioni stabili. Su queste superfici poi, si può osservare la dinamica
 marginalmente stabile (di dimensione ridotta: per il sistema di Lorenz, di dimensione 2, invece
 della dimensione 3 del sistema completo) del sistema: lo stato del sistema inizialmente
 oscilla attorno all' equilibrio \textbf{E2} (ad esempio),
 prima di essere attratta in maniera ``difficilmente prevedibile''
 dalla varietà centrale dell'equilibrio \textbf{E3} e iniziare ad oscillare attorno a 
 quest'ultimo equilibrio.

\vspace{0.3cm}
\noindent
L'approssimazione di Lorenz di dimensioni ridotte del sistema fisico continuo (e quindi
 di dimensione infinita) perde significato all'aumentare del numero di Rayliegh: all'aumentare
 del numero di Rayleigh infatti si attivano delle dinamiche più complesse, di dimensione
 maggiore, non descrivibili a un sistema tridimensionale.











\subsection{\dots}
