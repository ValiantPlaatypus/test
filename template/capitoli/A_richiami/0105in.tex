
\section{Rappresentazione del termine di derivata temporale nelle equazioni di bilancio}
Nella forma integrale delle equazioni di bilancio per un volume materiale $V(t)$ che descrivono la dinamica di un mezzo continuo compare la derivata temporale dell'integrale di volume della quantità fisica di interesse,
\begin{equation}
 \dfrac{d}{d t} \int_{V(t)} f .
\end{equation}
Il volume materiale $V(t)$ si muove con la velocità del mezzo continuo e il teorema di Reynolds \ref{thm:reynolds} permette di rielaborare questo termine ed esprimere questa derivata temporale in termini di integrali svolti su un volume di controllo $V_c$ fisso nello spazio.
%
Tramite il teorema di Reynolds \ref{thm:reynolds} lo stesso termine può essere scritto in termini di integrali svolti su un volume generico $v(t)$ in moto arbitario, che risulta solidale nel sistema di coordinate $\bm{\chi}$. Il cambio di coordinate preso in considerazione ora è $\bm{x}(\bm{\chi},t)$, dove $\bm{\chi}$ può essere pensato come il sistema di ``etichettatura'' applicato ai punti del volume $v(t)$. Rispetto al sistema di riferimento fisso $\bm{x}$, le particelle di tale
 volume sono in moto con velocità 
\begin{equation}
  \bm{w} = \dfrac{\partial}{\partial t}\bigg|_{\bm{\chi}} \bm{x}(\bm{\chi},t) .
\end{equation}
Il volume $v(t)$ risulta quindi definito dalla sua condizione iniziale e dalla velocità $\bm{w}$ dei punti della sua superficie.
\noindent
La derivata sul volume mobile $v(t)$ è
\begin{equation}
\begin{aligned}
  \frac{d}{dt} \int_{v(t)} f dv  & = & \text{(thm Reynolds per $v(t)$)} \\
                                 & = \int_{v(t)} \frac{\partial }{\partial t}\bigg|_{\bm{x}} f dv + \oint_{\partial v(t)} f\bm{w} \cdot \bm{\hat{n}} ds &
                                        \text{($+\oint_{\partial v(t)} f\bm{v} \cdot \bm{\hat{n}} -\oint_{\partial v(t)} f\bm{v} \cdot \bm{\hat{n}} $)} \\
                                 & = \int_{v(t)} \frac{\partial }{\partial t}\bigg|_{\bm{x}} f dv + \oint_{\partial v(t)} f\bm{v} \cdot \bm{\hat{n}} ds 
                                      - \oint_{\partial v(t)} f(\bm{v}-\bm{w}) \cdot \bm{\hat{n}} ds  & \text{(thm di Reynolds per $V(t)$)}\\
                                 & = \frac{d}{dt} \int_{V(t)\equiv v(t)} f dV - \oint_{\partial v(t)} f(\bm{v}-\bm{w}) \cdot \bm{\hat{n}} ds , \\
\end{aligned}
\end{equation}
dove è stata usata la formula ricavata in precedenza per $\frac{d}{dt} \int_{V(t)} f dV$.

\noindent
Quanto scritto vale per qualsiasi volume $v(t)$ in moto generico con velocità $\bm{w}$ della superficie $\partial v(t)$
 a qualsiasi istante di tempo. Se si considerano il volume di controllo $V_c$
 (fisso, $\bm{w}=0$) e il volume generico $v(t)$ coincidenti al tempo $t$ con
 il volume materiale $V(t)$ (in moto con la stessa velocità $\bm{w}=\bm{v}$
 del continuo), si ricava la regole per trasformare la derivata temporale su un dominio materiale $V(t)$ usata nell'approccio lagrangiano, in termini di integrali svolti sul volume di controllo $V_c$ e sul suo contorno $S_c$, impiegati nella descrizione euleriana del problema, come caso particolare della descrizione euleriana di un volume $v(t)$, 
\begin{equation}
\begin{aligned}
 \frac{d}{dt} \int_{V(t)} f  & =  & \text{(Lagrange)} \\
    & = \frac{d}{dt} \int_{V_c \equiv V(t)} f  + \oint_{\partial V_c \equiv \partial V(t)} f\bm{v} \cdot \bm{\hat{n}}            & \text{(Eulero)} \\
    & = \frac{d}{dt} \int_{v(t)\equiv V(t)} f  + \oint_{\partial v(t)\equiv \partial V(t)} f(\bm{v}-\bm{w}) \cdot \bm{\hat{n}}   & \text{(``ALE'')} 
\end{aligned}
\end{equation}
%\frac{d}{dt} \int_{V(t)} f dV & =  & \text{(Lagrange)} \\
%    & = \frac{d}{dt} \int_{V_c \equiv V(t)} f dV + \oint_{\partial V_c \equiv \partial V(t)} f\bm{v} \cdot \bm{\hat{n}} dS  & \text{(Eulero)} \\
%    & = \frac{d}{dt} \int_{v(t)\equiv V(t)} f dv + \oint_{\partial v(t)\equiv \partial V(t)} f(\bm{v}-\bm{w}) \cdot \bm{\hat{n}} ds  & \text{(``ALE'')} 


\noindent
Si sono ottenute le tre forme di rappresentazione più comuni per la derivata di integrali su volumi mobili per problemi di meccanica del continuo.
Quando vengono scritte le equazioni di bilancio (ad esempio massa, quantità di moto, energia \dots) in forma integrale, queste possono essere scritte
 per
\begin{itemize}
 \item un volume materiale, le cui particelle sono in moto con la stessa velocità del continuo (sia esso un fluido, un solido elastico o no, \dots):
   se la velocità $\bm{v}$ dei punti volume $V(t)$ coincide con quella di punti del continuo, il volume $V(t)$ è un volume materiale.
 \item un volume di controllo, fisso: $V_c$
 \item un volume in moto generico: usiamo il volume $v(t)$ i cui punti si muovono con velocità $\bm{w}$ generica.
\end{itemize}

Quando le equazioni di bilancio vengono scritte riferite ai tre volumi elencati sopra, si definiscono i tre tipi di rappresentazione usati in meccanica
 del continuo (di nuovo, sia esso un fluido, un solido, \dots)

\begin{itemize}
 \item rappresentazione \textbf{lagrangiana}: si segue l'evoluzione del continuo seguendo la traiettoria di ogni particella. Il volume mobile nel sistema
       di riferimneto ``fisso'', si muove con la stessa velocità delle particelle; questa descrizione viene usata frequentemente in meccanica dei solidi,
       quando gli spostamenti delle particelle sono sufficientemente piccoli e il determinante $J$ della trasformazione è non singolare (e spesso $\approx
       1$, es. piccole deformazioni).
 \item rappresentazione \textbf{euleriana}: si segue l'evoluzione del continuo sfruttando il volume di controllo $V_c$ solidale al sistema di 
       riferimento ``fisso'' e scrivendo le equazioni di bilancio considerando le variazioni dovute alla sola variazione in tempo e ai flussi attraverso
       la superficie del volume di controllo. Questa descrizione viene frequentemente usata in meccanica dei fluidi, dove spesso gli spostamenti della
       particelle sono ``molto grandi''. 
 \item rappresentazione \textbf{arbitraria} (ALE: arbitrary Lagrangian Eulerian): il volume $v(t)$ ha un moto ``generale'', indipendente da quello delle 
       particelle del continuo: la variazione della quantità è composta dalla sola variazione temporale e dal contributo dei flussi ``relativi'' 
       attraverso la superficie del volume di controllo (se $\bm{v}$ è la velocità delle particelle e $\bm{w}$ quella della superficie del volume $v(t)$
       rispetto allo stesso sistema di riferimento, $\bm{v} - \bm{w}$ è la velocità relativa del fluido rispetto al volume $v(t)$).
       Può essere spesso usata quando il dominio del problema varia nel tempo: si pensi a una simulazione numerica di una corrente attorno a un corpo
       rigido che subisce degli spostamenti o rotazioni (moderate) nel tempo o una corrente attorno a un'ala deformabile.
\end{itemize}

\noindent
Si noti come la rappresentazione ``ALE'' generalizza le altre due. Se il volume $v(t)$ coincide con il volume materiale $V(t)$ si ritorna alla descrizione
 lagrangiana: allora $\bm{w}=\bm{v}$ e non c'è flusso attraverso la superficie. Se il volume $v(t)$ coincide con il volume materiale $V_c$, allora $\bm{w}
 =0$ e si ottiene la rappresentazione euleriana.
 
 \newpage

\section{Bilanci integrali}
I bilanci integrali vengono scritti partendo da un volume materiale $V(t)$ qualsiasi, partendo dai principi della fisica classica
\begin{itemize}
  \item conservazione della massa;
  \item prima e seconda equazione cardinale della dinamica;
  \item bilancio di energia totale, somma di meccanica ed interna, che include i termini di flusso di calore, e che può essere
        ricondotto al primo principio della termodinamica.
\end{itemize}

\subsection{Bilancio di massa}
 La massa di un volume materiale $V(t)$ è costante nel tempo, poichè il volume materiale è costituito sempre dalle stesse particelle del
 continuo.
\begin{equation}
 \dfrac{d}{dt} \int_{V(t)} \rho = 0
\end{equation}

\subsection{Prima equazione cardinale e bilancio della quantità di moto}
La prima equazione cardinale della dinamica lega la quantità di moto $\bm{Q}$ di un sistema alla risultante delle forze esterne $\bm{R}^{ext}$
 agenti su di esso.
\begin{equation}
 \dfrac{d\bm{Q}}{d t} = \bm{R}^{ext}
\end{equation}
Separando i contributi di forze di volume e di superficie, il bilancio di quantità di moto per un volume materiale $V(t)$ arbitrario
\begin{equation}
 \dfrac{d}{d t} \int_{V(t)} \rho \bm{u} = \oint_{S(t)} \bm{t_n} + \int_{V(t)} \rho \bm{g}
\end{equation}
avendo indicato con $\bm{t_n}$ il vettore sforzo agente sulla superficie esterna $S(t)$ del volume materiale e $\bm{g}$ le forze per unità di
 massa (come ad esempio l'accelerazione di gravità).

\subsection{Seconda equazione cardinale e bilancio del momento della quantità di moto}
La seconda equazione cardinale della dinamica lega il momento della quantità di moto ${\Gamma}_H$ rispetto a un polo H con il momento risultante
 delle azioni esterne $\bm{M}^{ext}$ (e con il moto del polo H)
\begin{equation}
 \dfrac{d {\Gamma}_H}{dt} = -\dot{\bm{x}}_H \times \bm{Q} + \bm{M}^{ext}
\end{equation}
Se si considera un polo H fisso e si indica con $\bm{r}$ il raggio vettore dal polo H ai ``punti fisici'' del volume materiale,
 il bilancio integrale di momento angolare per un volume materiale (in assenza di coppie esterne) è
\begin{equation}
 \dfrac{d}{d t} \int_{V(t)} \bm{r} \times \rho \bm{u} = \oint_{S(t)} \bm{r} \times \bm{t_n} + \int_{V(t)} \rho \bm{r} \times \bm{g}
\end{equation}

\subsection{Bilancio di energia totale}
L'energia totale di un sistema è la somma della sua energia interna e cinetica. La mariazione di energia totale è dovuta al lavoro delle forze 
 agenti sul sistema e ai flussi di calore attraverso la superficie del volume (in assenza di fonti di calori interne al volume). 
\begin{equation}
 \dfrac{ d E^{tot}}{d t} = L - Q
\end{equation}
dove con $L$ si è indicato il lavoro svolto sul sistema e con $Q$ i flussi di calore uscenti da esso. Per un volume materiale
\begin{equation}
 \dfrac{d}{d t} \int_{V(t)} \rho e^{tot} = \oint_{S(t)} \bm{u} \cdot \bm{t_n} + \int_{V(t)} \rho \bm{u} \cdot \bm{g}
  - \oint_{S(t)} \bm{q} \cdot \bm{\hat{n}}
\end{equation}
dove $\bm{q} \cdot \bm{\hat{n}}$ positivo indica un flusso di calore uscente. L'energia totale per unità di massa può essere scritta come
somma del contributo interno e del contributo cinetico
\begin{equation}
 e^t = e + \dfrac{1}{2}|\bm{u}|^2
\end{equation}
Introducendo la definizione di entalpia $h = e + Pv = e + P/\rho$ ci si può ricondurre a molti casi analizzati durante il corso di 
 Fisica Tecnica, partendo ora da unquadro generale sui bilanci integrali: partendo dai bilanci generali, si possono introdurre
 le ipotesi di sistema chiuso, adiabatico o isolato, annullando i termini di flusso di massa, di flusso di calore o i termini di 
 energia e calore. Il bilancio di energia per un volume di controllo $V_c$ fisso (vedi sezione successiva), dopo aver scritto
 il termine di sforzo separando il contributo di pressione da quello di sforzi viscosi $\bm{t_n} = -p\bm{\hat{n}} + \bm{s_n}$, diventa
\begin{equation}
\begin{aligned}
\dfrac{d}{d t} \int_{V_c} \rho e^t & = - \oint_{S_c} \rho e^t \bm{u} \cdot \bm{\hat{n}}
  - \oint_{S_c} P \bm{u} \cdot \bm{\hat{n}} + \oint_{S_c} \bm{u} \cdot \bm{s_n} + \int_{V_c} \rho \bm{u} \cdot \bm{g}
  - \oint_{S_c} \bm{q} \cdot \bm{\hat{n}} = & (\rho h^t = \rho(e^t + P/\rho))\\
 & = - \oint_{S_c} \rho h^t \bm{u} \cdot \bm{\hat{n}}
   + \oint_{S_c} \bm{u} \cdot \bm{s_n} + \int_{V_c} \rho \bm{u} \cdot \bm{g}
  - \oint_{S_c} \bm{q} \cdot \bm{\hat{n}}
\end{aligned}
\end{equation}
avendo messo in evidenza il flusso di entalpia totale $h^t$.

\subsection{Bilanci integrali per volumi in moto arbitrario}
I bilanci integrali per un volume $v(t)$ in moto generico con velocità
 $\bm{w}$ possono essere ricavati partendo da quelli per un volume $V(t)$
 materiale, ricavati nella
 sezione precedente, con l'utilizzo del teorema del trasporto di Reynolds per
 modificare il termine di derivata temporale. Per un volume $v(t)$, la cui
 superficie $\partial s(t)$ ha velocità $\bm{w}$

\begin{equation}\label{eqn:bilanciIntegrali:ale}
 \begin{aligned}
 & \dfrac{d}{d t} \int_{v(t)} \rho + \oint_{s(t)} \rho (\bm{u} - \bm{w}) \cdot \bm{\hat{n}} = 0  \\
 & \dfrac{d}{d t} \int_{v(t)} \rho \bm{u} + \oint_{s(t)} \rho \bm{u} (\bm{u} - \bm{w}) \cdot \bm{\hat{n}} =
     \oint_{s(t)} \bm{t_n} + \int_{v(t)} \rho \bm{g} \\
 & \dfrac{d}{d t} \int_{v(t)} \bm{r} \times \rho \bm{u} + \oint_{s(t)} \bm{r} \times (\rho \bm{u}) (\bm{u} - \bm{w}) \cdot \bm{\hat{n}} = 
    \oint_{s(t)} \bm{r} \times \bm{t_n} + \int_{v(t)} \rho \bm{r} \times \bm{g} \\
 & \dfrac{d}{d t} \int_{v(t)} \rho e^{tot} + \oint_{s(t)} \rho e^t (\bm{u} - \bm{w}) \cdot \bm{\hat{n}}
  = \oint_{s(t)} \bm{u} \cdot \bm{t_n} + \int_{v(t)} \rho \bm{u} \cdot \bm{g}
  - \oint_{s(t)} \bm{q} \cdot \bm{\hat{n}}
 \end{aligned}
\end{equation}

\noindent
Per un volume di controllo $V_c$ fisso, $\bm{w}=0$
\begin{equation}\label{eqn:bilanciIntegrali:eulerian}
 \begin{aligned}
 & \dfrac{d}{d t} \int_{V_c} \rho + \oint_{S_c} \rho \bm{u} \cdot \bm{\hat{n}} = 0  \\
 & \dfrac{d}{d t} \int_{V_c} \rho \bm{u} + \oint_{S_c} \rho \bm{u} \bm{u} \cdot \bm{\hat{n}} =
     \oint_{S_c} \bm{t_n} + \int_{V_c} \rho \bm{g} \\
 & \dfrac{d}{d t} \int_{V_c} \bm{r} \times \rho \bm{u} + \oint_{S_c} \bm{r} \times (\rho \bm{u}) \bm{u} \cdot \bm{\hat{n}} = 
    \oint_{S_c} \bm{r} \times \bm{t_n} + \int_{V_c} \rho \bm{r} \times \bm{g} \\
 & \dfrac{d}{d t} \int_{V_c} \rho e^t + \oint_{S_c} \rho e^t \bm{u} \cdot \bm{\hat{n}}
  = \oint_{S_c} \bm{u} \cdot \bm{t_n} + \int_{V_c} \rho \bm{u} \cdot \bm{g}
  - \oint_{S_c} \bm{q} \cdot \bm{\hat{n}}
 \end{aligned}
\end{equation}


%\subsection{Operatori differenziali in diversi sistemi di coordinate}

%Le equazioni della fluidodinamica (e non solo) sono relazioni vettoriali (o più in generale
% hanno carattere tensoriale): le equazioni sono indipendenti dal sistema di riferimento nel
% quale vengono scritte.

% La scrittura delle equazioni in un sistema di coordinate (cartesiano ortogonale, polare,
% cilindrico, sferico) è solo un metodo per rappresentare le equazioni. La scelta del sistema
% di riferimento 'più opportuno' è spesso dettato dalla geometria del problema: ad esempio, un
% fluido in un tubo cilindrico verrà descritto comodamente scrivendo le equazioni in coordinate
% cilindriche, invece di utilizzare le coordinate cartesiane. \textit{Il mondo non è fatto
% a quadretti.}

% Nelle equazioni della fluidodinamica (e non solo) compaiono operatori differenziali, i quali
% devono essere opportunamente espressi nelle coordinate nelle quali viene descritto il problema.
% Capita la differenza tra carattere tensoriale delle equazioni e la loro scrittura in un sistema
% di riferimento 'comodo', viene rivacata la forma in componenti cartesiane e polari degli operatori
% differenziali, limitandoci per semplicità al caso bidimensionale.

% 


%\subsubsection*{Cambiamento di coordinate tra coordinate cartesiane e polari}

%Trasformzaione di coordinate.
%\begin{equation}
%  \begin{cases}
%    x = r \cos \theta \\
%    y = r \sin \theta
%  \end{cases}
%  \qquad
%  \begin{cases}
%    r = \sqrt{x^2 + y^2} \\
%    \tan \theta = \frac{y}{x}
%  \end{cases}
%\end{equation}

%Derivate.
%\begin{equation}
% \begin{aligned}
%  \frac{\partial}{\partial x} =\frac{\partial r}{\partial x}\frac{\partial}{\partial r} +
%                 \frac{\partial \theta}{\partial x}\frac{\partial}{\partial \theta} = 
%            \frac{x}{\sqrt{x^2+y^2}}\frac{\partial}{\partial r} 
%              - \frac{y}{x^2+y^2}\frac{\partial}{\partial \theta} =
%            \cos \theta \frac{\partial}{\partial r} 
%              - \frac{\sin \theta}{r} \frac{\partial}{\partial \theta} \\ 
%  \frac{\partial}{\partial y} =\frac{\partial r}{\partial y}\frac{\partial}{\partial r} +
%                 \frac{\partial \theta}{\partial y}\frac{\partial}{\partial \theta} = 
%            \frac{y}{\sqrt{x^2+y^2}}\frac{\partial}{\partial r} 
%              + \frac{x}{x^2+y^2}\frac{\partial}{\partial \theta} =
%            \sin \theta \frac{\partial}{\partial r} 
%              + \frac{\cos \theta}{r} \frac{\partial}{\partial \theta} \\           
% \end{aligned}
%\end{equation}

%Versori.
%\begin{equation}
%  \begin{cases}
%    \bm{\hat{x}} = \cos \theta \bm{\hat{r}} - \sin \theta \bm{\hat{\theta}} \\
%    \bm{\hat{y}} = \sin \theta \bm{\hat{r}} + \cos \theta \bm{\hat{\theta}} \\
%  \end{cases}
%  \qquad
%  \begin{cases}
%    \bm{\hat{r}}      =   \cos \theta \bm{\hat{x}} + \sin \theta \bm{\hat{y}} \\
%    \bm{\hat{\theta}} = - \sin \theta \bm{\hat{x}} + \cos \theta \bm{\hat{y}} \\
%  \end{cases}
%\end{equation}

%Componenti dei vettori.
%\begin{equation}
% \bm{u} = u   \bm{\hat{x}} + v \bm{\hat{y}}
%        = u_r \bm{\hat{r}} + u_\theta \bm{\hat{\theta}}
%\end{equation}

%Cambio di coordinate: relazione tra le componenti.
%\begin{equation}
%  \begin{cases}
%    u = u_r \cos \theta - u_\theta \sin \theta  \\
%    v = u_r \sin \theta + u_\theta \cos \theta  \\
%  \end{cases}
%  \qquad
%  \begin{cases}
%    u_r      =   u \cos \theta  + v \sin \theta \\
%    u_\theta = - u \sin \theta  + v \cos \theta \\
%  \end{cases}
%\end{equation}

%\subsubsection*{Gradiente in coordinate cartesiane}

%\begin{equation}
% \bm{\nabla}f = \frac{\partial f}{\partial x} \bm{\hat{x}} +
%                \frac{\partial f}{\partial y} \bm{\hat{y}}
%\end{equation}

%\subsubsection*{Gradiente in coordinate polari}

%\begin{equation}
% \bm{\nabla}f = \frac{\partial f}{\partial r} \bm{\hat{r}} +
%                \frac{1}{r}\frac{\partial f}{\partial \theta} \bm{\hat{\theta}}
%\end{equation}

%\begin{equation}
%\begin{aligned}
%  \bm{\nabla}f & = \frac{\partial f}{\partial x} \bm{\hat{x}} +
%                \frac{\partial f}{\partial y} \bm{\hat{y}} = \\
%   & =     \displaystyle \left[ \cos \theta \frac{\partial f}{\partial r} 
%              - \frac{\sin \theta}{r} \frac{\partial f}{\partial \theta} \right] 
%              ( \cos \theta \bm{\hat{r}} - \sin \theta \bm{\hat{\theta}} ) + 
%     \displaystyle \left[ \sin \theta \frac{\partial f}{\partial r} 
%              + \frac{\cos \theta}{r} \frac{\partial f}{\partial \theta} \right] 
%              ( \sin \theta \bm{\hat{r}} + \cos \theta \bm{\hat{\theta}} )  \\
%   & =  \frac{\partial f}{\partial r} \bm{\hat{r}} +
%                \frac{1}{r}\frac{\partial f}{\partial \theta} \bm{\hat{\theta}}
%\end{aligned}
%\end{equation}

%\subsubsection*{Divergenza in coordinate cartesiane}

%\begin{equation}
%\bm{\nabla} \cdot \bm{u} = \bm{\nabla} \cdot (u   \bm{\hat{x}} + v \bm{\hat{y}}) = 
% \frac{\partial u}{\partial x} + \frac{\partial v}{\partial y}
%\end{equation}

%\subsubsection*{Divergenza in coordinate polari}

%\begin{equation}
%\bm{\nabla} \cdot \bm{u} = \bm{\nabla} \cdot (u_r \bm{\hat{r}} + u_\theta \bm{\hat{\theta}})
%\end{equation}

%Svolgendo i conti per la derivata $\frac{\partial u}{\partial x}$:
%\begin{equation}
%\begin{aligned}
% \frac{\partial u}{\partial x} 
%& = \frac{\partial r}{\partial x}\frac{\partial u}{\partial r} +
%    \frac{\partial \theta}{\partial x}\frac{\partial u}{\partial \theta} = \\
%& = \cos \theta \frac{\partial}{\partial r} (u_r \cos \theta - u_\theta \sin \theta) - 
%    \frac{\sin \theta}{r} \frac{\partial }{\partial \theta} (u_r \cos \theta - u_\theta \sin \theta)  \\
%& = \cos^2 \theta \frac{\partial u_r}{\partial r}
%     - \sin \theta \cos \theta \frac{\partial u_\theta}{\partial r}
%     - \frac{\sin \theta \cos \theta}{r} \frac{\partial u_r}{\partial \theta}
%     + \frac{\sin^2 \theta}{r} u_r
%     + \frac{\sin^2 \theta}{r} \frac{\partial u_\theta}{\partial \theta}
%     + \frac{\sin \theta \cos \theta}{r} u_\theta \\
%\end{aligned}
%\end{equation}

%Si calcolano in maniera analoga $\frac{\partial v}{\partial y}$ e tutte le altre derivate.

%\begin{equation}
%\begin{aligned}
%  \frac{\partial u}{\partial x} = \cos^2 \theta \frac{\partial u_r}{\partial r}
%     - \sin \theta \cos \theta \frac{\partial u_\theta}{\partial r}
%     - \frac{\sin \theta \cos \theta}{r} \frac{\partial u_r}{\partial \theta}
%     + \frac{\sin^2 \theta}{r} u_r
%     + \frac{\sin^2 \theta}{r} \frac{\partial u_\theta}{\partial \theta}
%     + \frac{\sin \theta \cos \theta}{r} u_\theta \\
%  \frac{\partial v}{\partial y} = \sin^2 \theta \frac{\partial u_r}{\partial r}
%     + \sin \theta \cos \theta \frac{\partial u_\theta}{\partial r}
%     + \frac{\sin \theta \cos \theta}{r} \frac{\partial u_r}{\partial \theta}
%     + \frac{\cos^2 \theta}{r} u_r
%     + \frac{\cos^2 \theta}{r} \frac{\partial u_\theta}{\partial \theta}
%     - \frac{\sin \theta \cos \theta}{r} u_\theta
%\end{aligned}
%\end{equation}

%Sommando, si ottiene:

%\begin{equation}
%  \bm{\nabla} \cdot \bm{u} = 
% \frac{\partial u}{\partial x} + \frac{\partial v}{\partial y} =
%   \frac{\partial u_r}{\partial r} + \frac{u_r}{r} + \frac{1}{r}\frac{\partial u_\theta}{\partial \theta} = 
%   \frac{1}{r} \frac{\partial (r u_r)}{\partial r} + \frac{1}{r}\frac{\partial u_\theta}{\partial \theta}
%\end{equation}


%\subsubsection*{Rotore in coordinate cartesiane}

%\begin{equation}
%\bm{\nabla} \times \bm{u} = 
%\begin{vmatrix}
%\bm{\hat{x}} & \bm{\hat{y}} & \bm{\hat{z}} \\
%\frac{\partial }{\partial x} & \frac{\partial }{\partial y} &
%\frac{\partial }{\partial z} \\
%u & v & w
%\end{vmatrix}= 
%\begin{vmatrix}
%\bm{\hat{x}} & \bm{\hat{y}} & \bm{\hat{z}} \\
%\frac{\partial }{\partial x} & \frac{\partial }{\partial y} & 0 \\
%u & v & 0
%\end{vmatrix}
% = 	\bm{\hat{z}} \displaystyle\left[
%\frac{\partial v}{\partial x} - \frac{\partial u}{\partial y} \right]
%\end{equation}

%\subsubsection*{Rotore in coordinate polari}

%\begin{equation}
%\bm{\nabla} \times\bm{u} = \bm{\nabla}\times (u_r \bm{\hat{r}} + u_\theta \bm{\hat{\theta}})
%\end{equation}

%\begin{equation}
%\begin{aligned}
%  \frac{\partial v}{\partial x} = \sin \theta \cos \theta \frac{\partial u_r}{\partial r}
%     + \cos^2 \theta \frac{\partial u_\theta}{\partial r}
%     - \frac{\sin^2 \theta}{r} \frac{\partial u_r}{\partial \theta}
%     - \frac{\sin \theta \cos \theta}{r} u_r
%     - \frac{\sin \theta \cos \theta}{r} \frac{\partial u_\theta}{\partial \theta}
%     + \frac{\sin^2 \theta}{r} u_\theta \\
%  \frac{\partial u}{\partial y} = \sin \theta \cos \theta \frac{\partial u_r}{\partial r}
%     - \sin^2 \theta  \frac{\partial u_\theta}{\partial r}
%     + \frac{ \cos^2 \theta}{r} \frac{\partial u_r}{\partial \theta}
%     - \frac{\cos \theta \sin \theta}{r} u_r
%     - \frac{\sin \theta \cos \theta}{r} \frac{\partial u_\theta}{\partial \theta}
%     - \frac{\cos^2 \theta}{r} u_\theta
%\end{aligned}
%\end{equation}

%\begin{equation}
%  \bm{\hat{z}}\cdot\bm{\nabla} \times \bm{u} = \frac{\partial v}{\partial x} - \frac{\partial u}{\partial y} =
%   \frac{1}{r}  u_\theta + \frac{\partial u_\theta}{\partial r} - \frac{1}{r}\frac{\partial u_r}{\partial \theta} =
%   \frac{1}{r} \frac{\partial (r u_\theta)}{\partial r} - \frac{1}{r}\frac{\partial u_r}{\partial \theta}
%\end{equation}

%\vspace{0.5cm}
%\textit{Osservazioni}. Notare le somiglianze (e le differenze) tra divergenza e rotore in coordinate polari.
%\begin{equation}
%\begin{aligned}
% \bm{\nabla} \cdot \bm{u} = 
%   \frac{1}{r} \frac{\partial (r u_r)}{\partial r} + \frac{1}{r}\frac{\partial u_\theta}{\partial \theta}\\
%\bm{\hat{z}}\cdot\bm{\nabla} \times \bm{u} = 
%  \frac{1}{r} \frac{\partial (r u_\theta)}{\partial r} - \frac{1}{r}\frac{\partial u_r}{\partial \theta}
%\end{aligned}
%\end{equation}

