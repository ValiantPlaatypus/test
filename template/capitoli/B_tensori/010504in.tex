\subsection{Esempio: coordinate cilindriche}\label{ch:coordCyl}
Questo paragrafo viene dedicato alle coordinate cilindriche, un sistema di coordiante ortogonali. Vengono calcolate le espressioni delle coordinate degli operatori applicati a campi tensoriali che verranno incontrati nella scrittura delle equazioni di bilancio, come ad esempio le equazioni di Navier--Stokes per fluidi incomprimibili introdotte alla fine del paragrafo.

Lo spazio tridimensionale viene descritto in coordinate cilindriche dalle tre coordinate
 $(q^1,q^2,q^3) = (r,\theta,z)$. L'elemento di lunghezza $ds$ ha la forma
\begin{equation}
 ds^2 = g_{ij} dq^i dq^j = dr^2 + r^2 d\theta^2 + dz^2
\end{equation}

\subsubsection{Tensore metrico.}
Il sistema di coordinate cilindriche è ortogonale. Il tensore metrico è diagonale $g_{ij} = 0, \ i\ne j$.
In particolare, dall'elemento di lunghezza si ricavano le componenti del tensore metrico
\begin{equation}
\begin{aligned}
  g_{11} = 1 ,& \quad g_{22} = q^{(1)^2}     ,& \quad g_{33} = 1 & \qquad \text{(componenti covarianti)} \\
  g^{11} = 1 ,& \quad g^{22} = 1 / q^{(1)^2} ,& \quad g^{33} = 1 & \qquad \text{(componenti contravarianti)}
\end{aligned}
\end{equation}

\subsubsection{Vettore posizione e base naturale.}
Rispetto alla base cartesiana $(\bm{\hat{x}},\bm{\hat{y}},\bm{\hat{z}})$, il vettore posizione $\bm{x}$ è
\begin{equation}
 \bm{x} = q^1 \cos(q^2) \bm{\hat{x}} + q^1 \sin(q^2) \bm{\hat{y}} + q^3  \bm{\hat{z}}
\end{equation}
I vettori della base naturale sono definiti $\bm{b}_i = \frac{\partial \bm{x}}{\partial q^i}$. La base 
 reciproca si ottiene da $\bm{b}^i = g^{ik}\bm{b}_k$
\begin{equation}
 \begin{cases}
   \bm{b}_1 =      \cos q^2 \bm{\hat{x}} +     \sin q^2 \bm{\hat{y}} \\
   \bm{b}_2 = -q^1 \sin q^2 \bm{\hat{x}} + q^1 \cos q^2 \bm{\hat{y}} \\
   \bm{b}_3 = \bm{\hat{z}}
 \end{cases}
 \qquad
 \begin{cases}
   \bm{b}^1 = \bm{b}_1 \\ \bm{b}^2 = g^{22} \bm{b}_2 \\ \bm{b}^3 = \bm{b}_3 \\
 \end{cases}
\end{equation}

\subsubsection{Componenti contravarianti, covarianti e fisiche.}
 I vettori $\bm{b}_1$, $\bm{b}_3$, $\bm{b}^1$, $\bm{b}^3$ delle basi naturale e reciproca sono privi di dimensioni fisiche, mentre il vettore $\bm{b}_2$ ha la dimensione di una lunghezza (la coordinata $q^1$ coincide con il raggio $r$) e il vettore $\bm{b}^2$ ha la dimensione dell'inverso di 
 una lunghezza. \'E quindi necessario definire ula base fisica e le componenti fisiche di un vettore (in maniera analoga si definiranno le componenti fisiche di un tensore di ordine qualsiasi).
 Utilizzando il valore delle componenti del tensore metrico trovate e la (\ref{eqn:baseFisica}), la base fisica nel sistema di coordiante cilindriche è
 \begin{equation}
 \left\{
 \begin{aligned}
  \bm{\hat{r}}     & = \bm{\hat{b}}^1 && = \bm{b}^1                 && =  \cos q^2 \bm{\hat{x}} +     \sin q^2 \bm{\hat{y}} \\
  \bm{\hat{\theta}}& = \bm{\hat{b}}^2 && = \bm{b}^2/{\sqrt{g^{22}}} && = -\sin q^2 \bm{\hat{x}} +     \cos q^2 \bm{\hat{y}} \\
  \bm{\hat{z}}     & = \bm{\hat{b}}^3 && = \bm{b}^3                 && =  \bm{\hat{z}} .
 \end{aligned}
 \right.
 \end{equation}

\subsubsection{Simboli di Christoffel.}
 Si può verificare tramite calcolo diretto che gli unici simboli di Christoffel del secondo tipo diversi da zero sono
 \begin{equation}
  \Gamma_{21}^2 = \Gamma_{12}^2 = 1 / q^1  , \quad \Gamma_{22}^1 = -q^1 .
 \end{equation}
 
\subsubsection{Gradiente di uno scalare.}
 Utilizzando la forma in componenti del gradiente, il legame tra componenti contravarianti, covarianti e fisiche, si può
 scrivere il gradiente di un campo scalare in componenti contravarianti, covarianti e fisiche.
 \begin{equation}
 \begin{aligned}
  \text{grad} f & = \frac{\partial f}{\partial r} \bm{b}^1 + \frac{\partial f}{\partial \theta} \bm{b}^2 + \frac{\partial f}{\partial z} \bm{b}^3 = \\ 
      & = \frac{\partial f}{\partial r} \bm{b}_1 + \frac{1}{r^2}\frac{\partial f}{\partial \theta} \bm{b}_2 + \frac{\partial f}{\partial z} \bm{b}_3 = \\
      & = \frac{\partial f}{\partial r} \bm{\hat{r}} + \frac{1}{r}\frac{\partial f}{\partial \theta} \bm{\hat{\theta}}
         + \frac{\partial f}{\partial z} \bm{\hat{z}}
 \end{aligned}
 \end{equation}

\subsubsection{Divergenza di un vettore.}
 Si riporta qui l'espressione della divergenza di un campo vettoriale, già calcolata in \S\ref{ch:operatoriDiff}
 \begin{equation}
 \begin{aligned}
  \text{div} \bm{v} & = \frac{\partial v^i}{\partial q^i} + \Gamma_{il}^i v^l = \\
                    & = \frac{\partial v^1}{\partial q^1} + \frac{\partial v^2}{\partial q^2} + \frac{\partial v^3}{\partial q^3}
                       + \Gamma_{12}^2 v^1 = \\
                    & = \frac{\partial v_r}{\partial r} + \frac{1}{r}\frac{\partial v_\theta}{\partial \theta} + \frac{\partial v_z}{\partial z}
                       + \frac{1}{r} v_r \\
                    & = \frac{1}{r}\dfrac{\partial (r v_r)}{\partial r} + \frac{1}{r}\frac{\partial v_\theta}{\partial \theta} +
                        \frac{\partial v_z}{\partial z} .
 \end{aligned}
 \end{equation}
 
\subsubsection{Gradiente di un vettore.}
Si riporta qui l'espressione della gradiente di un campo vettoriale, già calcolata in \S\ref{ch:operatoriDiff}
\begin{equation}
\begin{aligned}
    \text{grad} \bm{v} & = \left[ \dfrac{\partial v^i}{\partial q^k} + \Gamma_{lk}^i v^l \right] \bm{b}_i \otimes \bm{b}^k = \\
     & = \frac{\partial v^1}{\partial q^1}                                   \bm{b}_1 \otimes \bm{b}^1   + 
        \left( \frac{\partial v^1}{\partial q^2} + \Gamma_{22}^1 v^2 \right) \bm{b}_1 \otimes \bm{b}^2   + 
        \frac{\partial v^1}{\partial q^3}                                    \bm{b}_1 \otimes \bm{b}^3   + \\
     & + \left( \frac{\partial v^2}{\partial q^1} + \Gamma_{21}^2 v^2 \right)\bm{b}_2 \otimes \bm{b}^1   + 
        \left( \frac{\partial v^2}{\partial q^2} + \Gamma_{12}^2 v^1 \right) \bm{b}_2 \otimes \bm{b}^2   + 
        \frac{\partial v^2}{\partial q^3}                                    \bm{b}_2 \otimes \bm{b}^3   + \\
     & + \frac{\partial v^3}{\partial q^1}                                   \bm{b}_3 \otimes \bm{b}^1   + 
        \frac{\partial v^3}{\partial q^2}                                    \bm{b}_3 \otimes \bm{b}^2   + 
        \frac{\partial v^3}{\partial q^3}                                    \bm{b}_3 \otimes \bm{b}^3   = \\
     & = \frac{\partial v_r}{\partial r}                                     \bm{\hat{r}} \otimes \bm{\hat{r}}   + 
        \frac{1}{r}\left( \frac{\partial v_r}{\partial \theta} - v_\theta \right) \bm{\hat{r}} \otimes \bm{\hat{\theta}}  + 
        \frac{\partial v_r}{\partial z}                                     \bm{\hat{r}} \otimes \bm{\hat{z}}  + \\
     & +  \frac{\partial v_\theta}{\partial r}                              \bm{\hat{\theta}} \otimes \bm{\hat{r}}  + 
        \frac{1}{r}\left( \frac{\partial v_\theta}{\partial \theta} +  v_r \right) \bm{\hat{\theta}} \otimes \bm{\hat{\theta}}  + 
        \frac{\partial v_\theta}{\partial z}                                \bm{\hat{\theta}} \otimes \bm{\hat{z}}  + \\
     & + \frac{\partial v_z}{\partial r}                                    \bm{\hat{z}} \otimes \bm{\hat{r}} + 
        \frac{1}{r}\frac{\partial v_z}{\partial \theta}                     \bm{\hat{z}} \otimes \bm{\hat{\theta}}  + 
        \frac{\partial v_z}{\partial z}                                     \bm{\hat{z}} \otimes \bm{\hat{z}}  ,
\end{aligned}
\end{equation}
 
 \subsubsection{Laplaciano.} 
 La forma in componenti dell'operatore laplaciano di un vettore si ottiene partendo dalla definizione
 $\Delta \bm{v} = \text{div(grad $\bm{v}$)}$. Il gradiente è stato appena scritto in componenti miste; per poter utilizzare la
 formula in componenti contravarianti di un tensore, è quindi necessario prima trasformare (tramite il tensore metrico) le componenti
 miste in contravarianti: viene scritto esplicitamente solo il termine $g^{22}$ poiché è l'unico elemento diagonale diverso da uno e
 viene eseguito il calcolo solo per la prima componente, per motivi di sintesi
 \begin{equation}
 \begin{aligned}
  \Delta \bm{v} = &\bm{b}_1 \left\{ \frac{\partial}{\partial q^1}\frac{\partial v^1}{\partial q^1} +
               \frac{\partial}{\partial q^2} \left[ g^{22}\left( \frac{\partial v^1}{\partial q^2} + \Gamma_{22}^1 v^2 \right) \right] +
               \frac{\partial}{\partial q^3}\frac{\partial v^1}{\partial q^3} + 
               g^{22}\Gamma_{22}^1  \left( \frac{\partial v^2}{\partial q^2} + \Gamma_{12}^2 v^1 \right) +
               \Gamma_{21}^2 \frac{\partial v^1}{\partial q^2} \right\} \\
               & + \bm{b}_2 \left\{ \dots \right\} + \bm{b}_3 \left\{ \dots \right\} = \\
               & = \dots  = \\
               & = \left[ \Delta u_r - \frac{u_r}{r^2} - \frac{2}{r^2} \frac{\partial u_\theta}{\partial \theta}\right]\bm{\hat{r}} + \\
               & \quad + \left[ \Delta u_\theta + \frac{2}{r^2} \frac{\partial u_r}{\partial \theta} - \frac{u_\theta}{r^2} \right]\bm{\hat{\theta}}+ \\
               & \quad + \left[ \Delta u_z  \right]\bm{\hat{z}}
 \end{aligned}
 \end{equation}
 dove con $\Delta f$ è stato indicato il laplaciano di uno scalare
 \begin{equation}
   \Delta f = \frac{1}{r}\frac{\partial}{\partial r}\left( r \frac{\partial f}{\partial r}\right) +
               \frac{1}{r^2} \frac{\partial^2 f}{\partial \theta^2} + \frac{\partial^2 f}{\partial z^2}
 \end{equation}
 
 \subsubsection{Termine advettivo.}
 \begin{equation}
  \begin{aligned}
   (\bm{a} \cdot \bm{\nabla}) \bm{v} = & \left[ \frac{\partial v^i}{\partial q^j} +\Gamma_{lj}^i v^l \right] a^j \bm{b}_i = \\
     & = \left[  a^k \frac{\partial v^1}{\partial q^k} + a^2 \Gamma_{22}^1 v^2 \right] \bm{b}_1 + 
       \left[  a^k \frac{\partial v^2}{\partial q^k} + a^1 \Gamma_{12}^1 v^2 + a^2 \Gamma_{21}^1 v^1 \right] \bm{b}_2 + 
       \left[  a^k \frac{\partial v^3}{\partial q^k} \right] \bm{b}_3 = \\
     & = \dots =  \\
     & = \left[ a_r \frac{\partial v_r}{\partial r} + \frac{a_\theta}{r}\frac{\partial v_r}{\partial \theta} +
                a_r \frac{\partial v_r}{\partial r} - \frac{a_\theta v_\theta}{r}\right] \bm{\hat{r}} + \\
       & \quad  + \left[ a_r \frac{\partial v_\theta}{\partial r} + \frac{a_\theta}{r}\frac{\partial v_\theta}{\partial \theta} +
                a_r \frac{\partial v_\theta}{\partial r} + \frac{a_\theta v_r}{r}\right] \bm{\hat{\theta}} + \\
       & \quad  + \left[ a_r \frac{\partial v_z}{\partial r} + \frac{a_\theta}{r}\frac{\partial v_z}{\partial \theta} +
                a_r \frac{\partial v_z}{\partial r} \right] \bm{\hat{z}}  
  \end{aligned}
 \end{equation}
 
 
 \subsubsection{Equazioni di Navier-Stokes}
 Utilizzando l'espressione in coordinate cilindriche degli operatori ricavate nel paragrafo precedente, è ora possibile
 scrivere le equazioni di Navier-Stokes in coordinate cilindriche.
 
\begin{equation}
\begin{cases}
 \dfrac{\partial \bm{u}}{\partial t} + \left( \bm{u} \cdot \bm{\nabla} \right) \bm{u} - \dfrac{1}{Re} \Delta \bm{u} + \bm{\nabla} p = \bm{f} \\
 \bm{\nabla} \cdot \bm{u} = 0
\end{cases}
\end{equation}

\begin{equation}
\begin{cases}
 \dfrac{\partial u_r}{\partial t}      + \left( u_r \dfrac{\partial u_r}{\partial r} + \dfrac{u_\theta}{r}\dfrac{\partial u_r}{\partial \theta} +
                u_r \dfrac{\partial u_r}{\partial r} - \dfrac{u_\theta^2}{r} \right) - \dfrac{1}{Re} \left[ 
                \Delta u_r - \dfrac{u_r}{r^2} - \dfrac{2}{r^2} \dfrac{\partial u_\theta}{\partial \theta}  \right] 
                + \dfrac{\partial p}{\partial r} = f_r   \\
 \dfrac{\partial u_\theta}{\partial t} + \left( u_r \dfrac{\partial u_\theta}{\partial r} + \dfrac{u_\theta}{r}\dfrac{\partial u_\theta}{\partial \theta} +
                u_r \dfrac{\partial u_\theta}{\partial r} + \dfrac{u_\theta u_r}{r} \right) - \dfrac{1}{Re} \left[ 
                \Delta u_\theta + \dfrac{2}{r^2} \dfrac{\partial u_r}{\partial \theta} - \dfrac{u_\theta}{r^2}  \right] 
                + \dfrac{1}{r}\dfrac{\partial p}{\partial \theta}= f_\theta \\
 \dfrac{\partial u_z}{\partial t}      + \left( u_r \dfrac{\partial u_z}{\partial r} + \dfrac{u_\theta}{r}\dfrac{\partial u_z}{\partial \theta} +
                u_r \dfrac{\partial u_z}{\partial r} \right) - \dfrac{1}{Re}  \Delta u_z  + \dfrac{\partial p}{\partial z} = f_z \\
 \dfrac{1}{r}\dfrac{\partial (r u_r)}{\partial r} + \dfrac{1}{r}\dfrac{\partial u_\theta}{\partial \theta} +
                        \dfrac{\partial u_z}{\partial z} = 0
\end{cases}
\end{equation} 
dove si è usato il laplaciano di un campo scalare, definito in precedenza. Per inciso, si ricorda che le equazioni hanno bisogno
 di condizioni iniziali (poichè sono evolutive nel tempo), di condizioni al contorno (poichè compaiono derivate spaziali) e,
 quando necessario, di condizioni di compatibilità, tra condizioni iniziali e al contorno.

 
 
 
