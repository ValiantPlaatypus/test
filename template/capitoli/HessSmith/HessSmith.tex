\documentclass[pdftex,10pt,a4paper]{article} %definisce come compilare, la grandezza del carattere di testo e della pagina, e lo stile del documento
\author{}
\title{LATEX}

%\usepackage[latin1]{inputenc} %definisce i caratteri con cui � scritto il testo, in questo caso latini

\usepackage[utf8x]{inputenc}

\usepackage[italian]{babel} %definisce la lingua principale del documento
\usepackage[pdftex]{graphicx} %serve per introdurre immagini
\DeclareGraphicsExtensions{.pdf,.png,.jpg,.mps,.eps} %estensione delle immagini permesse
\usepackage{amsmath} % pacchetto per formule e caratteri matematici
\usepackage{mathrsfs}
\usepackage{bm}
\usepackage{icomma} % pacchetto per limitare lo spazio standard posto dopo la virgola in caso che la virgola sia tra cifre
\usepackage{amsfonts} % amplia i caratteri matematici disponibili
\usepackage{amssymb} % amplia i simboli matematici
\makeindex %crea l'indice
\usepackage{indentfirst} % rientra la prima riga dei paragrafi

\usepackage{picinpar} % permette di inserire immagini nel paragrafo
\usepackage{setspace} %permette di modificare l'interlinea
%\usepackage{enumitem}
\singlespacing %interlinea 1,5. Per interlinea singola: \singlespacing Per interlinea doppia: \doublespacing per interlinea diversa \linespread{numero} con numero fattore di scala

\usepackage{epstopdf}  % converts .eps to .pdf for pdflatex

\usepackage{multicol}
%\usepackage{lipsum}
\usepackage[margin={2.5cm,2.5cm}]{geometry}

\usepackage{booktabs} %consente i comandi del tipo toprule nelle tabelle
\usepackage{longtable} %consente di spezzare le tabelle troppo lunghe per una sola pagina
\usepackage[printonlyused, withpage]{acronym} % pacchetto per gestire gli acronimi
%\usepackage{subfigure}
\usepackage{subfig} %pacchetto per affiancare le figure
%\usepackage{siunitx} %Pacchetto per l'uso delle unit� di misura del SI
\usepackage{fancyhdr} % pacchetto per le testatine delle pagine
%\usepackage{fancybox} % pacchetto per la gestione di box
\setlength{\headheight}{24pt} %modifica l'altezza delle testatine
\usepackage{rotating}
%
\usepackage[square,numbers,sort&compress]{natbib} % pacchetto per la bibliografia
\usepackage{cite}

%\usepackage[autostyle,italian=guillemets]{csquotes}
%\usepackage[backend=bibtex,hyperref]{biblatex}
%\addbibresource{Bibliografia.bib}

\usepackage{hyperref} % permette collegamenti ipertestuali nel documento-da inserire come ultimo pacchetto
%\hypersetup{pdffitwindow=false,pdftitle = {Analisi del segnale del grasso in NMR},pdfsubject = {Rilassometria}, pdfkeywords = {segnale, grasso}, pdfauthor = {Elena Pierpaoli},colorlinks=false,} %crea l'header file del pdf

\usepackage{float}
\usepackage{morefloats}

%-__-__-__-__-__-__-__-__-__-__-__-__-__-__-__-__-__-__-__-__-__-__-__-__-__-__-__-__-__-__-__-__-__-__-__-__-__-__-__-__-__-__-__-__-__-__-__-


\title{Metodo di Hess-Smith}
%\author{D. Montagnani}

\pagestyle{myheadings}
%\markright{\textit{D. Montagnani}, Velocimetria a immagini di particelle \hfill }
\markright{Metodo di Hess-Smith \hfill }

\begin{document}

\maketitle

%\begin{multicols}{2}

\tableofcontents

\vspace{1.5cm}
%\newpage

\section{Introduzione: metodi a pannelli}
\label{cap.00}
\begin{itemize}

 \item Ipotesi dei metodi a pannelli.

 \item Metodi diretti e inversi.

 \item Quando, come e perchè i metodi a pannelli:
  
  \begin{itemize}
    
   \item Problemi aerodinamici, con regioni di separazione limitate: 
   es. ok per profili e superfici alari a incidenze limitate (corrente attaccata, no stallo).
   
   \item Rispetto a soluzioni delle equazioni di NS o Eulero (non viscose), le incognite sono 
   concentrate sulla superficie del corpo: il numero delle ingocnite è ridotto e non serve una
   griglia in tutto il dominio fluido. 
   
   \item Per problemi 2D non stazionari e per problemi 3D stazionari e non, è necessario 
   modellare la scia.
   
   \item Per problemi aerodinamici 2D stazionari (no scia) e per problemi 2D e 3D con scia imposta
   il problema da risolvere è lineare.
      
   \item Ancora utilizzati in problemi di accoppiamento fluido-struttura (aeroelasticità,...).
   
   \item Riferimenti: Katz, Plotkin, Low Speed Aerodynamics; Bisplinghoff, Aeroelasticity.
  
  \end{itemize}

\end{itemize}


\section{Metodo di Hess-Smith}
\label{cap.01}
Il metodo di Hess-Smith è un metodo indiretto a pannelli.
Il potenziale della corrente è la somma del potenziale della corrente asintotica, della distribuzione di
sorgenti e della distribuzione di vortici. 

\subsection{Descrizione geometria}

Discretizzazione geometria (pannellizzazione).

\subsection{Equazioni: condizioni al contorno e condizione di Kutta}

Se il corpo è discretizzato in N pannelli, N equazioni del sistema descrivono la condizione 
al contorno di non penetrazione in N punti del profilo (es. centri dei pannelli).

\begin{equation}
 \bm{u} \cdot \bm{\hat{n}} = 0 \qquad \text{on S}
\end{equation}

Per corpi portanti 2D, deve essere imposta la condizione di Kutta al fine di eliminare la singolarità del problema:
questa singolarità è legata alla non unicità della soluzione per l'equazione di Laplace nel caso di dominio non
semplicemente connesso.

\begin{equation}
 \bm{u_{d_{TE}}} \cdot \bm{\hat{t}} = \bm{u_{v_{TE}}} \cdot \bm{\hat{t}}
\end{equation}

In totale quindi il sistema è composto da $N+1$ equazioni.

\subsection{Incognite: distribuzione vortici e sorgenti}

Il potenziale è la somma del potenziale della corrente asintotica, della distribuzione di
sorgenti e della distribuzione di vortici. Nel metodo di Hess-Smith si assume che l'intensità della
distribuzione di sorgenti e vortici sia costante su ogni pannello. In questo caso si avrebbero $2N$ incognite:
viene quindi fatta l'ipotesi che l'intensità dei vortici $\gamma$ sia costante su tutti i pannelli.

Le incognite sono quindi $N+1$: le N intensità della distribuzione di sorgenti sui pannelli $q_i, i=1:N$ e 
l'intensità dei vortici $\gamma$.

\begin{equation}
  \phi = \phi_\infty + \phi_V + \phi_S
\end{equation}
Con:
\begin{equation}
 \begin{aligned}
  & \phi_\infty = V_\infty \cos{\alpha} x + V_\infty \sin{\alpha} y \\
  & \phi_V =  \oint_{S} \gamma \frac{\theta}{2 \pi} ds \\
  & \phi_S =  \oint_{S} q \frac{\ln{r}}{2 \pi} ds \\
 \end{aligned}
\end{equation}

\subsection{Sistema lineare risolvente}

bla bla




\section{Breve guida all'implementazione}
\label{cap.02}
\input{Guida}

\newpage

\appendix



\end{document}
