\begin{itemize}

 \item Ipotesi dei metodi a pannelli.

 \item Metodi diretti e inversi.

 \item Quando, come e perchè i metodi a pannelli:
  
  \begin{itemize}
    
   \item Problemi aerodinamici, con regioni di separazione limitate: 
   es. ok per profili e superfici alari a incidenze limitate (corrente attaccata, no stallo).
   
   \item Rispetto a soluzioni delle equazioni di NS o Eulero (non viscose), le incognite sono 
   concentrate sulla superficie del corpo: il numero delle ingocnite è ridotto e non serve una
   griglia in tutto il dominio fluido. 
   
   \item Per problemi 2D non stazionari e per problemi 3D stazionari e non, è necessario 
   modellare la scia.
   
   \item Per problemi aerodinamici 2D stazionari (no scia) e per problemi 2D e 3D con scia imposta
   il problema da risolvere è lineare.
      
   \item Ancora utilizzati in problemi di accoppiamento fluido-struttura (aeroelasticità,...).
   
   \item Riferimenti: Katz, Plotkin, Low Speed Aerodynamics; Bisplinghoff, Aeroelasticity.
  
  \end{itemize}

\end{itemize}
