Il metodo di Hess-Smith è un metodo indiretto a pannelli.
Il potenziale della corrente è la somma del potenziale della corrente asintotica, della distribuzione di
sorgenti e della distribuzione di vortici. 

\subsection{Descrizione geometria}

Discretizzazione geometria (pannellizzazione).

\subsection{Equazioni: condizioni al contorno e condizione di Kutta}

Se il corpo è discretizzato in N pannelli, N equazioni del sistema descrivono la condizione 
al contorno di non penetrazione in N punti del profilo (es. centri dei pannelli).

\begin{equation}
 \bm{u} \cdot \bm{\hat{n}} = 0 \qquad \text{on S}
\end{equation}

Per corpi portanti 2D, deve essere imposta la condizione di Kutta al fine di eliminare la singolarità del problema:
questa singolarità è legata alla non unicità della soluzione per l'equazione di Laplace nel caso di dominio non
semplicemente connesso.

\begin{equation}
 \bm{u_{d_{TE}}} \cdot \bm{\hat{t}} = \bm{u_{v_{TE}}} \cdot \bm{\hat{t}}
\end{equation}

In totale quindi il sistema è composto da $N+1$ equazioni.

\subsection{Incognite: distribuzione vortici e sorgenti}

Il potenziale è la somma del potenziale della corrente asintotica, della distribuzione di
sorgenti e della distribuzione di vortici. Nel metodo di Hess-Smith si assume che l'intensità della
distribuzione di sorgenti e vortici sia costante su ogni pannello. In questo caso si avrebbero $2N$ incognite:
viene quindi fatta l'ipotesi che l'intensità dei vortici $\gamma$ sia costante su tutti i pannelli.

Le incognite sono quindi $N+1$: le N intensità della distribuzione di sorgenti sui pannelli $q_i, i=1:N$ e 
l'intensità dei vortici $\gamma$.

\begin{equation}
  \phi = \phi_\infty + \phi_V + \phi_S
\end{equation}
Con:
\begin{equation}
 \begin{aligned}
  & \phi_\infty = V_\infty \cos{\alpha} x + V_\infty \sin{\alpha} y \\
  & \phi_V =  \oint_{S} \gamma \frac{\theta}{2 \pi} ds \\
  & \phi_S =  \oint_{S} q \frac{\ln{r}}{2 \pi} ds \\
 \end{aligned}
\end{equation}

\subsection{Sistema lineare risolvente}

bla bla


