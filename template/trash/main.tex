%%%%%%%%%%%%%%%%%%%%%%%%%%%%%%%%%%%%%%%%%
%  My documentation report
%  Objetive: Explain what I did and how, so someone can continue with the investigation
%
% Important note:
% Chapter heading images should have a 2:1 width:height ratio,
% e.g. 920px width and 460px height.
%
%%%%%%%%%%%%%%%%%%%%%%%%%%%%%%%%%%%%%%%%%


%----------------------------------------------------------------------------------------
%	PACKAGES AND OTHER DOCUMENT CONFIGURATIONS
%----------------------------------------------------------------------------------------

\documentclass[11pt,fleqn]{book} % Default font size and left-justified equations

\usepackage[top=3cm,bottom=3cm,left=3.2cm,right=3.2cm,headsep=10pt,letterpaper]{geometry} % Page margins

\usepackage{xcolor} % Required for specifying colors by name
\definecolor{ocre}{RGB}{52,177,201} % Define the orange color used for highlighting throughout the book

% Font Settings
\usepackage{avant} % Use the Avantgarde font for headings
%\usepackage{times} % Use the Times font for headings
\usepackage{mathptmx} % Use the Adobe Times Roman as the default text font together with math symbols from the Sym­bol, Chancery and Com­puter Modern fonts
\usepackage{microtype} % Slightly tweak font spacing for aesthetics
\usepackage[utf8]{inputenc} % Required for including letters with accents
\usepackage[T1]{fontenc} % Use 8-bit encoding that has 256 glyphs
\usepackage{amsthm}

% Bibliography
\usepackage[style=alphabetic,sorting=nyt,sortcites=true,autopunct=true,babel=hyphen,hyperref=true,abbreviate=false,backref=true,backend=biber]{biblatex}
\addbibresource{bibliography.bib} % BibTeX bibliography file
\defbibheading{bibempty}{}

%%%%%%%%%%%%%%%%%%%%%%%%%%%%%%%%%%%%%%%%%
% The Legrand Orange Book
% Structural Definitions File
% Version 2.0 (9/2/15)
%
% Original author:
% Mathias Legrand (legrand.mathias@gmail.com) with modifications by:
% Vel (vel@latextemplates.com)
% 
% This file has been downloaded from:
% http://www.LaTeXTemplates.com
%
% License:
% CC BY-NC-SA 3.0 (http://creativecommons.org/licenses/by-nc-sa/3.0/)
%
%%%%%%%%%%%%%%%%%%%%%%%%%%%%%%%%%%%%%%%%%

%----------------------------------------------------------------------------------------
%	VARIOUS REQUIRED PACKAGES AND CONFIGURATIONS
%----------------------------------------------------------------------------------------

\usepackage[top=3cm,bottom=3cm,left=3cm,right=3cm,headsep=10pt,a4paper]{geometry} % Page margins

\usepackage{graphicx} % Required for including pictures
\graphicspath{{Pictures/}} % Specifies the directory where pictures are stored

\usepackage{lipsum} % Inserts dummy text

\usepackage{tikz} % Required for drawing custom shapes

% \usepackage[english]{babel} % English language/hyphenation
\usepackage[english,italian]{babel} % English language/hyphenation

\usepackage{enumitem} % Customize lists
\setlist{nolistsep} % Reduce spacing between bullet points and numbered lists

\usepackage{booktabs} % Required for nicer horizontal rules in tables

\usepackage{xcolor} % Required for specifying colors by name
\definecolor{ocre}{RGB}{243,102,25} % Define the orange color used for highlighting throughout the book

%----------------------------------------------------------------------------------------
%	FONTS
%----------------------------------------------------------------------------------------

\usepackage{avant} % Use the Avantgarde font for headings
%\usepackage{times} % Use the Times font for headings
% \usepackage{mathptmx} % Use the Adobe Times Roman as the default text font together with math symbols from the Sym­bol, Chancery and Com­puter Modern fonts
\usepackage{bm}

\usepackage{microtype} % Slightly tweak font spacing for aesthetics
\usepackage[utf8x]{inputenc} % Required for including letters with accents
\usepackage[T1]{fontenc} % Use 8-bit encoding that has 256 glyphs
\usepackage{lmodern}

%----------------------------------------------------------------------------------------
%	BIBLIOGRAPHY AND INDEX
%----------------------------------------------------------------------------------------

%\usepackage[style=numeric,citestyle=numeric,sorting=nyt,sortcites=true,autopunct=true,babel=hyphen,hyperref=true,abbreviate=false,backref=true,backend=biber]{biblatex}
%\addbibresource{bibliography.bib} % BibTeX bibliography file
%\defbibheading{bibempty}{}

\usepackage[numbers,authoryear,round]{natbib}

\usepackage{calc} % For simpler calculation - used for spacing the index letter headings correctly
\usepackage{makeidx} % Required to make an index
\makeindex % Tells LaTeX to create the files required for indexing

%----------------------------------------------------------------------------------------
%	MAIN TABLE OF CONTENTS
%----------------------------------------------------------------------------------------

\usepackage{titletoc} % Required for manipulating the table of contents

\contentsmargin{0cm} % Removes the default margin

% Part text styling
\titlecontents{part}[0cm]
{\addvspace{20pt}\centering\large\bfseries}
{}
{}
{}

% Chapter text styling
\titlecontents{chapter}[1.25cm] % Indentation
{\addvspace{12pt}\large\sffamily\bfseries} % Spacing and font options for chapters
{\color{ocre!60}\contentslabel[\Large\thecontentslabel]{1.25cm}\color{ocre}} % Chapter number
{\color{ocre}}  
{\color{ocre!60}\normalsize\;\titlerule*[.5pc]{.}\;\thecontentspage} % Page number

% Section text styling
\titlecontents{section}[1.25cm] % Indentation
{\addvspace{3pt}\sffamily\bfseries} % Spacing and font options for sections
{\contentslabel[\thecontentslabel]{1.25cm}} % Section number
{}
{\hfill\color{black}\thecontentspage} % Page number
[]

% Subsection text styling
\titlecontents{subsection}[1.25cm] % Indentation
{\addvspace{1pt}\sffamily\small} % Spacing and font options for subsections
{\contentslabel[\thecontentslabel]{1.25cm}} % Subsection number
{}
{\ \titlerule*[.5pc]{.}\;\thecontentspage} % Page number
[]

% List of figures
\titlecontents{figure}[0em]
{\addvspace{-5pt}\sffamily}
{\thecontentslabel\hspace*{1em}}
{}
{\ \titlerule*[.5pc]{.}\;\thecontentspage}
[]

% List of tables
\titlecontents{table}[0em]
{\addvspace{-5pt}\sffamily}
{\thecontentslabel\hspace*{1em}}
{}
{\ \titlerule*[.5pc]{.}\;\thecontentspage}
[]

%----------------------------------------------------------------------------------------
%	MINI TABLE OF CONTENTS IN PART HEADS
%----------------------------------------------------------------------------------------

% Chapter text styling
\titlecontents{lchapter}[0em] % Indenting
{\addvspace{15pt}\large\sffamily\bfseries} % Spacing and font options for chapters
{\color{ocre}\contentslabel[\Large\thecontentslabel]{1.25cm}\color{ocre}} % Chapter number
{}  
{\color{ocre}\normalsize\sffamily\bfseries\;\titlerule*[.5pc]{.}\;\thecontentspage} % Page number

% Section text styling
\titlecontents{lsection}[0em] % Indenting
{\sffamily\small} % Spacing and font options for sections
{\contentslabel[\thecontentslabel]{1.25cm}} % Section number
{}
{}

% Subsection text styling
\titlecontents{lsubsection}[.5em] % Indentation
{\normalfont\footnotesize\sffamily} % Font settings
{}
{}
{}

%----------------------------------------------------------------------------------------
%	PAGE HEADERS
%----------------------------------------------------------------------------------------

\usepackage{fancyhdr} % Required for header and footer configuration

\pagestyle{fancy}
\renewcommand{\chaptermark}[1]{\markboth{\sffamily\normalsize\bfseries\chaptername\ \thechapter.\ #1}{}} % Chapter text font settings
\renewcommand{\sectionmark}[1]{\markright{\sffamily\normalsize\thesection\hspace{5pt}#1}{}} % Section text font settings
\fancyhf{} \fancyhead[LE,RO]{\sffamily\normalsize\thepage} % Font setting for the page number in the header
\fancyhead[LO]{\rightmark} % Print the nearest section name on the left side of odd pages
\fancyhead[RE]{\leftmark} % Print the current chapter name on the right side of even pages
\renewcommand{\headrulewidth}{0.5pt} % Width of the rule under the header
\addtolength{\headheight}{2.5pt} % Increase the spacing around the header slightly
\renewcommand{\footrulewidth}{0pt} % Removes the rule in the footer
\fancypagestyle{plain}{\fancyhead{}\renewcommand{\headrulewidth}{0pt}} % Style for when a plain pagestyle is specified

% Removes the header from odd empty pages at the end of chapters
\makeatletter
\renewcommand{\cleardoublepage}{
\clearpage\ifodd\c@page\else
\hbox{}
\vspace*{\fill}
\thispagestyle{empty}
\newpage
\fi}

%----------------------------------------------------------------------------------------
%	THEOREM STYLES
%----------------------------------------------------------------------------------------

\usepackage{amsmath,amsfonts,amssymb,amsthm} % For math equations, theorems, symbols, etc
\usepackage{bm}

\newcommand{\intoo}[2]{\mathopen{]}#1\,;#2\mathclose{[}}
\newcommand{\ud}{\mathop{\mathrm{{}d}}\mathopen{}}
\newcommand{\intff}[2]{\mathopen{[}#1\,;#2\mathclose{]}}
\newtheorem{notation}{Notazione}[chapter]

% Boxed/framed environments
\newtheoremstyle{ocrenumbox}% % Theorem style name
{0pt}% Space above
{0pt}% Space below
{\normalfont}% % Body font
{}% Indent amount
{\small\bf\sffamily\color{ocre}}% % Theorem head font
{\;}% Punctuation after theorem head
{0.25em}% Space after theorem head
{\small\sffamily\color{ocre}\thmname{#1}\nobreakspace.\thmnumber{\@ifnotempty{#1}{}\@upn{#2}}% Theorem text (e.g. Theorem 2.1)
\thmnote{\nobreakspace\the\thm@notefont\sffamily\bfseries\color{black}---\nobreakspace#3.}} % Optional theorem note
\renewcommand{\qedsymbol}{$\blacksquare$}% Optional qed square

\newtheoremstyle{blacknumex}% Theorem style name
{5pt}% Space above
{5pt}% Space below
{\normalfont}% Body font
{} % Indent amount
{\small\bf\sffamily}% Theorem head font
{\;}% Punctuation after theorem head
{0.25em}% Space after theorem head
{\small\sffamily{\tiny\ensuremath{\blacksquare}}\nobreakspace\thmname{#1}\nobreakspace\thmnumber{\@ifnotempty{#1}{}\@upn{#2}}% Theorem text (e.g. Theorem 2.1)
\thmnote{\nobreakspace\the\thm@notefont\sffamily\bfseries---\nobreakspace#3.}}% Optional theorem note

\newtheoremstyle{blacknumbox} % Theorem style name
{0pt}% Space above
{0pt}% Space below
{\normalfont}% Body font
{}% Indent amount
{\small\bf\sffamily}% Theorem head font
{\;}% Punctuation after theorem head
{0.25em}% Space after theorem head
{\small\sffamily\thmname{#1}\nobreakspace\thmnumber{\@ifnotempty{#1}{}\@upn{#2}}% Theorem text (e.g. Theorem 2.1)
\thmnote{\nobreakspace\the\thm@notefont\sffamily\bfseries---\nobreakspace#3.}}% Optional theorem note

% Non-boxed/non-framed environments
\newtheoremstyle{ocrenum}% % Theorem style name
{5pt}% Space above
{5pt}% Space below
{\normalfont}% % Body font
{}% Indent amount
{\small\bf\sffamily\color{ocre}}% % Theorem head font
{\;}% Punctuation after theorem head
{0.25em}% Space after theorem head
{\small\sffamily\color{ocre}\thmname{#1}\nobreakspace\thmnumber{\@ifnotempty{#1}{}\@upn{#2}}% Theorem text (e.g. Theorem 2.1)
\thmnote{\nobreakspace\the\thm@notefont\sffamily\bfseries\color{black}---\nobreakspace#3.}} % Optional theorem note
\renewcommand{\qedsymbol}{$\blacksquare$}% Optional qed square
\makeatother

% ENGLISH +++++++
%% Defines the theorem text style for each type of theorem to one of the three styles above
%\newcounter{dummy} 
%\numberwithin{dummy}{section}
%\theoremstyle{ocrenumbox}
%\newtheorem{theoremeT}[dummy]{Theorem}
%\newtheorem{problem}{Problem}[chapter]
%\newtheorem{exerciseT}{Exercise}[chapter]
%\theoremstyle{blacknumex}
%\newtheorem{exampleT}{Example}[chapter]
%\theoremstyle{blacknumbox}
%\newtheorem{vocabulary}{Vocabulary}[chapter]
%\newtheorem{definitionT}{Definition}[section]
%\newtheorem{corollaryT}[dummy]{Corollary}
%\theoremstyle{ocrenum}
%\newtheorem{proposition}[dummy]{Proposition}

% ITALIANO ++++
% Defines the theorem text style for each type of theorem to one of the three styles above
\newcounter{dummy} 
\numberwithin{dummy}{section}
\theoremstyle{ocrenumbox}
\newcounter{numExer}[section]
\newtheorem{theoremeT}[dummy]{Teorema}
\newtheorem{problem}{Problem}[chapter]
\newtheorem{lemmaeT}[dummy]{Lemma}
\newtheorem{exerciseT}{Esercizio}[chapter]
\newtheorem{exerciseN}[numExer]{Esercizio \thechapter\!}
\theoremstyle{blacknumex}
\newtheorem{exampleT}{Esempio}[chapter]
\theoremstyle{blacknumbox}
\newtheorem{vocabulary}{Vocabulary}[chapter]
\newtheorem{definitionT}{Definizione}[section]
\newtheorem{operatorT}{Operatore}[section]
\newtheorem{corollaryT}[dummy]{Corollary}
\theoremstyle{ocrenum}
\newtheorem{proposition}[dummy]{Proposizione}
%\newtheorem{regola}[dummy]{Regola}
\newtheorem{regolaT}{Regola}[chapter]


%----------------------------------------------------------------------------------------
%	DEFINITION OF COLORED BOXES
%----------------------------------------------------------------------------------------

\RequirePackage[framemethod=default]{mdframed} % Required for creating the theorem, definition, exercise and corollary boxes

% Theorem box
\newmdenv[skipabove=7pt,
skipbelow=7pt,
backgroundcolor=black!5,
linecolor=ocre,
innerleftmargin=5pt,
innerrightmargin=5pt,
innertopmargin=5pt,
leftmargin=0cm,
rightmargin=0cm,
innerbottommargin=5pt]{tBox}

% Formula box
\newmdenv[skipabove=7pt,
skipbelow=7pt,
backgroundcolor=black!5,
linecolor=ocre,
innerleftmargin=5pt,
innerrightmargin=5pt,
innertopmargin=-3pt,
leftmargin=0cm,
rightmargin=0cm,
innerbottommargin=10pt]{fBox}

% Exercise box	  
\newmdenv[skipabove=7pt,
skipbelow=7pt,
rightline=false,
leftline=true,
topline=false,
bottomline=false,
backgroundcolor=ocre!10,
linecolor=ocre,
innerleftmargin=5pt,
innerrightmargin=5pt,
innertopmargin=5pt,
innerbottommargin=5pt,
leftmargin=0cm,
rightmargin=0cm,
linewidth=4pt]{eBox}	

% Definition box
\newmdenv[skipabove=7pt,
skipbelow=7pt,
rightline=false,
leftline=true,
topline=false,
bottomline=false,
linecolor=ocre,
innerleftmargin=5pt,
innerrightmargin=5pt,
innertopmargin=0pt,
leftmargin=0cm,
rightmargin=0cm,
linewidth=4pt,
innerbottommargin=0pt]{dBox}	

% Corollary box
\newmdenv[skipabove=7pt,
skipbelow=7pt,
rightline=false,
leftline=true,
topline=false,
bottomline=false,
linecolor=gray,
backgroundcolor=black!5,
innerleftmargin=5pt,
innerrightmargin=5pt,
innertopmargin=5pt,
leftmargin=0cm,
rightmargin=0cm,
linewidth=4pt,
innerbottommargin=5pt]{cBox}

% Creates an environment for each type of theorem and assigns it a theorem text style from the "Theorem Styles" section above and a colored box from above
\newenvironment{theorem}{\begin{tBox}\begin{theoremeT}}{\end{theoremeT}\end{tBox}}
\newenvironment{lemma}{\begin{tBox}\begin{lemmaeT}}{\end{lemmaeT}\end{tBox}}
\newenvironment{exercise}{\begin{eBox}\begin{exerciseT}}{\hfill{\color{ocre}\tiny\ensuremath{\blacksquare}}\end{exerciseT}\end{eBox}}
\newenvironment{exerciseS}{\begin{eBox}\begin{exerciseN}}{\hfill{\color{ocre}\tiny\ensuremath{\blacksquare}}\end{exerciseN}\end{eBox}}				  
\newenvironment{definition}{\begin{dBox}\begin{definitionT}}{\end{definitionT}\end{dBox}}
\newenvironment{operator}{\begin{dBox}\begin{operatorT}}{\end{operatorT}\end{dBox}}	
\newenvironment{example}{\begin{exampleT}}{\hfill{\tiny\ensuremath{\blacksquare}}\end{exampleT}}		
\newenvironment{corollary}{\begin{cBox}\begin{corollaryT}}{\end{corollaryT}\end{cBox}}
\newenvironment{regola}{\begin{eBox}\begin{regolaT}}{\end{regolaT}\end{eBox}}	

%----------------------------------------------------------------------------------------
%	REMARK ENVIRONMENT
%----------------------------------------------------------------------------------------

\newenvironment{remark}{\par\vspace{10pt}\small % Vertical white space above the remark and smaller font size
\begin{list}{}{
\leftmargin=35pt % Indentation on the left
\rightmargin=25pt}\item\ignorespaces % Indentation on the right
\makebox[-2.5pt]{\begin{tikzpicture}[overlay]
\node[draw=ocre!60,line width=1pt,circle,fill=ocre!25,font=\sffamily\bfseries,inner sep=2pt,outer sep=0pt] at (-15pt,0pt){\textcolor{ocre}{R}};\end{tikzpicture}} % Orange R in a circle
\advance\baselineskip -1pt}{\end{list}\vskip5pt} % Tighter line spacing and white space after remark

%----------------------------------------------------------------------------------------
%	SECTION NUMBERING IN THE MARGIN
%----------------------------------------------------------------------------------------

\makeatletter
\renewcommand{\@seccntformat}[1]{\llap{\textcolor{ocre}{\csname the#1\endcsname}\hspace{1em}}}                    
\renewcommand{\section}{\@startsection{section}{1}{\z@}
{-4ex \@plus -1ex \@minus -.4ex}
{1ex \@plus.2ex }
{\normalfont\large\sffamily\bfseries}}
\renewcommand{\subsection}{\@startsection {subsection}{2}{\z@}
{-3ex \@plus -0.1ex \@minus -.4ex}
{0.5ex \@plus.2ex }
{\normalfont\sffamily\bfseries}}
\renewcommand{\subsubsection}{\@startsection {subsubsection}{3}{\z@}
{-2ex \@plus -0.1ex \@minus -.2ex}
{.2ex \@plus.2ex }
{\normalfont\small\sffamily\bfseries}}                        
\renewcommand\paragraph{\@startsection{paragraph}{4}{\z@}
{-2ex \@plus-.2ex \@minus .2ex}
{.1ex}
{\normalfont\small\sffamily\bfseries}}

%----------------------------------------------------------------------------------------
%	PART HEADINGS
%----------------------------------------------------------------------------------------

% numbered part in the table of contents
\newcommand{\@mypartnumtocformat}[2]{%
\setlength\fboxsep{0pt}%
\noindent\colorbox{ocre!20}{\strut\parbox[c][.7cm]{\ecart}{\color{ocre!70}\Large\sffamily\bfseries\centering#1}}\hskip\esp\colorbox{ocre!40}{\strut\parbox[c][.7cm]{\linewidth-\ecart-\esp}{\Large\sffamily\centering#2}}}%
%%%%%%%%%%%%%%%%%%%%%%%%%%%%%%%%%%
% unnumbered part in the table of contents
\newcommand{\@myparttocformat}[1]{%
\setlength\fboxsep{0pt}%
\noindent\colorbox{ocre!40}{\strut\parbox[c][.7cm]{\linewidth}{\Large\sffamily\centering#1}}}%
%%%%%%%%%%%%%%%%%%%%%%%%%%%%%%%%%%
\newlength\esp
\setlength\esp{4pt}
\newlength\ecart
\setlength\ecart{1.2cm-\esp}
\newcommand{\thepartimage}{}%
\newcommand{\partimage}[1]{\renewcommand{\thepartimage}{#1}}%
\def\@part[#1]#2{%
\ifnum \c@secnumdepth >-2\relax%
\refstepcounter{part}%
\addcontentsline{toc}{part}{\texorpdfstring{\protect\@mypartnumtocformat{\thepart}{#1}}{\partname~\thepart\ ---\ #1}}
\else%
\addcontentsline{toc}{part}{\texorpdfstring{\protect\@myparttocformat{#1}}{#1}}%
\fi%
\startcontents%
\markboth{}{}%
{\thispagestyle{empty}%
\begin{tikzpicture}[remember picture,overlay]%
\node at (current page.north west){\begin{tikzpicture}[remember picture,overlay]%	
\fill[ocre!20](0cm,0cm) rectangle (\paperwidth,-\paperheight);
\node[anchor=north] at (4cm,-3.25cm){\color{ocre!40}\fontsize{220}{100}\sffamily\bfseries\thepart}; 
\node[anchor=south east] at (\paperwidth-1cm,-\paperheight+1cm){\parbox[t][][t]{8.5cm}{
\printcontents{l}{0}{\setcounter{tocdepth}{1}}%
}};
\node[anchor=north east] at (\paperwidth-1.5cm,-3.25cm){\parbox[t][][t]{15cm}{\strut\raggedleft\color{white}\fontsize{30}{30}\sffamily\bfseries#2}};
\end{tikzpicture}};
\end{tikzpicture}}%
\@endpart}
\def\@spart#1{%
\startcontents%
\phantomsection
{\thispagestyle{empty}%
\begin{tikzpicture}[remember picture,overlay]%
\node at (current page.north west){\begin{tikzpicture}[remember picture,overlay]%	
\fill[ocre!20](0cm,0cm) rectangle (\paperwidth,-\paperheight);
\node[anchor=north east] at (\paperwidth-1.5cm,-3.25cm){\parbox[t][][t]{15cm}{\strut\raggedleft\color{white}\fontsize{30}{30}\sffamily\bfseries#1}};
\end{tikzpicture}};
\end{tikzpicture}}
\addcontentsline{toc}{part}{\texorpdfstring{%
\setlength\fboxsep{0pt}%
\noindent\protect\colorbox{ocre!40}{\strut\protect\parbox[c][.7cm]{\linewidth}{\Large\sffamily\protect\centering #1\quad\mbox{}}}}{#1}}%
\@endpart}
\def\@endpart{\vfil\newpage
\if@twoside
\if@openright
\null
\thispagestyle{empty}%
\newpage
\fi
\fi
\if@tempswa
\twocolumn
\fi}

%----------------------------------------------------------------------------------------
%	CHAPTER HEADINGS
%----------------------------------------------------------------------------------------

% A switch to conditionally include a picture, implemented by  Christian Hupfer
\newif\ifusechapterimage
\usechapterimagetrue
\newcommand{\thechapterimage}{}%
\newcommand{\chapterimage}[1]{\ifusechapterimage\renewcommand{\thechapterimage}{#1}\fi}%
\newcommand{\autodot}{.}
\def\@makechapterhead#1{%
{\parindent \z@ \raggedright \normalfont
\ifnum \c@secnumdepth >\m@ne
\if@mainmatter
\begin{tikzpicture}[remember picture,overlay]
\node at (current page.north west)
{\begin{tikzpicture}[remember picture,overlay]
\node[anchor=north west,inner sep=0pt] at (0,0) {\ifusechapterimage\includegraphics[width=\paperwidth]{\thechapterimage}\fi};
\draw[anchor=west] (\Gm@lmargin,-9cm) node [line width=2pt,rounded corners=15pt,draw=ocre,fill=white,fill opacity=0.5,inner sep=15pt]{\strut\makebox[22cm]{}};
\draw[anchor=west] (\Gm@lmargin+.3cm,-9cm) node {\huge\sffamily\bfseries\color{black}\thechapter\autodot~#1\strut};
\end{tikzpicture}};
\end{tikzpicture}
\else
\begin{tikzpicture}[remember picture,overlay]
\node at (current page.north west)
{\begin{tikzpicture}[remember picture,overlay]
\node[anchor=north west,inner sep=0pt] at (0,0) {\ifusechapterimage\includegraphics[width=\paperwidth]{\thechapterimage}\fi};
\draw[anchor=west] (\Gm@lmargin,-9cm) node [line width=2pt,rounded corners=15pt,draw=ocre,fill=white,fill opacity=0.5,inner sep=15pt]{\strut\makebox[22cm]{}};
\draw[anchor=west] (\Gm@lmargin+.3cm,-9cm) node {\huge\sffamily\bfseries\color{black}#1\strut};
\end{tikzpicture}};
\end{tikzpicture}
\fi\fi\par\vspace*{270\p@}}}

%-------------------------------------------

\def\@makeschapterhead#1{%
\begin{tikzpicture}[remember picture,overlay]
\node at (current page.north west)
{\begin{tikzpicture}[remember picture,overlay]
\node[anchor=north west,inner sep=0pt] at (0,0) {\ifusechapterimage\includegraphics[width=\paperwidth]{\thechapterimage}\fi};
\draw[anchor=west] (\Gm@lmargin,-9cm) node [line width=2pt,rounded corners=15pt,draw=ocre,fill=white,fill opacity=0.5,inner sep=15pt]{\strut\makebox[22cm]{}};
\draw[anchor=west] (\Gm@lmargin+.3cm,-9cm) node {\huge\sffamily\bfseries\color{black}#1\strut};
\end{tikzpicture}};
\end{tikzpicture}
\par\vspace*{270\p@}}
\makeatother

%----------------------------------------------------------------------------------------
%	HYPERLINKS IN THE DOCUMENTS
%----------------------------------------------------------------------------------------

\usepackage{hyperref}
\hypersetup{hidelinks,backref=true,pagebackref=true,hyperindex=true,colorlinks=true,breaklinks=true,urlcolor= ocre,bookmarks=true,bookmarksopen=false,pdftitle={Title},pdfauthor={Author}}
\usepackage{bookmark}
\bookmarksetup{
open,
numbered,
addtohook={%
\ifnum\bookmarkget{level}=0 % chapter
\bookmarksetup{bold}%
\fi
\ifnum\bookmarkget{level}=-1 % part
\bookmarksetup{color=ocre,bold}%
\fi
}
}
 % Insert the commands.tex file which contains the majority of the structure behind the template

%----------------------------------------------------------------------------------------
%	Definitions of new commands
%----------------------------------------------------------------------------------------

\def\R{\mathbb{R}}
\newcommand{\cvx}{convex}
\begin{document}

%----------------------------------------------------------------------------------------
%	TITLE PAGE
%----------------------------------------------------------------------------------------

\begingroup
\thispagestyle{empty}
\AddToShipoutPicture*{\put(0,0){\includegraphics[scale=1.25]{esahubble}}} % Image background
\centering
\vspace*{5cm}
\par\normalfont\fontsize{35}{35}\sffamily\selectfont
\textbf{CPSC 542F WINTER 2017}\\
{\LARGE Convex Analysis and Optimization}\par % Book title
\vspace*{1cm}
{\Huge Lecture Notes}\par % Author name
\endgroup

%----------------------------------------------------------------------------------------
%	COPYRIGHT PAGE
%----------------------------------------------------------------------------------------

\newpage
~\vfill
\thispagestyle{empty}

%\noindent Copyright \copyright\ 2014 Andrea Hidalgo\\ % Copyright notice

\noindent \textsc{Summer Research Internship, University of Western Ontario}\\

\noindent \textsc{github.com/LaurethTeX/Clustering}\\ % URL

\noindent This research was done under the supervision of Dr. Pauline Barmby with the financial support of the MITACS Globalink Research Internship Award within a total of 12 weeks, from June 16th to September 5th of 2014.\\ % License information

\noindent \textit{First release, August 2014} % Printing/edition date

%----------------------------------------------------------------------------------------
%	TABLE OF CONTENTS
%----------------------------------------------------------------------------------------

\chapterimage{head1.png} % Table of contents heading image

\pagestyle{empty} % No headers

\tableofcontents % Print the table of contents itself

%\cleardoublepage % Forces the first chapter to start on an odd page so it's on the right

\pagestyle{fancy} % Print headers again

%----------------------------------------------------------------------------------------
%	CHAPTER 1
%----------------------------------------------------------------------------------------

\chapterimage{head2.png} % Chapter heading image
\chapter{Convex Sets}
\section{Convexity}
\subsection{Cone}
\begin{definition}[Cone]
A set $K \in \R^n$, when $x \in K $ implies $\alpha x \in K$.
\end{definition}
A non convex cone can be hyper-plane.\\
For convex cone $x + y \in K, \forall x,y \in K$.\\
Cone don't need to be "pointed". e.g. \\
Direct sums of cones $C_1 + C_2 = \{ x = x_1+x_2 | x_1 \in C_1, x_2 \in C_2 \}$.\\
\begin{example}
$S_1^n  \{ X | X=X^n ,\lambda(x) \ge 0\}$\\
A matrix with positive eigenvalues.
\end{example}

\subsubsection{Operations preserving convexity}
\begin{itemize}
\item[Intersection] $C  \cap_{i \in \mathbb{I}}C_i$
\item[Linear map] Let $A : \mathbb{R}^n \to  \R^n$ be a linear map. If $C \in \R^n$ is convex, so is $A(C) = \{Ax \forall x \in C \}$
\item[Inverse image] $A^{-1}(D) = \{ x \in \R |Ax \in D \}$
\end{itemize}

\subsubsection{Operations that induce convexity}
Convex hull on $S = \cap \{C | S\in C, C is convex\}$\\
\begin{example}
$Co \{ x_1,x_2,\cdots,x_m\} = \{ \sum_{i=1}^m \alpha_i x_i | \alpha \in \delta_m \}$
\end{example}
For a convex set $x \in C \Rightarrow x = \sum \alpha_i x_i$. 
\begin{theorem}[Carathéodory's theorem]
If a point $x \in \R^d$ lies in the convex hull of a set $P$, there is a subset $P'$ of $P$ consisting of $d + 1$ or fewer points such that $x$ lies in the convex hull of $P'$. Equivalently, x lies in an r-simplex with vertices in P.
\end{theorem}

\section{Convex Functions}
\begin{definition}[Convex function]
Let $C \in \R^n$ be convex, $f:C \to \R$ is convex on f if $x,y \in C \times C$. $\forall \alpha \in (0,1)$, $f(\alpha x + (1-\alpha) y) \le f(\alpha x) + f((1-\alpha) y)$
\end{definition}

\begin{definition}[Strictly Convex function]
Let $C \in \R^n$ be convex, $f:C \to \R$ is strictly convex on f if $x,y \in C \times C$. $\forall \alpha \in (0,1)$, $f(\alpha x + (1-\alpha) y) \langle f(\alpha x) + f((1-\alpha) y)$
\end{definition}

\begin{definition}[Strongly convex]
$f:C \to \R$ is strongly convex with modules $u \ge 0$ if $f - \frac{1}{2}u || \cdot ||^2$ is convex.
\end{definition}
Interpretation: There is a convex quadratic $\frac{1}{2}u || \cdot ||^2$ that lower bounds f.
\begin{example}
$\min_{x \in C} f(x) \leftrightarrow \min \bar{f}(x)$
Useful to turn this into an unconstrained problem. \\
$$\bar{f}(x) = \begin{cases}
f(x) \quad if x \in C \\
\infty \quad  elsewhere
\end{cases}$$
\end{example}
\begin{definition}
A function $f : \R^n \to \R \cup \infty \ \bar{\R}$ is convex if $x,y \in \R^n \times \R^n$, $\forall x,y , \bar{f}(\alpha x + (1-\alpha) y) \le f(\alpha x) + f((1-\alpha) y)$
\end{definition}
Definition 1 is equivalent to definition 2 if $f(x) = \infty$.
\begin{example}
$f(x) = \sup_{j \in J} f_j(x)$
\end{example}

\subsection{Epigraph} 
\begin{definition}[Epigraph]
For $f: \R^n \rightarrow \bar{R}$, its epigraph $epi(f) \in \R^{n+1} is the set epi(f) \{ (x,\alpha) | f(x) \in \alpha \}$
\end{definition}
Next: a function is convex i.f.f. its epigraph is convex.

\begin{definition}
A function $f : C \rightarrow \R, C \in \R^n$ is convex if $\forall x, y \in C$, $f(ax + (1-a)x) \le af(x) + (1-a)f(x) \quad \forall a \in (0,1)$.\\ 
Strict convex: $x \neq y \Rightarrow f(ax + (1-a)x) \le af(x) + (1-a)f(x) $
\end{definition}
\begin{remark}
$f$ is convex $\Rightarrow$ $-f$ is concave.
\end{remark}
Level set: $S_{\alpha}f = \{ x | f(x) \le \alpha \}$.\\ 
$S_{\alpha}f$ is convex $\Leftrightarrow$ $f$ is convex. \\
\begin{definition}[Strongly convex]
$f : C \rightarrow \R$ is strongly convex with modules $\mu$ if $\forall x, y \in C$, $\forall \alpha \in (0,1)$, $f(ax + (1-a)x) \le af(x) + (1-a)f(x) - \frac{1}{2\mu}\alpha(1- \alpha) \|x-y\|^2$.
\end{definition}

\begin{remark}
\begin{itemize}
\item $f$ is 2nd-differentiable, $f$ ix \cvx $\iff$ $\nabla^2f(x) \rangle  0$.
\item $f$ is strongly \cvx $\iff$ $\nabla^2f(x) \rangle  \mu I$ $\iff$ $x \ge \mu$
\end{itemize}
\end{remark}
\begin{definition}[2]
$f : \R^n \to \bar{\R} $ is \cvx  if $x, y  \in \R , \alpha \in (0,1), f(ax + (1-a)x) \le af(x) + (1-a)f(x)$.  
\end{definition}
The effective domain of $f$ is $dom f = \{x | f(x) \langle + \infty \}$ 
\begin{example}[ludcator function]
$\delta_c(x) = \begin{cases}
0 \quad  x \in C \\
+ \infty \quad elsewhere
\end{cases}$.\\
$dom \space \delta_c(x) = C$
\end{example}
\begin{definition}[Epigraph]
The epigraph of f is $epi \space f = \{(x,\alpha) | f(x) \le \alpha\}$
\end{definition}
The graph of $epi \space f$ is $\{ (x,f(x) | x \in dom \space f\}$.
\begin{definition}[III]
A function $f : \R^n \to \bar{\R}$ is %\cvx  if $\epi \space f $ is \cvx
\end{definition}
\begin{theorem}
$f : \R^n \to \bar{\R}$ is \cvx  $\iff$ $\forall x,y \in \R^n, \alpha \in (0,1), f(ax + (1-a)x) \le af(x) + (1-a)f(x)$.
\end{theorem}
\begin{proof}
$\Rightarrow$ take $x,y \in dom \space f$, $(x,f(x)) \in epi \space f$,$(y,f(y)) \in epi \space f$.
\end{proof}

\begin{example}[Distance]
Distance to a \cvx  set $d_c(x) = \inf \{ \| z-x \| | z \in C \}$. Take any two sequence $\{ y_k\} and \{ \bar{y}_k\} \subset C$ s.t. $\| y_k - x\| \to d_c(x)$, $\| \bar{y}_k - \bar{x}\| \to d_c(\bar{x})$. $z_k = \alpha y_k + (1 - \alpha) \bar{y}_k$.
\begin{align*}
d_c(\alpha x + (1-\alpha) \bar{x}) &\le \| z_k - \alpha x - (1 - \alpha) \bar{x}\| \\
& = \| \alpha(y_k - x) + (1 - \alpha)(\bar{y}_k - \bar{x})\| \\
& \le \alpha \| y_k - x\| + (1 - \alpha ) \|\bar{y}_k - \bar{x}\|
\end{align*}
Take $k \to \infty$, $d_c(\alpha x + (1 - \alpha) \bar{x}) \le \alpha d(x) + (1 - \alpha) d(\bar{x})$
\end{example}
\begin{example}[Eigenvalues]
Let $X \in S^n := \{ n \times n symmetric matrix\}$. $\lambda_1(x) \ge \lambda_2(X) \ge \ldots \ge \lambda_n(x)$.\\
$f_k(x) = \sum_{1}^n \lambda_i(x)$.\\
Equivalent characterization 

\begin{align*}
f_k(x) & = \max\{ \sum_{i} v_i^T Xv_i | v_i \perp v_j , i \neq j\} \\
& =  \max\{ tr( V^TXV | V^T V = I_k \} \\
\max \{tr(VV^TX) \} \text{by circularity}
\end{align*}
Note $\langle A,B\rangle  = tr(A,B)$ is true for symmetric matrix. \\
$\langle A,A\rangle  = |A |_F^2 = \sum_{i} A_{ii}^2$
\end{example}

\section{Support Function}
Take a set $C \in \R^n$, not necessarily convex.The support function is $\sigma_C = \R^n \to \bar{\R}$. $\sigma_C(x) = \sum \{ \langle x,u\rangle  | u \in C\}$.
\includegraphics[scale=0.5]{1_1.png}
\begin{fact}
The support function binds the supporting hyper-plane.
\end{fact}

Supporting functions are
\begin{itemize}
\item Positively homogeneous\\
$\sigma_C(\alpha x) = \alpha \sigma_C(x) \forall \alpha \rangle  0$ \\
$\sigma_C(\alpha x ) = \sup_{u \in C} \langle \alpha x, u\rangle  = \alpha \sup_{u \in C} \langle x, u\rangle  = \alpha \sigma_C(x)$
\item Sub-linear( a special case of convex, linear combination holds $\forall \alpha$.\\
$\sigma_C(\alpha x + (1 - \alpha) y ) = \sup_{u \in C} \langle \alpha x + (1 - \alpha) y,u\rangle  \le \alpha\sup_{u \in C}\langle x,u\rangle  + (1 - \alpha)\sup_{u \in C}\langle y,u\rangle  $
\end{itemize}
\begin{example}[L2-norm]
$\| x \| = \sup_{u \in C} \{ \langle x, u \rangle, u \in \R^n \}$.\\
$\|x \|_p = \sup \{ \langle x, u \rangle, u \in B_q \}$ where $\frac{1}{p} + \frac{1}{q} = 1$. $B_q = \{ \|x \|_q \le 1\}$.\\
The norm is 
\begin{itemize}
\item Positive homogeneous
\item sub-linear
\item If $0 \in C$, $\sigma_C$ is non-negative.
\item If $C$ is central-symmetric, $\sigma_C(0) = 0$ and $\sigma_C(x) = \sigma_C(-x)$
\end{itemize}
\end{example}

\begin{fact}[Epigraph of a support function]
$epi \space \sigma_C = \{ (x,t) | \sigma_C(x) \le t\}$.
Suppose $(x,t) \in epi \space \sigma_C$. Take any  $\alpha > 0$. $\alpha(x,t) = (\alpha x, \alpha t)$.\\
$\alpha \sigma_C(x) = \alpha \sigma_C(x) \le \alpha t$. $\alpha(x,c) \in epi 
\sigma_C$\\
\includegraphics[]{1_2}
\end{fact}

\section{Operations Preserve Convexity of Functions}
\begin{itemize}
\item Positive affine transformation \\
$f_1,f_2,\ldots,f_k \in \space cvx \R^n$.\\
$f = \alpha_1 f_1 + \alpha_2 f_2 + \ldots + \alpha_k f_k$
\item Supremum of functions. Let $\{ f_i \}_{i \in I}$ be arbitrary family of functions. If $\exists x \sup_{j \in J} f_j(x) < \infty \Leftrightarrow f(x) = \sup_{j \in J} f_j(x) $\\
\includegraphics[]{1_3}
\item Composition with linear map.\\
$f \in cvx \R^n$, $A:\R^n \to \R^m$ is a linear map.
$f \circ A (x) = f(Ax) \in cvx \R^n$\\
\begin{align*}
f \circ A (x) & = f(A(\alpha x + (1-\alpha) y)) \\
& = f(A \alpha x + (1-\alpha) A y) \\
& \le \alpha f(Ax) + (a - \alpha) f(Ay)
\end{align*}
\end{itemize}

\end{document}
